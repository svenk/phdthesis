% Definitions from exahype-advanced-hydro.tex
% in math mode: Description before symbol
\newcommand{\desc}[1]{\text{#1}\quad} % -> shall inline
% abbrevations in math mode
\newcommand{\hydro}{\text{HD}}
\newcommand{\mhd}{\text{MHD}}
\newcommand{\adm}{\text{ADM}}
\newcommand{\hd}{\text{HD}}
\newcommand{\srhd}{\text{SRHD}}
\newcommand{\grhd}{\text{GRHD}}
\newcommand{\grmhd}{\text{GRMHD}}
\newcommand{\bmag}{\text{B}}
\newcommand{\magneto}{\text{MD}} % magnetodynamics

% left aligned vectors (pmatrix is center aligned)
\newenvironment{pvector}{\begin{pmatrix*}[l]}{\end{pmatrix*}}

This chapter summarizes efforts of solving the equations of general relativistic
magnetohydrodynamics (GRMHD) with the finite-volume limited ADER-DG scheme
introduced in Chapter~\vref{chapter:numerics}. 
An introduction into the problem and a review of previous work given in
Section~\vref{sec:hd-intro}. This chapter is prepended a motivation abou
the physical modeling of neutron stars spacetimes.
The subsequent Chapter~\vref{chapter:bnslt}
will demonstrate an actual application (beyond academic benchmark scenarios).
This chapter relies partially on the publications \cite{Fambri2018,Koeppel2017}.

\section[
  Motivation: An effective theory for dense and hot matter
]{Motivation: An effective theory for dense and hot nuclear 
matter}\label{sec:hd-nuclear-matter}
The most compact astrophysical object after a black hole is a neutron star.
Having roughly the mass of the sun, the star radius is only at the kilometer
scale, thus only an order of magnitude larger then its Schwarzschild radius.
\footnote{To provide some numbers, the solar mass $\Msol \approx 2 \times 
10^{30}$~kg complies with its Schwarzschild radius $R_\sol \approx 3$~km.
n contrast, typical neutron stars with $M \approx 2 \Msol$
have a radius at the order of $R\approx 10$~km.}. 
Astrophysically, these objects are interesting as the endpoint of a supernovae.
Pulsars, being neutron stars (or white dwarfs) emitting characteristic
highly intensive beams of electromagnetic radiation, are amongst the most
fascinating and best studied compact objects in astronomy.

For an high energy physicist, neutron stars are interesting because they allow 
to probe all fundamental forces: Of course gravity, electromagnetism, but also
the weak and strong force. In
neutron stars, energy and density regimes can be reached which are out of
range for particle accelerators. An understanding of neutron star interiours
thus allows to falsify or set limits in nuclear and fundamental theories.
As an example, within the merger of a binary neutron star system, it is likely
that the quark phase transition can be probed.
\todo[color=green]{HD, Intro: Would nice to have references for QGP excursion.}

In order to describe the physics of a neutron star, the best fundamental
theories available are the standard model of particle physics, written in the
language of special-relativistic quantum field theory, and general relativity,
the (non-quantum) theory of gravity \footnote{In fact, a neutron star does not 
even come close to take backreactions of quantum matter onto the spacetime into
account, and therefore quantum corrections are not required in Einstein field
equations (in contrast as in section~\ref{sec:motivation-qbh}).}.

The typical approach to solve the \emph{spacetime} of a single isolated
neutron star requires taking in an averaged, thermalized
energy-momentum tensor at a given spatial position. This motivates to find a
suitable effective theory (or phenomenological model) for the matter dynamics
which is thermodynamically motivated, leaving the (fundamental) particle
physics picture behind \footnote{Section~\vref{sec:tov} will go into detail of
an ideal fluid model of a neutron star, the TOV solution.}. The physical branch
of fluid mechanics provides a suitable theory, which is (relativistic)
hydrodynamics. 
% It subsumes all physics on micro scales within an \emph{equation
%of state}.

\section{Introduction of Hydrodynamics}
Hydrodynamics adopts two features: Coarse graining, \ie the averaged 
thermodynamic description
of a system, and fast thermalization, \ie the fact that the fluid is locally
assumed to be in a well-defined thermodynamic equilibrium state
\footnote{
 For it's power of describing effective phenomenae, hydrodynamics is sometimes
 called an ``effective theory of everything''.
}.
\emph{Ideal} hydrodynamics is based on the \emph{perfect fluid hypothesis}: 
Dynamical timescales are much larger then viscous or heat transfer timescales. 
Furthermore, the fluid is assumed to be isentropic, \ie there are no prefered 
direction effects. This allows to assume stresses to be isotropic. Furthermore,
there is no heat transport and viscosity and no dissipation. Beyond the ideal
theory, there are (more) realistic theories of hydrodynamics which add
particular features, such Navier-Stokes equations, the classical theory of
viscous fluids \footnote{In this thesis, only ideal fluids are covered, while
 references to extensions are given at some points.}.

The \emph{dynamical timescale} of a gravitational systems is given by
\cite{Rezzolla_book:2013}
\begin{equation}
\tau_\text{dyn} \sim 1/\sqrt{G\bar\rho} \,,
\end{equation}
with an average rest mass density $\bar \rho$ and here the explicit Newton's
constant~$G$. The dynamical time scale of nuclear
matter (nuclear density $\rho_\text{nuc} \approx 2.3\times 10^{17}$kg/m$^3$)
is $\tau_\text{dyn} \sim 2 \times 10^{-4}\text{s}$.
%$\sim 200\mu\text{s} \sim 40 \Msol$.
In comparison, the viscous and heat timescales
are typically of the order of 10$^{+8}$s and therefore can be ignored.
\todo[color=green]{HD Intro: Find reference for viscous/heat timescales}

The relativistic perfect ideal fluid is described by the stress energy tensor
\begin{equation}
T_{\mu\nu} = \rho h u_\mu u_\nu + p g_{\mu\nu}\,,
\end{equation}
with rest mass density $\rho$, 4-velocity $u_\mu=\d{x^\mu}/\d{\tau}$ 
(which is timelike, $u^\mu u_\mu=-1$) and specific internal energy
$\epsilon$ as part of the specific enthalpy $h=1+\epsilon+p/\rho$. The
pressure $p$ subsumes all physics on micro scales within an
\emph{equation of state} $p=p(\rho,\epsilon)$.
The relativistic total energy density is $e=\rho(1+\epsilon)$ and
$g_{\mu\nu}$ is the usual 4-metric.

Widespread simple equations of state are the one-parametric ideal-fluid
(or ``Gamma-law'') equation of state $p=\rho\epsilon(\Gamma-1)$
with the adiabatic / polytropic index $\Gamma$, and the polytropic
EOS $p=\kappa \rho^\Gamma$.

Hydrodynamics can be derived from a number of different fundamental principles,
for instance the principle of minimal action \cite{Andersson2007}
or from kinetic theoryby applying a moment scheme to the Boltzmann equations
\cite{Rezzolla_book:2013}. While a derivation is out of the scope of this
text, the equations of motion as well as the resulting PDEs are discussed
in the subsequent sections. The fluid dynamics are dictated by two conservation
equations for the 4-energy momentum tensor $T^{\mu\nu}$ and for the rest mass
(density flux) $\rho u^\mu$, \ie
\begin{equation}
T^{\mu\nu}=0, \quad \nabla_\mu (\rho  u^\mu) = 0\,.
\end{equation}

Instead of deriving the actual PDEs, the different parts of the 
flux-conser\-vative formulation of general realtivistic magnetohydrodynamics
(GRMHD)  shall be discussed in the next sections \footnote{
  Naturally, such a discussion can either be ordered \emph{top-down}, \ie
  starting from the advanced theory and advancing into the low-energy limit,
  or \emph{bottom-up}, \ie starting with the classical, easy theory and
  advancing by proposing additions neccessary to fulfill relativity, curved
  background and to incorporate Maxwell's theory. Both approaches are well 
  known in theoretical physics. In this text, I decided for
  the \emph{bottom-up} strategy.
}. The state vector of GRMHD,
%\footnote{
%	Remind that the ordering of state vectors has no physical meaning
%	(Section~\ref{sec:conservation-laws}).
%}
\begin{equation}
Q_\grmhd = (Q_\hd, Q_\adm, Q_\magneto) \,,
\end{equation}
is a composite of state vectors which origin from three different theories:
The state vector of hydrodynamics $Q_\hd$ (discussed in this section), 
the curved background metric, collected in the parameter $Q_\adm$ 
(discussed in Section~\ref{sec:grhd}) and the state vector of
magnetodynamics $Q_\magneto$ (Section~\ref{sec:grmhd}). The seperation of
theories by fundamental building blocks is also sketched in Figure
\ref{fig:composite-SVEC} and implemented in the \code{SVEC} GRMHD code.

\subsection{Classical Hydrodynamics}\label{sec:hydro-definitions}
% Some texts come from my Exahype-Engine/ApplicationExamples/GRMHD/doc
\begin{marginfigure}
	\begin{tikzpicture}[node distance=1cm]
	\node (Vhd) {$V_{hydro}$};
	\node[right=0.2cm of Vhd] (maxwell) {$Q_{B}$};
	\node[right=0.7cm of maxwell] (adm) {$Q_{ADM}$};
	
	\node[below of=Vhd] (Qhd) {$Q_{hydro}$}
	edge[<-] (Vhd.south);
	
	\node[below of=Qhd] (mhd) {$Q_{MHD}$}
	edge[<-] (Qhd.south)
	edge[<-] (maxwell.south);
	
	\node[below left of=mhd] (dens) {Densitied $Q_{MHD}$}
	edge[<-] (mhd.south);
	
	\node[below right of=dens] (grmhd) {GRMHD State vector}
	edge[<-] (adm.south)
	edge[<-] (dens.south);
	\end{tikzpicture}
	\caption[
	GRMHD State vector composition, Cartoon (tikZ), \exclusive
	]{State vector composition in GRMHD, or in particular, in
		the \code{SVEC} code.}%
	\label{fig:composite-SVEC}
\end{marginfigure}

%
Hydrodynamics can be introduced by defining the \emph{primitive} and 
\emph{conserved} state vector, where the later is evolved in time and the
former is required for the system closure, given by the pressure/equation of
state. The vectors are given as
\begin{alignat}{4}
\label{eq.hd.prim}
&\begin{array}{l}\text{Primitive}\\\text{vector}\end{array}
&
\quad u &=
\begin{pvector}
\text{rest mass density} \\
\text{velocity} \\
\text{internal energy}
\end{pvector}
= \begin{pmatrix} \rho \\ v_i \\ \epsilon \end{pmatrix}\,,
\\
\label{eq.hd.cons}
&
\begin{array}{l}\text{Conserved}\\\text{vector}\end{array}
&
Q_\hydro
&=
\begin{pvector}
\text{Conserved density} \\
\text{Momentum} \\
\text{Energy density}
\end{pvector}
=
\begin{pmatrix} D \\ S_j \\ E \end{pmatrix}\,,
\\
&
\begin{array}{l}\text{related by}%\\\text{Prim2Cons}
	\end{array}
%\desc{related by Prim2Cons}
~~
&
Q_\hd(u)
&=
\begin{pvector}
\rho \\
\rho v_j \\
\rho/2 v^2 + \epsilon
\end{pvector}\footnotemark\,.
\label{eq.hd.Qu}
\\
\intertext{
Classical hydrodynamics can be expressed in flux-conservative form 
\eqref{intro:conservation-law},
}
	\label{eq.hd.fluxes}
	&\text{with fluxes}
	& F^i_\hydro(u) &=
	\begin{pmatrix}
		\text{fluxes for } D \\
		\text{fluxes for } S_j \\
		\text{fluxes for } E
	\end{pmatrix}
	=
	\begin{pvector}
		\rho v^i \\
		W^i_j \\
		v^i (E + p)
	\end{pvector}
	\,.
\end{alignat}
\footnotetext{Note how easily the primitives $u(Q)=(\rho, S_j/\rho, 
E-\rho/(2v^2))$
	can be analytically recovered from the conserved variables (in contrast to
	relativistic hydrodynamics, presented in the next section). Therefore,
	the fluxes in equation \eqref{eq.grhd.fluxes} are also an algebraic 
	function of
	the conserved state vector only.
}%
The other PDE terms in \eqref{intro:conservation-law} are vanishing.
%
Here, we used the hydrodynamic energy-momentum-tensor (EM tensor), given by
\footnote{
	The EM tensor is also refered to as $W^{ij}=S^{ij}$ in the literature
	(and in Section~\ref{sec.3p1-emtensor}).}%
\begin{equation}\label{eq.hd.emtensor}
W^{ij} = S^i v^j + p \, \delta^{ij}  \, . \,\,
\end{equation}%
%
Classical hydrodynamics is a nonlinear theory. Within the \code{ExaHyPE} code,
it is widely used for performance measurements. Deriving the strong 
hyperbolicity of classical hydrodynamics is part of many books
\cite{Toro09,Rezzolla_book:2013}. The eigenvalues $\lambda_i$ in $k$-direction
in $d$ dimensions are then found to be
\begin{equation}\label{eq.hd.evs}
\lambda_1 = v_k - c,
\quad \lambda_{2,\dots,d+1} = v_k,
\quad \lambda_{d+2} = v_k + c \,,
\end{equation}
where $c=\sqrt{\left(\partial_\rho p\right)_s}$ is the \emph{sound speed} of 
the fluid. The three different wave speeds stand out as different waves in the
Riemann problem (Section~\ref{sec:riemann-problem}).

\subsection{Special relativistic hydrodynamics (SRHD)}
Special relativistic hydrodynamics is the extension of classical
hydrodynamics valid for high velocities ($v\to c$), but remaining on flat
background spacetime.

The 3-velocity $v^i$ of the fluid is extended in favour of the
Lagrangian 4-velocity $u^\mu = (W,W v^\mu)$, with Lorentz factor $W=(1- 
v^2)^{-1/2}$
\footnote{
	In the literature, alternative popular namings are $W=\Gamma$, $E=U$.
	In any case, do not confuse $W$ with the conformal factor used in
	some conformal formulations of Einstein Equations (see Section~\ref{sec:conformal-factor}).
}.
The 4-momentum/rest-mass current is then given by $S^\mu = \rho u^\mu$.
The content of the primitive \eqref{eq.hd.prim} and conserved vector
\eqref{eq.hd.cons} does not change. However, the relationship \eqref{eq.hd.Qu}
is replaced by
\begin{equation}
	Q_\srhd(u)
	=
	\begin{pvector}
	\rho W \\
	\rho h W^2 v_j \\
	\rho h W^2 - p - \rho W
	\end{pvector}
	\label{eq.srhd.Qu}
\end{equation}
with specific enthalpy $h=1+\epsilon+p/\rho$
\footnote{Some groups prefer to evolve $Q_\srhd=(D,S_j,\tau)$ instead
	of $Q_\srhd=(D,S_j,E)$ with $\tau=E-D$ the rescaled energy density.
	In fact we will also stick to this convention.
}.
The conservative formulation of SRHD is then given by is then given by the 
fluxes
\begin{align}
	\label{eq.srhd.fluxes}
	F^i_\hydro(u) &=
	\begin{pmatrix}
		\text{fluxes for } D \\
		\text{fluxes for } S_j \\
		\text{fluxes for } \tau
	\end{pmatrix}
	=
	\begin{pvector}
		D v^i \\
		W^i_j \\
		S^i - v^i D
	\end{pvector}
	\,.
\end{align}
The SRHD equations are hyperbolic for causal EOS \cite{Anile_book,Font08}. The
wave speeds decompose similarly as in \eqref{eq.hd.evs}, where however the
relationship to the speed of sound $c$ is nonlinear \cite{Font08},
\begin{equation}
\lambda_\pm
=
\frac{v^k (1-c^2) \pm c \sqrt{(1-v^2)(1-v^2c^2 - v_k^2 (1-c^2))}}{1-v^2 c^2}
\end{equation}
and $\lambda_- := \lambda_1, \lambda_+ := \lambda_{d+2}$ in the notation of
\eqref{eq.hd.evs}.

% Note that if you add terms
%to the energy-momentum tensor (such as the magnetic terms), also $S_j$
%will change (c.f. \eqref{eq.grhd.Qu} to XXX).

\subsection{The primitive recovery in relativistic hydrodynamics}\label{sec:c2p}
Due to the nonlinear Lorentz factor $W=W(v^2)$, the recovery of the primitive
variables \eqref{eq.hd.prim} from the conserved ones \eqref{eq.hd.cons} is no 
more possible analytically in SRHD. Instead, the inverse of \eqref{eq.srhd.Qu}
can be approximated by numerical root finding. The standard approach is to
solve a single or multiple nonlinear equations. For instance, the authors of
\cite{DelZanna2007,Noble2006} propose to solve a simple nonlinear $2 \times 2$
system which reads for SRHD as
\begin{equation}
\begin{rcases}
  y^2 x - S^2 &= 0 \\
  y - p - E   &= 0 
\end{rcases}\,,
\end{equation}
with $x=v^2, y=\rho h W^2$. Once
$x$ and $y$ are determined numerically, all primitives can be recovered by
computing $W=(1-x)^{-1/2}$, $\rho=D/W$, $v_j=S_j/y$,
$h=S_j/(v_j\rho W^2)$ and for instance with the ideal gas (from above), 
$\epsilon=h/(1+W)$\footnote{Obviously any non-analytic EOS raises new issues at
	this point which are again subject to numerical treatment.}.


\section{General relativistic hydrodynamics (GRHD)}\label{sec:grhd}
General relativistic hydrodynamics is the theory which described fluids moving
within an (external) gravitational potential. The (special) relativistic fluid
follows the definitions from the previous section. The coupling of the
background spacetime is mediated by a source term in the law of motion.

A fully general relativistic description implies also matter backreaction on
the spacetime (the fluid bends spacetime itself), according to Einstein's field
equations (EFE, Chapter~\ref{chapter:gr}). In this case, the EFE determine the
dynamics of the geravitational field under the presence of a source term
(the energy momentum tensor of the fluid). Section \ref{sec:Cowling-approx}
discusses the simplifications which can be made to the GRMHD equations
when the backreaction is neglected.

\subsection{3+1 split of special relativistic hydrodynamics}
The ``Valencia formulation'' of GRHD presented in this thesis dates back to the
pioneering work of Mart\'i et al. \cite{Marti91, Banyuls97, Ibanez01} in 1991.
They where the first to make a characteristic approach to relativistic
hydrodynamics in a 3+1 split of spacetime \footnote{
see Section~\vref{sec:adm-foliation} for the definitions of normal vector 
$n^\mu$, lapse $\alpha$, shift $\beta^i$ and 3-metric $\gamma_{ij}$.}.

In the 3+1 split,
the Lagrangian velocity $u^\mu$ can be casted as $u^\mu = W(n^\mu + v^\mu)$ and
the 3-velocity as $v^i  =  u^i/W + \beta^i/\alpha$. The 3+1 language also offers
beatiful interpretations, for instance is the projection of the fluid 4-velocity
on the purely spatial hypersurfaces just the Lorentz-decorated 3-velocity,
$\gamma^\mu_\nu u^\nu = W v^\mu$. Formally, the 3-energy-momentum-tensor can be
interpreted as extension from the flat case,
\begin{equation}\label{eq.wij.grhd}
W_{\hydro}^{ij} = W^{ij}_\srhd = S^i v^j + p \delta^{ij}
~\leftrightarrow~
W^{ij}_\grhd = S^i v^j + p \gamma^{ij}
\,.
\end{equation}
The SRHD fluxes are refined as
\begin{align}
\label{eq.grhd.fluxes}
F^i_\hydro(u) &=
\begin{pmatrix}
\text{fluxes for } D \\
\text{fluxes for } S_j \\
\text{fluxes for } \tau
\end{pmatrix}
=
\begin{pvector}
D w^i \\
\alpha W^i_j - \beta^i S_j \\
\alpha (S^i - v^i D) - \beta^i \tau
\end{pvector}
\quad
\text{(general $W^{ij}$)}
\\
&=
\begin{pvector}
D w^i \\
S_j w^i + p \delta^i_j \\
\tau w^i + p v^i
\end{pvector}
\hspace{2.7cm} % todo: push right
\text{(only $W^{ij}=S^i v^j + p\gamma^{ij}$)}
\label{eq.grhd.flux2}
\end{align}
It is convenient to write the fluxes with the vector
$w^i=\alpha v^i - \beta^i$ which is refered to as the advection
velocity relative to the coordinates, or just \emph{transport
	velocity}\footnote{Note that $w^i$ does not transform like a 3-vector. However,
	we introduce it only for abbreviation on the
    paper and for saving contractions in the computer.}. With the replacment
of $v^i \to w^i$, the GRHD fluxes \eqref{eq.grhd.flux2} have (almost) a similar
shape as the SRHD fluxes \eqref{eq.srhd.fluxes}.

In order to fully describe the state of a GRHD system, the local curvature must
be encoded in the state vector
\footnote{The ADM state is constant for the GRHD PDE, as the modification of the
	background spacetime is prescribed by Einstein field equations, which are 
	however
	a PDE on their own (Chapter~\ref{chapter:gr}). In \code{ExaHyPE} language, 
	an
	entry in the state vector which is not evolved by the PDE, \ie which 
	fulfills
	$\partial_t Q^k=0$, is called a ``material parameter''. This term comes from
	seismology where non-evolved parameters describe the immutable soil 
	properties
	(in GR lingua, the ``background'' spacetime) which are unchanged by the 
	waves
	described by the PDE.
}. Therefore, the ADM state vector is defined as
\begin{equation}\label{eq.grhd.adm}
Q_\adm = (\alpha, \beta^i, \gamma_{ij}, K_{ij}) \,.
\end{equation}
%
The GRHD system is hyperbolic \cite{Font00}, and the eigenvalues in $k$ direction
are given by
\begin{align}
 \lambda_0 &= \alpha v^k - \beta^k \,, \\
 \lambda_\pm &=
  \frac{\alpha}{1 - v^2 c_s^2}
  \big(  v^k (1-c_s^2) \nonumber\\
  &\phantom= \pm c_s
  \sqrt{(1-v^2)\left( \gamma^{kk}(1-v^2 c_s^2) - v^k v^k (1-c_s^2) \right)}
  \big) - \beta^k \,, 
\end{align}
with the local sound speed $c_s=\sqrt{\partial_\rho p + p/\rho^2 \partial_\epsilon p}/h$.

\subsection{Conformal factor}
In the Valencia formulation, the PDE system is written in terms
of tensor densities. The determinant of the metric
$\gamma=\text{det}(\gamma_{ij})$ relates the tensor densities with
an ordinary tensor. The 3-determinant is related to the determinant of
the four-metric by $\sqrt{-g}=\alpha\sqrt{\gamma}$. The system is
then formulated as\footnote{Here the full system \eqref{intro:ncp}
  is given, even if some terms are zero for certain systems
  (such as the algebraic and differential source for HD and SRHD).
}
\begin{equation}\label{eq:conformal-pde-grmhd}
\partial_t (\sqrt{\gamma} Q)
+ \partial_i (\sqrt{\gamma} F^i)
+ \sqrt{\gamma} B^i \partial_i (\sqrt{\gamma} Q)
= \sqrt{\gamma} S.
\end{equation}
The PDE as given in \eqref{eq:conformal-pde-grmhd} defines the
state $Q$, the fluxes $F^i$, the non-conserva\-tive part $B^i$ and the
algebraic source $S$ \emph{without} the factor $\sqrt{\gamma}$.
This convention will be retained for the following sections. Note that,
however, the vector $\sqrt \gamma Q$ is evolved in time.

\subsection{Sources for curved spacetimes}
In presence of a curved background, the spacetime coupling introduces
a source. The governing equations of GRHD are therefore a balance law
\eqref{intro:balance-law} instead of a conservation law
\eqref{intro:conservation-law}. The source term induced by the
spacetimes, written \emph{with Christoffel symbols}, read
\footnote{Note the use of 4-dimensional tensors except the lapse $\alpha$.
  See also Appendix~\ref{apx:symbols} for the standard definitions of
  the Christoffel symbols of first kind $\Gamma^k_{ij}$ and 4-metric
  $g_{\mu\nu}$.
}
\begin{equation}
S_\hydro(u) =
\begin{pvector}
\text{source for } D \\
\text{sources for } S_j \\
\text{source for } \tau
\end{pvector}
=
\begin{pvector}
0
\\
T^{\mu\nu} \partial_\mu g_{\nu j} - \Gamma^\delta_{\nu\rho} g_{\rho j} T^{\mu\nu}
\\
\alpha T^{\mu 0} \partial_\mu \ln \alpha - \alpha T^{\mu\nu} \Gamma^0_{\mu\nu}
\end{pvector}.
\end{equation}
In order to determine the 4-energy-momentum-tensor
\begin{equation}\label{eq.grhd.tmunu}
T^{\mu\nu} =
\rho h W^2 (n^\mu + v^\mu) (n^\nu + v^\nu)
   + p(\gamma^{\mu\nu} - n^\mu n^\nu)
\,,
\end{equation}
the 4-velocity $u^\mu$ or the normal vector $n^\mu$ have to be recovered.
This can be circumvented by using spatial tensors only: The equivalent
``Christoffel symbol free'' source terms, as used in \cite{Radice2013c},
read
\begin{equation}\label{eq:grmhd.source}
S_\hydro(u) =
\begin{pvector}
0
\\
\frac \alpha2 S^{lm} \partial_j \gamma_{lm} + S_k \partial_j \beta^k - E \partial_j \alpha
\\
\textcolor{red}{ \alpha S^{ij} K_{ij} } - S^i \partial_i \alpha
\end{pvector}
\,.
\end{equation}
The sources can also be written without computing the 
3-energy-momentum-tensor $S^{ij}$, exploiting
$\left(\partial_i \sqrt\gamma\right) / \sqrt\gamma = \frac 12 \gamma^{lm} 
\partial_i \gamma_{lm}$,
\footnote{
  However, \eqref{eq:grmhd.source2} requires to compute
  $\partial_i \sqrt \gamma$, which must then be (formally) part of the
  state vector if the non-conservative product approach is chosen.
}
\begin{equation}\label{eq:grmhd.source2}
S_\hydro(u) =
\begin{pvector}
0
\\
\frac \alpha2 S^l v^m \partial_j \gamma_{lm} + \alpha p \frac 1{\sqrt{\gamma}} 
\partial_j \sqrt{\gamma} + S_k \partial_j \beta^k - E \partial_j \alpha
\\
\textcolor{red}{
	\alpha S^l v^m K_{lm} - \alpha p \gamma^{lm} K_{lm}
} - S^i \partial_i \alpha
\end{pvector}
\end{equation}
All terms in the source except the red one(s) contain derivatives of the ADM
state vector \eqref{eq.grhd.adm}. This motivates to extend the definition of
the curvature state vector as
\begin{equation}\label{eq.grhd.admTilde}
\tilde Q_\adm = (
   \alpha, \beta^i, \gamma_{ij}, K_{ij},
   \partial_i \alpha, \partial_i \beta^j, \partial_i \gamma_{jk}
   ) \,.
\end{equation}
If \eqref{eq.grhd.admTilde} is chosen in favour of \eqref{eq.grhd.adm}, then
the GRHD source term is a purely algebraic one. In preparation of a coupled
evolution of GRHD with spacetime, either $Q_\adm$ or $\tilde Q_\adm$ must be
evolved in time. In fact, all quantities \eqref{eq.grhd.admTilde} are part of
the FO-CCZ4 state vector \eqref{eq:foccz4-state-vector}, proposed in
Section~\ref{sec:fo-ccz4}. Therefore, the GRHD part in a coupled evolution of
the presented GRHD system with the FO-CCZ4 formulation is of the form
\begin{equation}
\partial_t Q_\hd + \partial_i F^i_\hd(Q_\hd, \tilde Q_\adm)
 = S_\hd(Q_\hd, \tilde Q_\adm) \,.
\end{equation}
In contrast, an ordinary second order formulation of EFE would instead evolve
only $Q_\adm$ in time, \ie without the derivatives. In such a case, the 
differential
and algebraic split of the source term (Section~\ref{sec:ncp}) is applicable.
Part of the work carried out in \cite{Fambri2018} is to cast the PDE as
\eqref{intro:ncp}, \ie to cast all non-red summands as 
$B^{ij}_k(Q_\adm, Q_\grhd) \partial_i (Q_\adm)_j$.

In order to cast these PDEs into the algebraic-differential source
split \eqref{intro:ncp}, all black terms obtain an extra minus when they
are moved from the RHS to the LHS. The final GRHD system reads
then (colors now omitted):
\begin{align}
\left( B_\grhd \right)^{ij}_{k} \partial_i Q_j =%&=
\begin{pvector}
0
\\
- \frac \alpha2 S^{lm} \partial_j \gamma_{lm} - S_k \partial_j \beta^k + E \partial_j \alpha
\\
S^i \partial_i \alpha
\end{pvector} \,,
%\\
\quad
S_\grhd =%&=
\begin{pvector}
0
\\
0
\\
\alpha S^{ij} K_{ij}
\end{pvector} \,.
\end{align}

\subsection{Cowling approximation}\label{sec:Cowling-approx}
In case of a stationary spacetime (Cowling approximation
\cite{Cowling41}, characterized by $\partial_t g_{ij} = 0$),
one can simplify the source terms and
get rid of the contraction \citep{MTW1973, York79, Gourgoulhon2012}
\begin{equation}\label{eq:path-to-cowling}
S^{ij} K_{ij} = \frac{1}{2\alpha} S^{ik} \beta^j \partial_j
\gamma_{ik} + \frac 1\alpha S^j_i \partial_j\beta^i
\,.
\end{equation}
In this particular case, \eqref{eq:grmhd.source} simplifies to
\begin{equation}
S_{\hydro\text{,Cowling}}(u) =
\begin{pvector}
0
\\
\frac \alpha2 S^{lm} \partial_j \gamma_{lm} + S_k \partial_j \beta^k - E \partial_j \alpha
\\
\frac 12 S^{ik} \beta^j \partial_j \gamma_{ik} + S^j_i \partial_j \beta^i - S^i \partial_j \alpha
\end{pvector}
\,.
\label{eq.grmhd.cowling.source}
\end{equation}
In this special case, it is possible to write the GRHD equations without any
algebraic source term. All terms in \eqref{eq.grmhd.cowling.source} can
then be casted in the form $B\partial_i Q$. This allows to split the PDE as
\footnote{
  In the form \eqref{eq.grhd.as-cowling}, the flow on a background spacetime is 
  drescribed
  similiarly as in the shallow-water equations,
  where the bottom-slope term (which accounts for gravitational forces)
  can also be cast as a non-conservative product
  \cite{Pares2006,Castro2006,Castro2010}.  
}
\begin{equation}\label{eq.grhd.as-cowling}
\partial_t Q^k + \underbrace{\partial_i F^i(Q)}_\text{Hydro. contribution}
+ \underbrace{B^i_{jk} \partial_i Q_j}_\text{Background contribution}
= 0
\,.
\end{equation}
In the numerical schemes proposed in Section~\ref{sec:dg}, the absence of
an algebraic source is desirable, as it supports the well-balanced property
of these schemes.

\section{General relativistic magnetohydrodynamics (GRMHD)}\label{sec:grmhd}

The general relativistic magnetohydrodynamics (GRMHD) equations are
the consequence of the coupling of Euler equations 
(Hydrodynamics, GRHD) to Maxwell equations (Magnetodynamics, MD). In the
popular ideal MHD approximation, the electric field $\vec E = \vec B \times \vec v$
is fully determined by the moving fluid.
In this approximation, the Faraday tensor $F^{\mu\nu}$ (with 6 degrees of freedom,
$\vec E$  and $\vec B$) reduces to the magnetic field $\vec B$ only, therefore the
vector of conserved variables in Magnetodynamics is just $Q_\magneto = (B^i)$.
\footnote{We add further evolution equations in case of the divergence
   cleaning technique. However, note that there is no distinction between
   primitive and conserved variables in Magnetodynamics.}
This approximation is appropriate to describe a wide variety of astrophysical
phenomena where the electrical conductivity of the plasma (description
of matter) is very high. In the ideal MHD approximation, the electrical
conductivity $\sigma\to\infty$ is assumed to be divergent. The electrical field is
completely determined by the fluid velocity and the magnetic field.
The magnetic flux $\phi_B = B_i S^i$
over any surface $\boldsymbol S$ is conserved,
\begin{align}
\oint_{\partial \boldsymbol{S}} \left( \boldsymbol{E} +\boldsymbol{v}
\times \boldsymbol{B}\right) \cdot d\boldsymbol{\ell} = - \frac{d
	\phi_B}{d t} = 0\,,
\end{align}
and is advected with the fluid movement. The magnetic contribution to the
hydrodynamics equations, \ie the MHD equations, is then just a
conservation equation for the magnetic field.

\subsection{Magnetodynamics}
The photon field (Maxwell in vacuum, \ie without charge carriers) has the momentum
density $\boldsymbol S = \boldsymbol E \times \boldsymbol B$
(Poynting vector), total energy density $U = \frac{1}{2}\left(E^2 + B^2\right)$
and energy-momentum tensor $W^{jk}=U \gamma^{jk} - E^j E^k - B^j B^k$.

The electric field $\boldsymbol{E}$ in the Eulerian frame 
is determined by the simplified Ohm's law \ie $E_i = -
\tilde{\epsilon}_{ijk} v^j B^k$ in the ideal MHD limit
(\ie for diverging electrical conductivities).
The cross product is given by the spatial three-Levi-Civita tensor
density $\tilde{\epsilon}$ (see Appendix~\ref{apx:symbols} for the definition).
Therefore, the momentum density and energy momentum tensor can be expressed,
using only $\boldsymbol v$ and $\boldsymbol B$, as
\begin{align}
S^\magneto_i %= \left(\boldsymbol{E}\times\boldsymbol{B}\right)_i
&= \tilde{\epsilon}_{ijk} E^j B^k = -\tilde{\epsilon}_{ijk} \tilde{\epsilon}^{jmn} v_m B_n B^k
= v_i \left(B_k B^k \right) - B_i \left( v_k B^k\right)
\,,
\nonumber
\\
  W^{jk}_\magneto
   &=  U \gamma^{jk} - B^j B^k / W^2 - (B^k v_k) v^j B^k \,.
% = \boldsymbol{v} B^2 - \boldsymbol{B} \left(\boldsymbol{v}\cdot \boldsymbol{B}\right) .
\label{eq.md.contributions}
\end{align}

\subsection{The GRMHD coupling}
The Maxwell theory (Magnetodynamics) influence the hydrodynamic flow with the (energy)
momentum contributions \eqref{eq.md.contributions}, while the Euler theory
determines the electrical field in the presented ideal magnetodynamic approximation.
Furthermore, the pressure $p$ get's a magnetic contribution and is replaced in
\eqref{eq.wij.grhd} and \eqref{eq.grhd.tmunu} as
\begin{equation}
p \to p_\text{tot} = p_\hydro + p_\magneto
\quad\text{with}\quad
p_\magneto = \frac 12 \left( B_j B^j / W^2 + (B^j v_j)^2 \right)
\end{equation}
being the pressure contribution from ideal magnetodynamics.
The total energy-momentum tensor is the sum of all involved theories,
thus the GRM\-HD energy momentum tensor is given, for completeness, here as
\begin{equation}
\begin{aligned}
	W^{ij}_\grmhd &= W^{ij}_\hydro + W^{ij}_\magneto = W^{ij}_\hydro(p_\hydro + p_\magneto) + W^{ij}_\magneto
	\\
	&= 
	S^i v^j + p_\text{tot} \gamma^{ij} - \frac{B^i B^j}{W^2} - (B^k v_k)v^i B^j
	\,.
\end{aligned}
\end{equation}
Consequently, also the conserved quantities are given by
\begin{alignat}{3}
%	\desc{GRMHD P2C}
	&
	Q_\grmhd(u, Q_\magneto)
	&&=
	\begin{pvector}
		D \\
		S_j \\
		\tau
	\end{pvector}
	=
	\begin{pvector}
		\rho W \\
		D h W v_j + B^2 v_j - (B^i v_i) B_j \\
		D (h W -1) - p + \frac 12 \left( B^2 (1 + v^2) - (B^j v_j)^2 \right)
	\end{pvector}
	\label{eq.grhd.Qu}
\end{alignat}

\subsection{Fluxes and sources}
The evolution variables in GRMHD are
\todo[color=green]{MHD: Adress the issue that we
	evolve the \emph{covariant} momentum $S_j$ and the
	\emph{contravariant} magnetic field $B^j$. Why?}
\begin{equation}
Q_\mhd = (Q_\hydro, Q_\magneto) = (D, S_j, \tau, B^j)
\,.
\end{equation}
It should be emphasized that the vector of primitive variables \eqref{eq.hd.prim}
does not change (increase in size), the primitive recovery takes
only place for the hydrodynamic part of the GRMHD equations.
%
The MD conserved vector is just $Q_\magneto=(B^j)$ and its evolution is given
by the induction equation
\begin{equation}\label{eq.grmhd.pdeb.simple}
\partial_t B^j + \partial_i (w^i B^j - B^i w^j) = 0 \,,
\end{equation}
\ie the new PDE has fluxes and sources
\begin{equation}
F^i(Q) = w^i B^j - B^i w^j,
\quad S(Q) = 0 \,.
\end{equation}

The GRMHD equations are hyperpolic. For the characteristic wave speeds
in GRMHD, a popular choice is the magnetosonic approximation \cite{Gammie03}.
%
The evolution equation \eqref{eq.grmhd.pdeb.simple} does not handle the magnetic
field divergence, which is covered in the next section.

%
%\subsection{The pure GRMHD equations}
%In consequence, the simple tensor-free fluxes of GRHD as given in
%\eqref{eq.grhd.fluxes} are no more correct. Therefore, one typically
%employs the fluxes with the energy-momentum-tensor. We repeat the first
%three entries for the GRHD and give provide the new flux for the magnetic
%field in the pure GRMHD (\emph{without} divergence cleaning) as the 
%fourth entry:
%\begin{equation}
%F^i(u) =
%\begin{pmatrix}
%\text{fluxes for }D \\
%\text{fluxes for }S_j \\
%\text{fluxes for }D \\
%\text{fluxes for }B^j
%\end{pmatrix}
%=
%\begin{pmatrix}
%w^i D
%\\
%\alpha W^i_j - \beta^i S_j
%\\
%\alpha (S^i - v^i D) - \beta \tau
%\\
%\end{pmatrix}.
%\end{equation}
%The source terms for the magnetic fields $B^j$ are zero, similar to the
%source terms for the density $D$.
%
%This formulation is useful if one uses a technique to control the 
%Maxwell constraint $\partial_j B^j = 0$ which does not change the
%PDE system, for instance contraint transport.

\subsection[Divergence cleaning]{The divergence cleaning (constraint damping) 
formulation}\label{sec:divergence-cleaning}
The Maxwell magnetic monopole constraint $\partial_i B^i=0$ can be
casted as hyperbolic conservation law with a Generalized Lagrangian
Multiplier approach (GLM) also refered to as \emph{divergence cleaning},
initially proposed by \cite{Dedner:2002}. In this approach, the MHD
system is augmented with an additional auxiliary equation for an
artificial scalar field $\psi$ in order to propagate away numerical
violations of the divergence-free constraint\footnote{
  There are in fact other techniques to ensure in the divergence
  freedom of the magnetic field on a numerical level, such as
  constrained transport.
}.
\todo[color=green]{MHD: Tell more about constrained transport methods.}
Hence, the PDE \eqref{eq.grmhd.pdeb.simple} is replaced by two
different PDEs.
The MD conserved vector is now $Q_\magneto = (B^j, \phi)$. The
modified fluxes and sources read
\cite{Porth2017,Palenzuela:2008sf, Dionysopoulou:2012pp}
\begin{align}
	%\desc{fluxes}
	F^i(Q) &= \begin{pmatrix}
		\text{fluxes for }B^j \\
		\text{fluxes for }\phi
	\end{pmatrix}
	=
	\begin{pvector}
		w^i B^j - v^j B^i - B^i \beta^j \\
		\alpha B^i - \phi \beta^i
	\end{pvector}
	\\
	%\desc{and sources}
	S(Q) &=
	\begin{pmatrix}
		\text{sources for } B^j \\
		\text{sources for } \phi
	\end{pmatrix}
	=
	\begin{pvector}
		-B^i \partial_i \beta^j - \alpha \gamma^{ij} \partial_i \phi \\
		\textcolor{red}{- \alpha \kappa \phi} - \phi \partial_i \beta^i
		- \frac 12 \phi \gamma^{ij} \beta^k \partial_k \gamma_{ij}
		+ B^i \partial_i \alpha
	\end{pvector}
\end{align}
%
Here, $\kappa$ is the damping term which controls the amount of damping applied on the 
field $\phi$. In the sake of a sane balance law with purely differential 
source terms, $\kappa=0$ is a choice also carried out in \cite{Fambri2017}. This
choice reflects pure transport and no damping.

Notably, the divergence cleaning formulation introduces additional sourc\-es
which are mostly non-conservative (as in the case of GRHD) except of the
damping term (displayed in red). Thus, from all presented equations, only
a Cowling-GRMHD formulation with divergence cleaning and $\kappa=0$ has
zero algebraic sources.

