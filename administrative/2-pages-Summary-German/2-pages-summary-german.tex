% 1-2 pages summary of SvenK phd thesis in German language,
% as standalone document (not part of the thesis). This is required
% by the Pruefungsamt Naturwissenschaften
%
\documentclass[a4paper,12pt]{article}

\usepackage[utf8]{inputenc}
\usepackage[german]{babel}
\usepackage[a4paper]{geometry}
\usepackage{hyperref,a4wide}

\title{
	High-order methods in fully general-relativistic hydrodynamics and magnetohydrodynamics
	\\[0.5em]\large {\it Deutschsprachige zweiseitige Zusammenfassung}
}
\author{Sven Köppel}
\date{} % blank

\begin{document}

\maketitle

Schwarze Löcher und Gravitationswellen gehören zu den faszinierenden Vorhersagen
der allgemeinen Relativitäts\-theorie. Schwarze Löcher gibt es auf fast jeder
Längenskala und seit einigen Jahren gibt es mehr und mehr Möglich\-keiten, sie
zu messen. Das Laser-Interferometer-Gravitationswellen-Observatorium (LIGO)
konnte 2016 zum ersten Mal Gravitationswellen direkt beobachten, die durch den
Kollaps zweier stellarer schwarzer Löcher entstanden sind. Kurze Zeit später 
wurden
die Gravitationswellen von der Verschmelzung zweier Neutronensterne gemessen.
Das Signal bekam den Namen GW170817.
Neutronensterne gehören zu den kompaktesten astrophysikalischen Objekten,
mit einer Ausdehnung von wenigen Kilometern und einer mit der Sonne vergleichbaren
Masse. Die zentrale Massendichte in Neutronensternen erreicht ein Vielfaches
der nuklearen Sättigungsdichte. Damit sind Neutronensterne ein Ort, in dem sich
extrem heiße und kompakte Materie findet. Dank der
neuen Ära der Multimessenger-Astronomie können erstmals präzise Aussagen über 
die Beschaffenheit dieser Sterne gemacht werden.

Eine zentrale Rolle spielt dabei die computerbasierte Lösung von Einsteins
allgemeiner Relativitätstheorie (ART). Die numerische Relativitätstheorie ist ein
junges Feld. Erst vor 15 Jahren wurde mittels der Punktierungsmethode
der Durchbruch beim Zweikörperpro\-blem der ART erlangt. Seitdem gab es 
gewaltige
Fortschritte, die unter anderem den stetig verbesserter 
Anfangs-Randwertproblem-Formulierungen der ARt zuzuschreiben sind.
Darüber\-hinaus profitiert die numerische Relativitäts\-theorie von zunehmend
besser konvergierenden numerischen Methoden, welche allgemein
unter dem englischen Stichwort ``high order methods'' zusammengefasst werden.

Im Rahmen dieser Dissertation werden die neuartigen sogenannten 
diskontinuierlichen Galerkin-Verfahren in einem ebenfalls neuartigen adaptiven
Gitterverfeinerungscode auf astrophysikalische Probleme angwandt. Dabei handelt 
es sich um generische Verfahren, um gekoppelte Systeme von hyperbolischen 
partiellen Differentialgleichungen (PDG) zu lösen. Um nicht an
Unstetigkeitsstellen, etwa
in der Materie-Verteilung oder Singularitäten der Raumzeit, zu begegnen,
fällt das Schema als Prädiktor-Korrektor-Verfahren auf ein klassisches
robustes Finite Volumen-Verfahren zurück.

Ein weiterer Gegenstand dieser Arbeit ist die Herleitung einer
neuen Formulierung der Einsteingleichungen im $(3+1)$-Split der Raumzeit, die 
nur Ableitungen erster Ordnung in Zeit \emph{und} Raum aufweist. Es kann
gezeigt werden, dass das System aus 59 gekoppelten nichtlinearen PDGs streng
hyperbolisch ist und damit für eine numerische Zeitentwicklung sehr
gut geeignet. Zur Demonstration der Korrektheit der Formulierung sowie ihrer
Lösbarkeit mit den dargestellten numerischen Schemata werden einige
Standardtests der numerischen Relativitätstheorie erfolgreich demonstriert.

Zur Beschreibung der Materie in astrophysikalischen Raumzeiten wurden in einem
weiteren Projekt die Gleichungen der allgemein relativistischen 
Magnetohydrodynamik (engl. GRMHD, für ``general relativistic
magnetohydrodynamics'') an die benötigte
Formulierung angepasst. Mit den Gleichungen der GRMHD lassen sich eine Vielzahl
astrophysikalischer Phänomene beschreiben, etwa Jets, Pulsare und
Gammastrahlenblitze. In Benchmarks wird die Formulierung auf stationären
Raumzeiten (Cowling-Näherung) getestet.

Ein viertes Projekt ist eine Anwendung der vorangegangenen
Techniken, also der numerischen Zeitintegration des gekoppelten
Einstein-Euler-Systems. Dabei wird das Szenario eines
``verzögerten Kollapses'' bei der Vereinigung zweier Neutronensterne untersucht.
Es werden quantitative Kriterien entwickelt, um einen sofortigen Kollaps
zu einem schwarzen Loch von der Entstehung eines metastabilen hypermassiven
Neutronensternes zu unterscheiden.
Dazu wird eine Parameterstudie
durchgeführt, bei der für eine Vielzahl an realistischen nuklearen
Zustandsgleichungen eine Menge an Anfangswertprobleme gebildet werden, die sich
durch die Masse ihrer Konstituenten unterscheiden.
Bei der Untersuchung stellte sich heraus, dass die
dynamische Eichfixierung dank ihrer Eigenschaft, die Koordinaten aus dem
Gravitationspotential herauszutreiben, am besten dafür geeignet ist, den
Zeitpunkt der Verschmelzung und des Kollapses zu definieren. Als Ergebnis
kann ein nichtlinearer Zusammenhang zwischen Zustandsgleichungen und
Überlebenszeit hergestellt werden. Mithilfe von Messwerten von GW170817
kann ferner eine die minimal mögliche Ausdehnung eines Neutronensternes
ermittelt werden, diese untere Radius-Schranke beträgt $R \geq 9.74(\pm 0.1)$km.

Als letztes Projekt ist dieser Monographie ein Projekt aus dem
Bereich der Quantengravitation beigelegt. Es handelt sich dabei um die
Modellierung von schwarzen Mikro-Löchern mittels einer stringtheoretisch
motivierten impulsabhängigen Modifikation der heisenbergschen
Unschärferelation. Die neuen
Kommutatorrelationen sorgen für exakte Impulsoperatoren auf Kosten einer
minimalen Länge im Ortsraum.
Im Rahmen der Dissertation wurden mögliche Erweiterungen der Theorie 
auf große Extradimensionen untersucht. 
Zwei Erweiterungen stechen dabei heraus: Eine, welche in
$(4+1)$-dimensionaler Raumzeit eine konische Singularität aufweist, ist
die erste exakte Lösung einer Raumzeit, welche auf kurzen Skalen wie ein
gravitativer Monopol aussieht und auf großen Skalen wie ein schwarzes Loch.
%
Eine weitere modifizierte Unschärferelation wiederum reproduziert in
jeder Dimension die gleiche Impulsraum-Regularisierung, verfügt darüber
hinaus aber über einen neuen komplexeren thermodynamischen Zustandsraum.
Dabei handelt es sich um das Phänomen, dass das schwarze Loch kurz vor seiner
Verdampfung auf der Planck-Skala Temperaturoszillationen aufweist,
welche mit wiederholten Phasenübergängen zwischen negativer und positiver
Wärmekapazität einhergehen. Die damit verbundene stetig veränderte
Luminosität prägt den Begriff ``Leuchtturm-Effekt''.



	
\end{document}