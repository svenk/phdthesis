% Currently an intermediate text reservoir

\section{Extracting gravitational waves}
\label{sec:extracting-gws}

This appendix section is a tiny review of standard definitions in gravitational
wave definitions (continuum) and their extraction in numerical simulations.
For reviews, see \eg \cite{Centrella:2010,Bishop2016,Baumgarte2010b}.

\subsection*{Definitions}

Gravitational waves are defined within linearized gravity, \ie where the metric
$\gamma_{\mu\nu} = \eta_{\mu\nu} + h_{\mu\nu}$ is the sum of the (Minkowski)
vacuum $\eta_{\mu\nu}$ and a perturbation $h_{\mu\nu}$ which is small
($|h|\ll 1$). The wave equations for the perturbation are found after defining
the trace reversed perturbation $\bar h = h - \frac 12 \eta 
h_\alpha^\alpha$ and imposing the Coulomb gauge $\nabla_\mu \bar h^{\mu\nu}=0$;
they are given by the EFE $\nabla_\alpha \nabla^\alpha \bar h_{\mu\nu}
= \Box \bar h_{\mu\nu} = 0$, with the covariant d'Alambert operator $\Box$.
The remaining gauge freedom allows to fix transverse metric perturbations
(yielding in a transverse traceless formulation), so that $\bar h = h$. In
Cartesian coordinates, a gravitational wave (GW) propagating in $z$ direction can
then be written as
	$$
	h_{\mu\nu} = 
	\begin{bmatrix}
	0 & 0 & 0 & 0 \\
	0 & h_{+} & h_{\times} & 0 \\
	0 & h_{\times} & - h_{+} & 0 \\
	0 & 0 & 0 & 0
	\end{bmatrix} \,.
	$$
Here, the scalars $h_{+}$ and $h_\times$ are identified as the two polarization
states of the wave. Typically, they are collected in a complex field
$y := h_{+} + ih_{-}$. The GW \emph{strain} is then defined as
$\xi_i = \frac 12 \partial_t^2 h_{ij} \xi^j$. The experimentally accessible
relative displacement $\delta \xi / \xi \sim h$ is proportional to $h$.

\subsection*{Implementation}

The gravitational wave strains $h_+$ and $h_-$ can be computed online in a time
evolution code. At large distance from a central massive object (formally at
infinity in Schwarzschild coordinates), typically at radius $R = \mathcal O(500) M$,
where $M$ is the mass of the spacetime, a curvature invariant (Weyl scalar $\Psi_4$,
in \code{EinsteinToolkit} following \cite{Newman62a,Baker:2002qf}) is derived from the
spacetime (\ie from the Riemann or Weyl tensor which is derived from the
4-metric or the ADM quantities, see also \cite{Nerozzi:2004wv,Burko:2005fa} for definitions).
A spherical harmonic mode decomposition is
then performed, where the GW DOF are encoded in the $l=2, m=2$ mode of $\Psi_4^{lm}$.

In a postprocessing step, the \code{PyCactusET} software is used for time integrating
the strain $h_{+,\times}$ from $\Psi_4$, following \cite{Reisswig:2011}.
Figure \vref{fig:analsyis-example} shows an example of a GW signal from a binary neutron
star merger obtained with this method, with a cutoff angular frequency $w_0=0.02$ for
fixed frequency iteration.




