%
% This is the text of the PhdThesis-version of the BNSLT paper, i.e. a single
% chapter within the thesis of Sven. It ~is~_was_ maintained at the same time as the
% bnslt paper itself ~is~_was_ written, but as seperate document. Like all texts in
% the thesis, it is written with tufte-book in mind.
% 
% In this directory, there are two standalone compilation targets available
% to render the content of this tex file without the whole thesis:
%
% 1. standalone-bnslt-thesis-version.tex => Looks like tufte/the thesis
%    (good for quickly checking the style)
% 2. standalone-bnslt-plain-version.tex => Looks discreet like a working report
%    (good for communication with coauthors)
%

\providecommand{\cf}{cf.,~}
\providecommand{\ie}{i.e.,~}
\providecommand{\eg}{e.g.,~}
\providecommand{\dd}{\mathrm{d}}
\providecommand{\mev}{\,{\rm MeV}}
\providecommand{\s}{\rm s}
\providecommand{\ms}{\rm ms}
\providecommand{\km}{\,{\mathrm{km}}}
\providecommand{\gcm}{\,{\mathrm{g}/\mathrm{cm}^{3}}}
\providecommand{\Msun}{M_{\odot}}
\providecommand{\Msol}{M_{\odot}}
\providecommand{\tov}{\textsc{tov}} % small caps TOV, as an alternative to {_{\rm TOV}} or \mathrm{TOV}
\providecommand{\Mtov}{M_\tov}
\providecommand{\tautov}{\tau_{\tov}}
\providecommand{\Rtov}{R_\tov}
\providecommand{\Mth}{M_\textrm{th}}
\providecommand{\sys}{{\mathrm{sys}}}
\providecommand{\Ctov}{\mathcal{C}_{\rm{TOV}}}

% Translate the revtex table rules to booktab table rules
\providecommand{\colrule}{\midrule}
\providecommand{\botrule}{\bottomrule}

% table column widths https://tex.stackexchange.com/a/12712/49958
\newcolumntype{L}[1]{>{\raggedright\let\newline\\\arraybackslash\hspace{0pt}}m{#1}}
\newcolumntype{C}[1]{>{\centering\let\newline\\\arraybackslash\hspace{0pt}}m{#1}}
\newcolumntype{R}[1]{>{\raggedleft\let\newline\\\arraybackslash\hspace{0pt}}m{#1}}

In this chapter, the lifetimes of the remnant produced by the merger of two neutron
stars is studied. By determining a maximum mass at which a
binary neutron star system collapses immediately to a black hole,
constraints on neutron star masses as well as a lower limit on their radii can be given.
This chapter relies partially on the coauthored publication \cite{Koeppel2019}.
An introduction into the problem and a review of previous work given in
Section~\vref{sec:bnslt-intro}. As a motivation, the physical content of the
tested nuclear equations of state are revisited.

\section{Motivation: Nuclear equation of state}\label{sec:bnslt-motivation}
%\todo{Do not talk about critical phenomenae, 
%\cite{Baumgarte:2016xjw,Gundlach-2003-critical-review,Choptuik93}, interesting for
%example \url{http://crunch.ikp.physik.tu-darmstadt.de/erice/2018/sec/talks/sunday/bauswein.pdf}}

In order to describe neutron stars with the coupled Einstein-Euler equations
(Chapters~\ref{chapter:gr} and \ref{chapter:hydro}), a suitable 
(realistic) equation of state must be chosen which encodes all
microscopic phenomenology (Section~\ref{sec:hd-nuclear-matter}).
The equation of state depends on thermodynamic/hydrodynamic quantities
such as (rest mass) matter (baryon) density $\rho$, internal energy $\epsilon$
and temperature $T$. One of the common approximations made (in high energy
physics in general) is to ignore thermal effects, assuming a ``cold'' EOS in the
limit of $T=0$. In fact, these EOSs can be used in the inspiral phase of binary
neutron stars (where the two stars are approximatively still ``cold''),
but after merger, when the temperatures of the merged objects reaches values
of several tens of MeV, the approximation breaks down. To
counter this, it is not uncommon \cite{Bauswein2011,Takami2014} to model
the post-merger dynamics by modifying zero-temperature EOS and modelling thermal
effects in terms of a ``thermal contribution'' via an ideal-fluid EOS
\cite{Rezzolla_book:2013} that accounts for the shock heating
\cite{Janka93}. This approach is not self-consistent but robust and the
use of thereby defined ``hybrid EOSs'' has been employed
extensively in the literature \cite{Baiotti2016}.

In contrast, ``hot'' equations of state to try model nuclear matter
with taking temperature into account. A couple of examples shall be given which
are relevant in this chapter.
%which a subset is employed in  (Section~\ref{sec:id}).

The Lattimer-Swesty (LS) EOS \citep{Lalazissis1996} is a popular EOS in both
core-collapse supernovae and binary merger simulations which models 
the nucleous as a finite-temperature
compressible liquid droplet with Skyrme nuclear pseudo-potential
\citep{Lattimer91,Lattimer85}. For heavy nuclei, the single nucleus 
approximation (SNA)
is adopted. The number 220 in LS220 refers to the chosen nuclear
incompressibility $K_0=220$MeV.

TM1 is the name of a popular EOS which name refers to a particular
parametrization used in the employed relavistic mean field (RMF) theory,
formulated in \cite{Sugahara1993} who added a nonlinear $\sigma$ model
to describe nuclear matter with relativistic Hartree approximation.
TM1 was first used in \cite{Lattimer91,Typel2010} with a Thomas-Fermi
distribution for temperature effects. Hempel and Schaffner-Bielich
adopted a nuclear statistical equilibrium (NSE) model and a different
RMF parametrization for the TM1 EOS, leading to the modified
HS-TM1 EOS \cite{Hempel2010,Hempel2012} which is used in place of the 
original one in this text.

DD \cite{Typel2005} presents another RMF model with adopts density dependent (DD)
meson-nucleon coupling. DD2 has an improved experimental nucleon mass,
introduced by \cite{Typel2010}. As with TM1, there is the HS-DD2 variant
\cite{Hempel2012} which is adopted in this text.
\begin{margintable}
	\centering
	\begin{tabular}{lr}
		\toprule
		EoS
		& $\rho_0 / \rho_\text{nuc}$
		\\
		\colrule
		BHB-LP  &  8.0 \\
		DD2     &  7.2 \\
		LS220   &  9.4 \\
		SFHo    &  9.8 \\
		TM1     &  6.7 \\
		Togashi &  7.4 \\
		\botrule
	\end{tabular}
	\caption[
	  Nuclear EOS, small table: Central energy density and pressure
	]{Central energy density $\rho_0$ of the maximum nonrotating mass star,
		in units of the nuclear
		saturation density $\rho_\text{nuc} = 2.7\times 10^{11}$kg/m$^3$ for
		the EOS discussed in section~\ref{sec:bnslt-motivation}. See table
		\protect\ref{table:eos-properties} for further properies.
	    % Rho_nuc = 2461 in Msol units
	}\label{table:EOS-nuclear-properties}
\end{margintable}

The Steiner-Fischer-Hempel (SFH) EOS \cite{Steiner2013} employs a non-linear
Wal\-ecka model (nucleon mean-field interaction via $\sigma$, $\omega$ and $\rho$
mesons) with some covariant interactions added to the Lagrangian. The SFHo
model is fitted to observations from \cite{Steiner2010}.

The Banik-Hempel-Bandyopadhyay (BHB) EOS \cite{Banik2014} uses the DD2 RMF
parameter set for nucleons, the HS NSE model for light and heavy nuclei and
includes strangeness (only the $\Lambda^0=uds$ hyperon and the
$\phi\approx s\bar s$ meson for hyperon-hyperon interaction, hence the name
BHB-$\Lambda\Phi$) and a first-order phase transition between the baryonic phase
and the quark phase, \ie the quark-gluon plasma.

The Togashi EOS \cite{Togashi2016,Togashi2017} includes $\Lambda$ and $\Sigma$
hyperons and is obtained with cluster variational methods.

In fact, SFHo, TM1, DD2 and BHB-$\Lambda\Phi$ are ruled out by
observations~\cite{Radice2017b}, while LS220 was ruled out already earlier by
unitary-gas considerations/\-con\-straints of chiral perturbation theory
\cite{Kolomeitsev2016}, leaving Togashi alone as 
not-yet-ruled-out in the battery of presented EOS.
%Nevertheless, due to the universal
%predictions on nuclear equations of state we claim to provide, it makes sense
%to cover a broad range of different equations of state. Furthermore, all
%mentioned equations of state are still popular in the community.

\section{Methods}
% Figure positioning issue ... therefore it goes here for the time being.
\begin{marginfigure}
	\hspace*{-.6cm}
	\includegraphics[width=1.15\textwidth]{bnslt-figures/bnstracker-plot.pdf}
	\caption[
	2D Cartoon, showing the inspiral of two NS with trajectories, 
	\exclusive
	]{
		Cartoon of the distance measure for merger time determination:
		Orbits of a BHB$\Lambda\Phi$ $M=1.55\Msol$ binary in the equitorial
		plane. The shaded circles in the background indicate the TOV radii
		$R=8.78$km. The distance $d=2R$ is displayed in black.
		Interestingly, the orbits also exhibit large 
		eccentricity since they overlap.
	}
	\label{fig:bnstracker}
\end{marginfigure}
We have numerically solved Einsteins equations in the BSSNOK
formulation (Section~\ref{sec:bssn}), fully coupled to 
the relativistic Euler equations (Section~\ref{sec:grhd}).
We used the \code{Ein\-stein\- Toolkit} \cite{loeffler_2011_et, ET2013, 
EinsteinToolkit:web} to solve these coupled partial differential equations at 
the same time with Method of Lines and a strong stability preserving RK3 method
(Section~\ref{sec:time-and-mol}).
The same techniques were employed in a number of other works
\cite{Hanauske2017,Bovard2017}.

We evolved spacetime with the \code{McLachlan} code \cite{Baumgarte99,Baumgarte2010, 
Shibata95, Brown09, Reisswig:2011a}, which employs
4th order finite differencing with artificial dissipation added. The lapse was 
evolved with $1+\log$ slicing and the shift with the Gamma
driver.

In contrast, we evolved matter with the \code{WhiskyTHC} code 
\cite{Radice2012a, Radice2013b, Radice2013c}. Here, we employed a 4th order 
finite volume scheme with LLF flux splitting, HLLE Riemann solver 
\cite{Harten83}, MP5 reconstruction operator \cite{suresh_1997_amp} and 
positivity-preserving limiter \cite{Hu2013}.

The three dimensional Cartesian grid was managed by the
\code{Carpet} code \cite{Schnetter-etal-03b, Schnetter:2006pg, 
Goodale02a} 
which provided mesh refinement to evolve both the merger as well as the wave 
zone. We adopted six fixed mesh refinement levels and cubic 
cells $h=\Delta x=\Delta y=\Delta z$. The finest resolution within a given 
simulation is $h=0.15\Msol \sim 0.215 \km$ if not denoted otherwise. The outer 
boundary of the domain extends to at least $500 \Msun \sim 700 \km$.
For the outer boundary conditions, radiative boundary conditions
(Appendix~\ref{sec:apx-bc}) were used.



\section{Definition of merger and collapse time}
\begin{marginfigure}
	\hspace*{-.6cm}
	\includegraphics[width=1.15\linewidth]{bnslt-figures/bnslt_analysis_for_paper.pdf}
	\caption[
	    Cartoon for the different time definitions for BNS, \exclusive
	]{
		Cartoon for demonstrating the definition and extraction of the 
		\emph{survival time} of a merger remnant.
		The red line compares the merger time determination in three different
		ways, from top to bottom: Seperation $d$, Gravitational wave strain
		(polarization $h_+$ dashed, envelope $h$ solid), global lapse $\alpha$
		minimum.
		The green line shows the merger time, indicated by the lapse.
		\\ This example shows an equal mass binary
		$M=1.55 M_\tov$ for the BHB-$\Lambda\Phi$ equation of state.
		% From top
		%to bottom: Rest mass density $\rho$ maximum, Weyl scalar $|\psi_4|^2$
		%($\ell=2$, $m=2$ mode read off at a spherical surface at $R=300\Msol$),
		%integrated gravitational wave strain $h_+$, and lapse $\alpha$ minimum.
	}\label{fig:analsyis-example}
\end{marginfigure}

In order to do a quantitative study, a number of time measures and their
determination shall be introduced.
We define the survival time $T=t_c - t_m$ of a merger product as the
timespan between the merger event $t_m$ and the collapse to a black hole $t_c$.
The individual times $t_c, t_m$ are measured in coordinate (simulation) time,
\ie their clocks start when the initial data are evolved. Clearly, there is
no unique definition of the collapse and merger, which determine $t_c$ and $t_m$.
In the following, a couple of different definitions are reviewed.

\subsection{Definition of merger time}
For the definition of the binary merger (time $t_m$), we examined either the
gravitational wave (GW) signal, the proper separation of the binary
neutron stars or a decreasing treshold
value of the global minimum of the lapse function (indicating a characteristic
maximum strength of the gravitational potential). We do not find the peak
of the global maximum (L$_\textrm{inf}$ integral) of the rest mass
density a good measure for the merger time.
 
In our simulation, we derive the complex Weyl scalar $\Psi_4$ from the Riemann
tensor (see Appendix~\ref{sec:extracting-gws} for grativational wave extraction)
on a spherical surface at a large 
seperation (typically 300$\Msol$ or 500$\Msol$). The magnitude $|\Psi_4|^2$ can 
serve as a
protoypic gravitational wave signal. However, one can also proceed to integrate
the strain $h_{+}$ and $h_{-}$ and define the merger time $t_c$ by the
maximum of the actual gravitational wave strain 
$h=(h_{+}^2 + h_{-}^2)^{1/2}$ \cite{Takami2014}. While we found subtle 
differences between the peak of the gravitational wave strain and the magnitude
of the Weyl scalar $\Psi_4$, for many EOSs it is $\Delta t_c \ll \Msol$.

The neutron star positions can be determined by tracking the two (Newtonian)
center of masses $\vec r_i$ during the evolution (Figure~\ref{fig:bnstracker}). 
The seperation $d=|\vec r_1-\vec r_2|$ can be measured in the local coordinate
frame. The merger time
can be defined as the first time when $d<2R$ with $R$ the radius of the individual TOV 
stars. \footnote{It can be useful to introduce a dimensionless scale factor $a$ to adopt for
the tidal deformation in the late inspiral, thus defining merger time when
$d=2aR$. The value of $a$ could be derived emperically, a typical value is $a=0.8$.
}

\subsection{The lapse as indicator}
For the definition of the back hole formation (time $t_c$), we decided to use 
the global minimum of the dimensionless lapse $\alpha$. Thanks to the
singularity
avoiding Bona-Masso slicing conditions \cite{Campanelli06,Baker05a}, the
minimum of the lapse can serve as an indicator for the curvature of the
spacetime. Furthermore, this quantity has been shown to be a very good proxy for the
tracking and appearance of an apparent horizon \cite{Alcubierre:2008}.
\footnote{
  We also evaluated numerical apparent horizon finders
  \cite{Libson94b,Schnetter03a}. However, they seriously slow down the time 
  efficiency of the code. In \code{Cactus}, it is affordable to determine field 
  reductions every $\Delta T_r\sim 20 M \lesssim 0.1$ms, whereas the 
  horizon finder is only computed every $\Delta T_h \sim 150 
  M \sim 0.75$ms.
  Therefore there is a need for the high time resolution \emph{proxy} of the 
  appearance an apparent horizon.
}

The simplest criterion which can be derived from $\operatorname{min}(\alpha)$
is a treshold value $\alpha_\textrm{merged}$ in order to define 
$t_m$ as soon as $\operatorname{min}(\alpha) < \alpha_\textrm{merged}$ for the
first time.  
%
%Another criterion for the collapse is the emergence of
%an apparent horizon. However, the dynamical detection of apparent horizons is
%expensive and was done only at poor resolution in our simulations.
%
In addition, one can adopt the lapse function $\alpha$ for not only defining the
collapse time $t_c$ but also the merger time $t_m$. Following the same argument 
as above, the merger time $t_m$ can be defined as soon as
$\operatorname{min}(\alpha) < \alpha_\textrm{collapsed}$ for the first time. 

The evolution of the minimum of the lapse has particular \emph{features}
which are recognisable in Figure~\ref{fig:analsyis-example}: There is a
sudden dropoff at merger time and a change in the slope around collapse time.
Figure~\ref{fig:alp_reasoning} illustrates this by showing also the first and 
second time derivative of min$(\alpha)$ for a reference run. In all our
simulations we observe that slightly before merger, the second derivative
drops and has a minimum at GW peak. Similarly, at collapse the first derivative
has a minimum. These features are not robust, but serve as a motivation for
the treshold values
\begin{equation}\label{eq:arbitrary-alpha-values}
\alpha_\textrm{collapsed} = 0.35
\quad\text{and}\quad
\alpha_\textrm{merged} = 0.2
\,.
\end{equation} 

Considering the definiton of the merger time $t_m$, we conclude that all
presented methods (GW strain, coordinate distance, falling below a minimal
lapse) are equally suited for defining the start of the merger and provide
similar times within a small error in time. Within this chapter, we concentrate
on using exclusively the lapse because it turned out to be most robust.

\begin{marginfigure}
	\hspace*{-.6cm}
	\includegraphics[width=1.15\textwidth]{bnslt-figures/alp_reasoning.pdf}
	\caption[
	Three panels of the evolution of the lapse, with first and second 
	derivative, \exclusive
	]{
		Time evolution in a reference binary system around merger time. The three panels
		show the lapse minimum $\min(\alpha)$, its first and second time
		derivative. The time is measured since merger (defined by GW peak),
		\ie $t_m=0$ in this units.
		The green line indicates merger by $\partial_t^3 \min\alpha=0$.
        The red line indicates collapse by $\partial_t^2 \min\alpha=0$ at 
        $t\sim 3.5$ms.
        % Data is from LS220 $M=1.44\Msol$ binary, but slightly modified,
        % thus this is more a cartoon/illustration
    }
	\label{fig:alp_reasoning}
\end{marginfigure}

\subsection{The free fall timescale}
Quite generically, one expects that the lifetime $T$ of an hypermassive neutron
star (HMNS) will decrease as the mass of the system is increased. In order to
allow comparable times $T$ for different EOSs, the collapse time $T$ is 
considered as a dimensionless quantity by expressing it in terms of the
matter free-fall time
$\tau_{\rm ff}$, \ie \cite{Rezzolla_book:2013}
\begin{equation}
\tau_{\rm ff} = \frac \pi 2 \sqrt{\frac{R^3}{2M} } \,,
\end{equation}
for an object of mass $M$ and radius $R$. As only equal mass binaries are
considered, $M$ is the mass of a single star and $R$ its radius. The smallest
free-fall time will be archieved for the maximum-mass model (since
$R\sim 1/M$). Therefore, the shortest free-fall timescale is set to be with
$\tau_\tov := \tau_{\rm ff} (M_\tov, R_\tov)$.
Hence, we define the threshold mass $M_{\rm th}$
as the one for which the merger remnant will collapse over such a
timescale, \ie $M/M_\tov \to M_{\rm th}/M_\tov$ for
$T/\tau_\tov \to 1$.
Any survival time $T < \tau_\tov$ is classified as a \emph{prompt} collapse to
a black hole.

\subsection{Angular momentum}
Our approach is to evolve a series of BNS initial data for different system
masses for a given equation of state (Figure~\ref{fig:eos-properties} shows
an open circle for each simulation run). We found that nearby masses might not
be comparable because the individual BNS systems do a different number
of orbits, resulting in a substaintially different amount of
angular momentum present at the merger time. In order to investigate the effect 
of the angular momentum, we also performed headon collisions with no angular
momentum (only linear momentum, i.e. a boost of $v=0.1c$). In such a case,
we observed all systems to collapse promptly at any given mass. We conclude
that angular momentum is (1). crucial to yield a finite survival time $T>0$
and thus initial data in quasi-circular orbits are required and (2.) the final
angular momentum has to be comparable between different binary masses in order
to have a well-defined collapse time.

\section{Initial data and EOS} \label{sec:id}
%
\begin{figure*}[t]
	\makebox[\textwidth][c]{ % overwidth figure
		\includegraphics[width=1.1\linewidth]{bnslt-figures/eos_properties.pdf}
	}% end of makebox
	\caption[
	  TOV curves, showing different properties (Mass, tidal love number, tidal 
	  deformability, free-fall timescale), \exclusive, only first panel was
	  used in our publication \cite{Koeppel2019}.
	]{Properties of TOV sequences:
		Radius $R$, dimensionless tidal love number $k_2$,
		tidal deformability $\Lambda$ and free fall timescale $\tau$
		as functions of the star mass $M$. The maximum masses $\Mtov$
		and their relevant quantities $R_\tov$, $\kappa_\tov$, $\Lambda_\tov$
		and $\tau_\tov$ are displayed with fulled circles. Open circles refer to
		models used as initial data.
		First panel published in \cite{Koeppel2019}.
	}\label{fig:eos-properties}
\end{figure*}%
%
\begin{table*}[b]
	%\begin{minipage}{\textwidth} % avoid caption to be half text width
	\caption[
	  TOV maximum mass properties (mass, radius, compactness, baryon mass, 
	  tidal Love number, deformability, lifetime, literature reference)
	]{Various TOV properties of the equations of states taken into 
		account.
		These are the maximum mass of a nonrotating star $\Mtov$, its radius 
		$\Rtov$, its compactness $C_\tov = M_\tov / R_\tov$, its tidal Love
		number $\kappa_{2,\tov}$, its derived tidal deformability 
		$\Lambda_\tov$, and its free-fall timescale $\tau_\tov$. Furthermore,
		the literature which introduces the particular EOS is given for
		completeness in the last column. See also Table 
		\protect\ref{table:EOS-nuclear-properties} for the central density 
		$\rho_c$
		of the maximum nonrotating mass configuration.
	}\vspace*{.2cm}\label{table:eos-properties}%
	\begin{tabularx}{\textwidth}{Xrrrrrrrrrr}
		\toprule
		EoS               & $\Mtov [\Msol]$ & $\Rtov [\km]$ & $C_\tov$ & 
		$M_{b,\tov} [\Msol]$
		& $\kappa_{2,\tov}$ & $\Lambda_\tov$ & 
		$\tau_\tov [\mu \s]$  & Literature  \\
		\colrule
		BHB-$\Lambda\Phi$ &           2.10 &        11.64 &  0.26 & 2.74 & 
		0.020 
		&  13.50 & 83.31 
		& \cite{Banik2014} \\
		DD2               &           2.42 &        11.94 &  0.30 & 3.28 & 
		0.020 &   
		5.52 & 80.60 
		& \cite{Bildsten92,Hempel2010} \\
		LS220             &           2.04 &        10.68 &  0.29 & 2.79 & 
		0.014 
		&   7.65 & 74.22
		& \cite{Lalazissis1996} \\
		SFHo              &           2.06 &        10.34 &  0.29 & 2.79 & 
		0.019 &   5.81 & 70.44
		& \cite{shen98} \\
		TM1               &           2.22 &        12.60 &  0.26 & 2.86 & 
		0.029 &  
		16.60 & 91.70
		& \cite{Lattimer91,Typel2010} \\
		Togashi           &           2.23 &        10.17 &  0.32 & 3.13 & 
		0.014 &  2.65 & 66.12 
		& \cite{Togashi2016,Togashi2017} \\
		\botrule
	\end{tabularx}
	%\end{minipage}
\end{table*}%
%
We model the neutron stars with realistic tabulated hot equations of states 
which are
freely available at \href{http://stellarCollapse.org}{stellarCollapse.org}.
We compute equilibrium solutions of the Tolman-Oppenheimer-Volkoff equations 
(TOV) with the \texttt{PizzaTOV} code \cite{Kastaun2007} and its successor 
\texttt{MargheritaTOV} \cite{Most2018b}.
% is repeated below:
%See Figure~\ref{fig:eos-properties} and
%Table~\ref{table:eos-properties} for some TOV properties.
%
If not mentioned otherwise, initial data for the time evolution
(ie. binary neutron star initial data) are computed under the assumption
of irrotational quasi-circular equilibrium  with \code{Lorene}
\cite{Bonazzola98b, Gourgoulhon01,Damour:2002qh}. The initial
separation is 45km, so that the binaries perform around five orbits before the
merger. We note that since
the threshold mass for equal-mass binaries is always larger than for
unequal-mass binaries, \ie $\Mth(q=1) > \Mth(q<1)$, the use
of equal-mass binaries is not a restriction but optimises the search for
$\Mth$~\cite{Bauswein2017b}.
%Some TOV properties are displayed in figure \ref{fig:eos-properties}. Notably, 
%some maximum mass properties are listed in table~\ref{table:eos-properties}.

\subsection{Nuclear equations of state taken into account}
Since the overall goal is that of determining as accurately as
possible the threshold mass to prompt gravitational collapse, it is
essential that the description of the thermal effects in the matter is as
realistic and self-consistent as possible. In turn, this forces us to
consider EOSs that have a physically consistent dependence on
temperature (Section~\ref{sec:bnslt-motivation}).
Unfortunately, the number of EOSs that can be employed for
this scope and that do not violate some basic nuclear-physics requirement
(as it is the case for the widely employed LS220
EOS \cite{Lattimer91,Kolomeitsev2016}), is much more restricted.

Notwithstanding this limitation, we have employed here all of the (five)
``hot'' EOSs that have been proposed recently and whose properties are
reported in Table~\ref{table:eos-properties} when expressed in terms of
the masses and radii of the maximum-mass of the nonrotating configuration
(hereafter indicated as TOV). Similarly, Fig.~\ref{fig:eos-properties},
provides a graphical representation of the masses and radii of the
corresponding TOV equilibrium solutions, both stable (solid lines) and
unstable (dashed lines). The solid circles mark the
maximum-mass solutions, while the open circles refer to models used as
initial data (see below).

%We chose to take into account six equations of states (EOS) which are listed in
%Table~\ref{table:eos-properties} together with certain properties of the
%(nonrotating) maximum mass configurations of their TOV sequences.
%Figure~\ref{fig:eos-properties} displays the mass--radius diagram of these
%equations of state and highlights which regime was actually probed by
%the binary systems taken into account. All equations of states are realistic,
%and ``hot'' equations of states.

% Finde ich nicht mehr sinnvoll zu schreiben:
%
%All EOS are assumed in nuclear statistical
%equilibrium (NSE) for $T \leq 0.5$MeV, \ie below nuclear staturation. The 
%tables are computed in neutron-less $\beta$-equilibrium.

We recognize the maxima of each TOV properties to serve as characteristic 
values for the particular equation of state. These values can be used to 
normalize all properties of any matter distribution described with the equation 
of state \footnote{
  The TOV approximation can be further motivated by the fact that the
  irrotational stars within the binary share universal properties with
  their TOV counterparts. That is, the authors of \cite{Breu2016} have
  shown that the maximum mass of a rotating neutron 
  star can brought into a simple relationship to the non-rotating solution,
  $M_\text{max,rot} \approx 1.203 M_\tov$,
  and therefore it is sufficient to discuss non-rotating stars only in this 
  section.
}
Especially for the free-fall timescale $\tau$, which serves for
normalization in the next sections, there is a large discrepancy between
$\tau_\tov$ and the numbers in the relevant regime, which is up to a factor
two. However, we performed the 
whole analysis with both a dynamical $\tau(M)$ as well as a fixed 
$\tau(M_\tov)$ and find only a minimal difference $\Delta T \ll \Msol$.

%%%%%%%%%%%%%%%%%%%%%%%%%%%%%%%%%%%%%%%%%%%%%%%%%%%%%%%%%%%%%%%%%%
\section{Results on the treshold mass} \label{sec:results}
%%%%%%%%%%%%%%%%%%%%%%%%%%%%%%%%%%%%%%%%%%%%%%%%%%%%%%%%%%%%%%%%%%
\begin{figure}[t]
	\includegraphics[width=\linewidth]{bnslt-figures/tau_lin_scale.pdf}
	% Allow to adress the three panels with references
	\caption[
	   BNS lifetime extrapolation to free fall timescale, 
	   \publishedIn{Koeppel2019}
	]{The scatter plot of measured survival (collapse) times $T$,
	  here denoted as $t_{\rm{coll}} \equiv T$ for the different EOS
	  (circles). The solid line represents a Gaussian fit.
	  Stars represent the extrapolated treshold mass $\Mth$,
	  predicted by the fit. All survival times $t_{\rm{coll}}<\tau_\tov$
	  are considered as prompt collapse. Figure published in
	  \cite{Koeppel2019}.
     }%
    \label{fig:lin_noscale}
\end{figure}

%\subsection{Calculating the threshold mass}\label{sec:calc_thresh}

In order to calculate the threshold mass of a given EOS, over 200
simulations were run for system masses with a short, but finite, collapse time.
From these runs, only 15 suitable runs were selected for the analysis
which had survival times $T<1.0$ms.

Figure~\ref{fig:lin_noscale} reports the survival times $T$
normalised to the free-fall time\-scale of the maximum mass models
$\tau_\tov$ for the individual EOS. These times then are shown as a
function of the initial mass $M$ of the binary system normalized to the
EOS maximum mass. The adoption of such set of dimensionless
quantities has the goal of revealing a universal behaviour in the
treshold mass, if one is present \cite{Yagi2013a, Breu2016,Weih2017}.

The coordinates of filled circles of different colors in 
Fig.~\ref{fig:lin_noscale} are given by the system masses (initial data)
and their read-off lifetime (after evolution).
%Note that as $T$ decreases, even small differences in the initial
%masses can lead to rather large differences in the survival time.
For the Tagoshi$+$ EOS we only report two values and that these differ by
only $3.7\%$ in mass (\ie $M=1.440\,M_{\odot}$ and $M=1.435\,M_{\odot}$);
any other binary with a slightly smaller mass (\eg $M=1.430\,M_{\odot}$)
leads to a hypermassive neutron star that is effectively stable over the
timescales investigated here (\ie up to 1-10$\tau_\tov$).
Finally, since $t_{\rm{coll}}$ should diverge for
vanishingly small values of $M$, we fit the numerical data with a simple
exponentially decaying function of the type
\begin{margintable}
	\centering
	\begin{tabular}{lR{1cm}R{1.4cm}}
		\toprule
		EoS
		& $M_\text{th}$
		& $\Delta M_\text{th}$
		\\
		\colrule
		BHB-LP  &  1.503 & 0.005 \\
		DD2     &  1.364 & 0.020 \\
		%LS220   &  1.487 & 0.00012 \\
		SFHo    &  1.391 & 0.016 \\
		TM1     &  1.520 & 0.015 \\
		Togashi &  1.298 & 0.000 \\
		\botrule
	\end{tabular}
	\caption[
	  Numerical results of BNS prompt collapse treshold mass
	]{Results for the threshold mass, including uncertainties.	
		The masses are given in units of $M_\tov$. Errors are
		discussed in Section~\ref{sec:bnslt-errors}.
	}\label{table:Mtresh}
\end{margintable}
\begin{equation}\label{eq:bnslt-gauss}
M/M_\tov =
\tilde{a}\exp[-\tilde{b} ( t_{\rm coll}/\tau_\tov )^{2}]
\,.
\end{equation}
\begin{figure}[b]
	\caption[
	Lifetimes, exponential vs. linear extrapolation,
	taken from private referee communication about \cite{Koeppel2019}.
	]{Same as Fig. \protect\ref{fig:lin_noscale}, but also showing the linear extrapolation
		to the critical mass as a dashed line, as done in \cite{Bauswein2013}. In all
		cases, the extrapolation values for the critical mass (shown as open squares)
		are systematically larger then in the exponential model (see main text).
	}\label{fig:lin_exp_comp_lifetimes}
	\includegraphics[width=\columnwidth]{bnslt-figures/referee_reply__lin_vs_nonlin_fit.pdf}
\end{figure}
The behaviour reported in Fig.~\ref{fig:lin_noscale} reveals
that a a universal behaviour is present only very approximately and that the
threshold mass, averaged over all EOSs, is roughly given by
%
\begin{equation}
\frac{M_{\rm th}}{M_\tov} \approx 1.415\,,
\end{equation}
%
with a statistical error of $\Delta M_{\rm th}=0.05 M_{\odot}$, \ie with
a variance of about $4\%$ (see Table~\ref{table:Mtresh} for the individual
values).
When comparing with a linear approximation, as it was done in \cite{Bauswein2013},
\begin{equation}\label{eq:bnslt-linear}
M/M_\tov = \tilde a + \tilde b ( t_{\rm coll}/\tau_\tov ) \,,
\end{equation}
we find it yields a systematic overestimate of the treshold mass 
(Figure \ref{fig:lin_exp_comp_lifetimes}). When considering which of the functional
behaviours, \eqref{eq:bnslt-gauss} vs. \eqref{eq:bnslt-linear}, fits the
data best, the statistics do not provide a distinction criterion due to
the small number of points. \footnote{The reduced $\chi^2$ are 0.0055 for the linear
fit and slightly better, 0.0089, for the nonlinear fit.} More importantly,
we believe it is not reasonable to expect that near the free-fall limit the
behaviour should be a linear one. Such a limit, in fact, should be thought
as a regime where only infinitesimal changes in the mass should lead to a
prompt collapse, exactly because the merged object is very close to a stability
limit. Such a behaviour, which is seen frequently in critical-collapse
calculations (see \eg \cite{Gundlach-2003-critical-review} for a review),
necessarily requires that the function $M/M_\tov$ should have
vanishing derivative in the limit $t_\text{coll}/\tau_\tov \to 1$.
Clearly, the nonlinear fitting \eqref{eq:bnslt-gauss} reflects this behaviour while
a linear one \eqref{eq:bnslt-linear} does not.

\subsection*{Predicting the treshold mass from the TOV compactness}
The existence of a relation between the threshold mass and the corresponding
maximum mass has been suggested initially by~\cite{Bauswein2013}, who, by
employing a smooth-particle approximation for the hydrodynamics and a conformally
flat approximation to general relativity, conclude the linear relationship
\begin{equation}\label{eq:linear-c-tov-bauswein}
{M_{\rm th}}/{M_\tov}=\hat{a}\,\mathcal{C}_\tov+\hat{b}
\end{equation}
with $\mathcal{C}_\tov := M_\tov/R_\tov$ the 
maximum compactness. The universal ansatz proposed by \cite{Bauswein2013}
is that ${M_{\rm th}}/{M_\tov}$ is independent of the EOS, with
\begin{equation}
\hat{a}=3.38,\,\hat{b}=2.43 \,.
\end{equation}

\begin{figure}[t]
	%\makebox[\textwidth][c]{ % cheat; tinyurl.com/y8oxsaqr
	\includegraphics[width=\columnwidth]{bnslt-figures/uni_rel.pdf}
	%}% end of makebox
	\caption[
      Universal relations plot, \publishedIn{Koeppel2019}
	]{Universal relations for the treshold mass with the stars matching
	  the ones in figure~\protect\ref{fig:lin_noscale}. The solid blue line is
	  the non-linear fit~\protect\ref{eq:fit} while the red dashed line
	  is the linear fit~\protect\ref{eq:linear-c-tov-bauswein} from
	  \cite{Bauswein2013}. The green shaded area displays the compactness
	  expected for neutron stars (see main text).
	  The inset shows a bird's eye view on the possible whole range of
	  compactness, \ie $\mathcal{C}_\tov\in[0, \nicefrac 12]$.
	  Figure published in \cite{Koeppel2019}.
	}
	\label{fig:uni_rel}
\end{figure}

Such a linear ansatz does represent a reasonable first approximation to
the data.  However, it clearly overestimates the threshold mass in the
limit $\mathcal{C}_\tov\to0$, as we would assume
\begin{equation}\label{eq:mth-mtov-2-limit}
\Mth / \Mtov \to 2 \quad\text{for}\quad \Ctov \to 0 \,.
\end{equation}
This limit physically represents an infinitely extended self-gravitating
object. \footnote{
  Loosely speaking, the limit $\mathcal C_\tov \to 0$ is also a
  classical one, as one moves from relativistic self-gravitating
  configurations over to Newtonian ones.
}
We do \emph{not} impose this limit on the fitting function. However, it
is interesting (and revealing) that the data and the
fitting function naturally provides this limit \eqref{eq:mth-mtov-2-limit}:
For $\mathcal C_\tov \to 0$ the merged system would need to be nonorotating
and hence with a threshold mass that is exactly twice the TOV mass.

Furthermore, and more importantly, the linear relationship \eqref{eq:linear-c-tov-bauswein}
does not provide the expected black hole limit, which predicts that
\begin{equation}\label{eq:black-hole-constraint-limit}
{M_{\rm th}}/{M_\tov}\to0 \quad\text{for}\quad
\mathcal{C}_\tov\to1/2
\,.
\end{equation}
This constraint requires some clarification. For neutron stars, the treshold mass
${M_{\rm th}}$ must be larger then the TOV mass ${M_\tov}$, as the latter is the
limit for stability to collapse. However, neutron stars make only a part of
Figure~\ref{fig:uni_rel}, shaded in light green, \ie the most likely range
of possible values of  $\mathcal{C}_\tov$ for stable neutron
stars, with the lower limit $\mathcal{C}_\tov\gtrsim0.2$ being
deduced from a large statistical sample of possible EOSs
\cite{Most2018}, while the upper limit $\mathcal{C}_\tov\lesssim0.35$
is set by the limit on causality~\cite{Koranda1997, Lattimer2016}.
It is useful to consider the Buchdahl theorem \cite{Rezzolla_book:2013},
which states that any self-gravitating object whose compactness is
larger than $\mathcal C_B=4/9$ cannot be in equilibrium and must collapse to a black
hole. The theorem, which is valid for any EOS, does not specify what is
the mass of the object nor what is the corresponding TOV mass. All that
is required to produce a collapse for such ultra-compact configurations
is to reach such a compactness. Hence, it is possible to construct a
self-gravitating object with finite mass whose compactness is only
infinitesimally smaller than $\mathcal C_B$ and then add the (infinitesimal)
amount of mass that would lead the compactness to exceed $\mathcal C_B$. This
object would collapse promptly even if its mass is not large at
all. Taking this line of argument to the limit, the threshold mass will
have to go to zero when a black hole is already formed, \ie for
$\mathcal C \to 1/2$. The (extended) regime of extremely compact objects
(ECOs) is shaded in dark green in Figure~\ref{fig:uni_rel}.

Hence, we correct the linear approximation via a \emph{nonlinear} fit of the type
\begin{equation}
	\label{eq:fit}
	\frac{M_{\rm th}}{M_\tov} = a - \frac{b}{1 -
		c\,\mathcal{C}_\tov}\,,
\end{equation} 
where $a,b,c$ are to be determined from the data. However, imposing the
fulfilment of the black hole constraint limit 
\eqref{eq:black-hole-constraint-limit}
removes one free parameter and sets $a=2b/(2-c)$.

Figure~\ref{fig:uni_rel} reports in blue the fit of Eq.~\eqref{eq:fit}
with
\begin{equation}
b = 1.01, \quad c = 1.34 \,,
\end{equation}
against the numerical-relativity data shown with stars of the same
colors as in Fig.~\ref{fig:lin_noscale}. \footnote{
	See Table~\ref{table:Mtresh} for errors in the fit.
	We also made a linear fit \eqref{eq:linear-c-tov-bauswein}
	to our data (not shown). Considering the quality of the fit with the $\chi{2}$ test,
	\eg for DD2 we find for the linear fit a $\chi^2 = 1.4\times 10^{-5}$
	while the nonlinear fit has $\chi^2 = 1.1\times 10^{-5}$. Given the small
	number of data points (as in Section \ref{sec:results}), the $\chi^2$ for
	the nonlinear fit is slightly better, but the statistics do not provide a
	sufficient distinction criterion (in contrast to the proposed 
	physical arguments).
	Naturally, the linear model of Bauswein can be understood as the first term in
	a series expansion the nonlinear model~\eqref{eq:fit}.
	It is also qualitatively obvious from Figure~\ref{fig:uni_rel} that
	the linear result of \cite{Bauswein2017b} is a tangent to our curve.
} Also shown with a red-dashed
line is the linear approximation of~\cite{Bauswein2017b}, which clearly
suggests larger treshold masses.
We believe this result is a consequence of our fully
general-relativistic approach, which properly accounts for the
strong-curvature highly dynamical behaviour that characterizes the
threshold to black hole collapse and that are probably underestimated in
the conformally flat approximation of~\cite{Bauswein2013}. At the same
time, the difference with the linear approximation of
\cite{Bauswein2017b} is small ($8\%$ at most for the EOS considered here).

\section{Constraining Neutron Star radii}
%
The nonlinear expression \eqref{eq:fit} can be used to provide more stringent
(larger) lower limits on the radii of possible stellar models in the light of
the recent detection of the event GW170817 \cite{Abbott2017}. In particular,
following \cite{Bauswein2017b}, our treshold mass model
\eqref{eq:fit} can be used to constrain neutron star radii. To do so, we write
\eqref{eq:fit} formally as
\begin{equation}\label{eq:rewrite}
\Mth(M,R) = \left(a - b \left[1 - c \left( \frac{M}{R} 
\right)~\right]^{-1} \right) M
\end{equation}
and then plot $\Mth(M)$ for different $R$ in
Figure~\ref{fig:radius-constraint} (black lines). 
Also reported in
Fig.~\ref{fig:radius-constraint} with a gray-shaded area is the limit set
by causality and that requires $M_\tov/R_\tov \lesssim0.354$~\cite{Koranda1997, Lattimer2016}. 

\begin{figure}
	\includegraphics[width=\linewidth]{bnslt-figures/radius_constraint.pdf}
	\caption[
	BNS Lifetimes: Radius constraints, \publishedIn{Koeppel2019}
	]{The lower bound on $R_\tov$ (red) using the universal relation
		Eq.~\protect\eqref{eq:fit}. The horizontal dashed blue line
		represents the observed mass of GW170817. The red shared area shows the
		values excluded by the detection. The grey shaded area represents
		values excluded by the causality constraint. Figure published in
		\cite{Koeppel2019}.
	}
	\label{fig:radius-constraint}
\end{figure}

As noted by~\cite{Bauswein2017b}, given a measurement
of a binary neutron-star merger with a given total mass $M_{\rm tot}$,
and assuming that the merger product has collapsed to a black hole, it is
possible to set a lower limit on $M_{\rm th}$ and, in turn, a lower limit
(although not very stringent) on $R_\tov$. This is shown
graphically in the left panel Fig.~\ref{fig:radius-constraint}, where we
report with a horizontal blue-dashed line and the total gravitational
mass estimated for GW170817~\cite{Abbott2017},
\begin{equation}\label{eq:radius-constraint-outcome}
M_\text{tot} = 2.74 ^{+0.04}_{-0.01} \Msol
\,.
\end{equation}
The corresponding uncertainty band (blue-shaded area) gives a lower
constraint on $M_{\rm th}$, since GW170817 did not lead to a prompt
collapse.

The blue band thus cuts (``constrains'') the red shaded area from below.
Especially it gives us an estimate of a neutron star \emph{minimal} radius,
$R_\tov \ge 9.74^{+0.14}_{-0.04}\,$km (red solid line);
this is to be contrasted with the value deduced by~\cite{Bauswein2017b},
\ie $R_\tov \ge 9.26^{+0.17}_{-0.03}\,$km, on the basis of
their linear approximation. Interestingly, in order to obtain a similar
stringent constraint derived here, the authors
of \cite{Bauswein2017b} require a
hypothetical detection of a binary with a comparatively larger mass
$M_{\rm tot} \simeq 2.9\,M_{\odot}$.

\begin{figure}[b]
	\caption[
	Universal relations for different neutron star sizes,
	\publishedIn{Koeppel2019}
	]{Universal relation (black) for the lower limit on $R_x$
		for a given mass $M$ (blue crossed); the red arrow is the constraint
		from \cite{Bauswein2017b} for a 1.6$\Msol$ star.
		Figure published in \cite{Koeppel2019}.
	}
	\label{fig:radius-Rx}
	\includegraphics[width=\linewidth]{bnslt-figures/rad_fit.pdf}
\end{figure}
All of the procedure followed so far to derive the nonlinear fit
\eqref{eq:fit} for $\mathcal{C}_\tov$ can be repeated for an the
compactness of a fixed mass $M_x$, \ie $\mathcal{C}_{x}:=M_x/R_x$, thus
allowing us to set constraints not only on $R_\tov$, but on any
radius $R_x$ within a reasonable range. The result of this series of fits
is shown in Fig.~\ref{fig:radius-Rx}, where
the values of $M_x$ and $R_x$ are indicated with a blue crosses. Also
reported in black is the quadratic fit
%
\begin{equation}
\label{eq:fit_generic}
R_x = -0.88\,M^{2} + 2.66\,M + 8.91\,.
\end{equation}

The importance of \eqref{eq:fit_generic} is that it now offers a very
simple and handy expression for the lower limit of stellar models as
deduced from GW170817. A similar procedure has been followed also by
\cite{Bauswein2017b}, but only for a fixed mass of $1.6\,M_{\odot}$,
and it was deduced that $R_{1.6}\ge 10.30\,{\rm km}$; this result should
be contrasted with the the value derived from \eqref{eq:fit_generic},
which is instead $R_{1.6}\ge10.90\,{\rm km}$.  Similarly, for a reference
star of $1.4\,M_{\odot}$ we obtain $R_{1.4}\ge10.92\,{\rm km}$, which is
close to the estimate by~\cite{Bauswein2017b} for
$1.6\,M_{\odot}$\footnote{In \cite{Bauswein2017b}, an estimate is only
	provided for $R_{1.6}$.}. Our $R_{1.4}$ estimate is in good
agreement with those made by \cite{Most2018}, who have expoited a
statistical exploration of possible EOSs and building a set of one
billion stellar models, \ie $12.00<R_{1.4}/{\rm km}<13.45$.

\section{Convergence and error budget}\label{sec:bnslt-errors}
For a typical equation of state (SFHo), we made a convergence study with
different resolutions of $\Delta x$=215, 287 and 573m
($\Delta x/\Msol = 0.15, 0.20, 0.40$). Figure~\ref{fig:convergence}
shows the variation of the collapse time and treshold mass
as a function of resolution.
The linear continuum extrapolation clearly
shows the first order convergence of the results.

Concerning the error budget, 
systematic errors are introduced by the time resolution
of the repeated output of field integrals and derived quantities. For
instance, for efficiency we determine field reductions only every 128
timesteps ($\Delta T\sim 20 M \lesssim 0.1$ms). This read-off error goes
into the threshold mass determination. However, since a two-parametric
function is fitted to two to three data points
(Fig.~\ref{fig:lin_noscale}), there is virtually no statistical
read-off error of the threshold mass (as demonstrated by Table~\ref{table:Mtresh}).
These negligible errors are at the order
of $\Delta M_{\rm th} \sim 0.05 M_{\odot}$, \ie at the order of 1\%. These
errors go
into the universal relations plot (Fig.~\ref{fig:uni_rel}) where again a
two-parametric function is fitted to five data points, resulting again in
an inconclusive error budget of $\sim$1\%. The 5\% deviation from
\citep{Bauswein2017b} is obviously a systematic consequence of the
overall technique and not part of statistical errors. Therefore, the
radius constraining plot (Fig.~\ref{fig:radius-constraint}) has no
significant errors coming from the model~\eqref{eq:fit} and the error
given on the radius constraint is dominated by the observational error from
GW170817.

\begin{figure*}[t] % width=0.5\linewidth
	\includegraphics[height=4.5cm]{bnslt-figures/referee_reply__convergence_tcoll.pdf}
	\includegraphics[height=4.5cm]{bnslt-figures/referee_reply__convergence_Mth.pdf}
	\caption[
	  BNS Lifetimes, convergence study plot for collision time and critical
	  mass determination, \exclusive; part of private
	  conversation with our referee within \cite{Koeppel2019}.
	]{Convergence plots on the collapse
	  times and the determined critical masses for an exemplaric system
	  and equation of state. The colors of the dots represent the different resolutions.
	}
	\label{fig:convergence}
\end{figure*}

\section{Summary}
In Chapter~\ref{chapter:bnslt}, a large number of general relativistic
simulations of binary neutron star mergers has been carried out in order to
investigate the open question about the mass leading to a prompt collapse;
which again is informative on the material neutron stars could be made
of (\ie the nuclear equation of state). Using a fully general-relativistic
approach and a novel method for the determination of the threshold mass, we have
carried out simulations making use of all of the realistic EOSs
available to describe this process. In this way, we have found a
nonlinear universal relation for the threshold mass as a function of the
maximum compactness and which is potentially valid for all compactnesses.
At least for the temperature-dependent EOSs considered here, this universal
relation improves the linear relation found recently with methods that
are less accurate, but that also yield quantitatively similar results.
Furthermore, exploiting the detection of GW170817,
we have used the universal relation to set lower limits on the
stellar radii for any any mass.


