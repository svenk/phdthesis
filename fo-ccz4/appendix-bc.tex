% This text was written at 2017-10-24 for ExaHyPE and lives in the
% ExaHyPE-Astrophysics/doc directory as a standalone tex. The version
% for the thesis was adopted.

% Integrals with long limits above/below
\newcommand{\Int}[2]{\int\limits_{\mathrlap{#1}}^{\mathrlap{#2}}}

\section{Boundary conditions for Einstein Equations}
\label{sec:apx-bc}
This appendix briefly reviews a number of boundary condition choices
adopt\-ed within the implementation of the Einstein solver within the codes
discussed in Chapter~\ref{chapter:gr}.

When evolving full spacetimes of compact objects (\ie extended or collapsed
matter located within a small region around the origin of the coordinate
system), boundary conditions must take care of the finite simulation domain
(sometimes refered to as ``finite volume effect''), \ie the fact that the
physical domain is by concept infinite while the numerical one is finite.
Any kind of unphysical effects at this boundary, such as reflection of
outgoing waves, shall be prevented.

In a cartesian grid, the box-like geometry of the boundary renders this
requirement to a complex task. The grid-level solution is to overcome the
cartesian geometry and to adopt a spherical one in the outer wave zone/near
the boundary. Then, wave-absorbing Riemann solvers can be applied which have
good properties for orthogonally arriving waves.

However, within the Einstein Toolkit, the application of radiative boundary
conditions have turned out to be sufficiently successful for long-term 
evolutions (such as binary mergers) without significant reflections at the
boundary. Therefore, the focus within this section is on that candidate,
next to exact Riemann solvers. Other boundary conditions which were
successfully adopted are periodic boundary conditions (\ie evolving a
\emph{lattice} of black holes) and exact boundary conditions (not successful
in black hole spacetimes, but sufficient for simpler benchmarks).

\subsection{Sommerfeld boundary conditions}
This section shortly reviews a simple version of the Sommerfeld boundary
conditions, also known as radiative boundary conditions. The key idea of the
``radiative'' boundary conditions is to solve a different
PDE at the boundary which models radial outgoing scalar waves, independently
for every component of the state vector for a given system.  This PDE can be
cast in a similiar fully non-conservative way as the CCZ4 PDE system itself, \ie
\begin{equation}\label{eq:apx-bc-base-pde}
\partial_t Q + B(Q) \nabla Q = S(Q)
\,.
\end{equation}

A motivational `toy'' deviation introduces the scalar field $f=f(\vec r,t)$
as a place holder for a gravitational wave field, a component of the metric
tensor or any other field from the FO-CCZ4 state vector. For $f$, the
\emph{outgoing radial wave assumption}
\begin{equation}
r f(r,t) = \operatorname{const}
\end{equation}
is claimed, with $r=|\vec r|$. The time derivative of this assumption gives 
a partial differential equation:
\begin{equation}
0 = \frac{\mathrm d}{\mathrm d t} (r f)
   = v f + r \left( \frac{\partial f}{\partial t} + \sum_i \frac{\partial f}{\partial r_i} \frac{\mathrm d r_i}{\mathrm d t} \right)
\end{equation}
with a radial velocity $v=\mathrm d r / \mathrm d t$. The \emph{radial wave 
assumption}
\begin{equation}
\frac{\mathrm d r_i}{\mathrm d t} = v^i \approx v \, e^i := v \frac{r^i}{r}
\end{equation}
approximates the velocity $\vec v$ as a radial velocity $v=|\vec v|$.
$v$ is now related to the propagation speed of the $f$ which is
the speed of light, i.e. $v=c\equiv 1$ (however, for the sake of clarity
the symbol $v$ is kept for the time being). One thus ends up with the
nonconservative PDE
\begin{equation}
\partial_t f + v \frac{r^i}{r} \partial_i f = - v \frac{f}{r}
\end{equation}

The outgoing radial wave assumption implies $\lim{r\to\inf} f(r)=0$. For
some ADM variables (like $\alpha$ or $\gamma_{ii}$), the limites are $f_0 \neq 0$. In such a case, we define $g=f-f_0$, insert $g$ at the place of $f$ into
the PDE and obtain for the actual $f$ a different source term:
\begin{equation}\label{eq:pde}
\partial_t f + v \frac{r^i}{r} \partial_i f = - \frac{f-f_0}{r}
\end{equation}
This PDE can be seen as the generic case for an outgoing wave with value $f_0$ at the boundary.

In the CCZ4 system, we solve such a PDE for every single field of the 59 fields.
This gives us 59 non-coupled differential equations.

To be explicit, \eqref{eq:pde} shall be applied as follows: As initial data,
the system's solution at or near the boundary is taken. Depending on the
numerical interpretation, one might also extrapolate the solution ``to'' or
``beyond'' the boundary to serve as initial data. The boundary conditions 
itself adopted while integrating \eqref{eq:pde} are negligable, copy/outflow
BC are appropriate.

\subsection*{Diffusive Boundaries}
Solving a different PDE ``at the boundary'' is a blurry formulation. In a
FD/FV code, where ghostzones are maintained for holding the information just
outside the simulation domain, \eqref{eq:pde} can be solved directly in the
ghostzone, if the scheme is at least second order (so the ghostzones are
three-dimensional). In a DG code, where no ghostzones exist, \eqref{eq:pde}
could be solved evolving the whole last DG patch with the boundary PDE. This
either shifts the boundary into the simulation domain or is implemented in a way
that the cell is evolved with the actual PDE (say FO-CCZ4) but the results of
the boundary PDE are read off the lower-dimensional boundary.

This gives rise to decouple the two meanings of the term ``boundary'': The
mathematical simulation boundary does not neccessarily need to correspond to
the physical boundary. The generic formalization of the DG approach reads as
follows: The prototypic PDE \eqref{eq:apx-bc-base-pde} functions
could be composed as a ``meta'' PDE where any PDE term $X\in \{B, S\}$ is
substituted by
\begin{equation}
X = \alpha X_\text{FO-CCZ4} + (1-\alpha) X_\text{Boundary}
\end{equation}
Here, the scalar field $\alpha$ encodes the simulation domain, where
$\alpha=1$ means ``within'' the domain, $\alpha=0$ means ``outside'' the
domain and any $\alpha \in (0,1)$ corresponds to the diffusive interface
between inside and outside. An example for $\alpha(\vec x)$ would be the
radially symmetric Logistic function
\begin{equation}
\alpha(\vec r) = \frac{1}{1 + \exp \left\{ -k(r - r_0)  \right\}}
\end{equation}
with $r_0$ encoding the de-facto extend of the physical domain (\ie a typical
value in a BNS merger would be $r_0 \sim 2000M$) and $k$ encodes the width of
the diffusive interface (a typical value would be $k=1$). This idea was
rigorously implemented for linear elasticity equations in \cite{Tavelli2018},
but without solving a PDE outside the physical domain.

\subsection[
  Exact wave-absorbing Riemann Solver
]{Exact Riemann Solver for wave-absorbing Boundary Conditions}
For the FO-CCZ4 system the full eigenstructure is known, all eigenvalues and
eigenvectors for typical gauge choices are provided at \cite{Dumbser2017}.
The pure computational costs of evaluating the full eigenvector system of
FO-CCZ4 makes it way too expensive to adopt an exact Riemann solver in a
numerical scheme. However, this cost might be worth being spent in the
boundary: Ideally, thanks to AMR, there are only a small number of boundary
cells. This was implemented in the robustness tests for FO-CCZ4, however
\emph{greasing waves} accumulate in the corners of the rectangular domains.
% I also think it failed for the single BH but don't remember exactly any more. 
