%
% This is an attemp to graphically display the CCZ4 equations
% in a fancy way. It is rendered as standalone latex because
% there are so many tricks that this can be seen as a "Tikz-like"
% figure.
%
%

% \documentclass[12pt, tikz]{scrreprt}
\documentclass{standalone}

%\usepackage[tightpage]{preview}

\usepackage[T1]{fontenc}
\usepackage{textcomp}
\usepackage[utf8x]{inputenc}

% Left align all equations:
\usepackage[fleqn]{amsmath}
\setlength{\mathindent}{0pt}

\usepackage[table]{xcolor}
\usepackage{array,multirow,graphicx,mathtools, %
  nicefrac,booktabs,tikz,adjustbox,hyperref,cite,wrapfig,etoolbox}

\AtBeginDocument{\let\latexlabel\label}% was for tufte

\begin{document}
\begingroup%
%% Common definitions require by both the CCZ4 system table as well
%% as the Riemann helpers table

%
%% Determine how to fix this...
%\setlength{\mathindent}{0pt}% amsmath, also require fleqn locally somehow
%
% left aligned vector
\newenvironment{pvector}{\begin{pmatrix*}[l]}{\end{pmatrix*}}%
%
% Wrap aligns inside the big CCZ4 table so the table can be
% put wiht background colors
\makeatletter%
\newcommand{\embedEQ}[1]{ {\CT@everycr{\the\everycr} #1 }}%
\makeatother%
%
% the parbox is important for getting a width
%\newcommand{\embedNCP}[1]{ \parbox{5cm}{\embedEQ{#1}} }
\newcommand{\longNCP}[1]{ \parbox{4.0cm}{\embedEQ{#1}} }%
\newcommand{\longSource}[1]{ \parbox{4.0cm}{\embedEQ{#1}} }%
%
\newcommand{\NCP}[1]{{\left[#1\right]_\text{NCP}}}%
\newcommand{\SRC}[1]{{\left[#1\right]_\text{SRC}}}%
%
% colorized CCZ4 parameters
\newcommand{\param}[1]{\textcolor{red}{#1}}%
% colorized CCZ4 matter sources
\newcommand{\matter}[1]{\textcolor{blue}{#1}}%
%
% define colors for CCZ4 variables
\definecolor{alpha}{HTML}{D1F2A5}%
\definecolor{beta}{HTML}{FFE3EB}%
\definecolor{gamma}{HTML}{FF9F80}%
\definecolor{phi}{HTML}{FFDFAB}%
%
\definecolor{Atilde}{HTML}{93DFB8} %FFC48C}
\definecolor{K}{HTML}{FFC8BA} %EFFAB4}
\definecolor{Theta}{HTML}{E3AAD6} %FFB397} % too dark: F56991
\definecolor{Gamma}{HTML}{B5D8EB} %EFFAB4}
\definecolor{b}{HTML}{FFBDD8} %CAE2D0}
%
\colorlet{A}{alpha}%
\colorlet{B}{beta}%
\colorlet{D}{gamma}%
\colorlet{P}{phi}%
%
% leftovers:
%\definecolor{D}{HTML}{FFB397}
\definecolor{varPP}{HTML}{DCEBFA}%
\definecolor{varK0}{HTML}{FCF3CD}%
%\definecolor{K}{HTML}{84C1FF}
%
\newcommand{\einefarbe}[1]{#1}%
%
% background colors
% yields errors:
\newcommand{\colorSymbol}[1]{ \cellcolor{#1} }%
\newcommand{\colorNCP}[1]{ \cellcolor{#1!50!white} }%
\newcommand{\colorSource}[1]{ \cellcolor{#1!60!white} }%
%
% Rotate text in multirow table (source: https://tex.stackexchange.com/a/89116)
\newcommand{\verticalrow}[2]{\parbox[t]{2mm}{\multirow{#1}{*}{\rotatebox[origin=c]{90}{#2}}}}%
%
% Integrals with long limits above/below
\newcommand{\Int}[2]{\int\limits_{\mathrlap{#1}}^{\mathrlap{#2}}}%
%
% This fully works, but of course makes one equation per
% 
\providecommand{\eqnNum}[2]{%
%%\addtocounter{equation}{1}% works, but instead use refstepcounter for label
%%
%% Works, but disabled for the time being for display
%\makeatletter%
%\def\@currentlabel{#2}%
%\renewcommand{\theequation}{\arabic{equation}.#2}%
%\makeatother%
%%
%%
\refstepcounter{equation}%%%\label{#1} % This line works, keep it!!!
\latexlabel{#1}% Using \latexlabel instead of label for tufte.
(\theequation)% mimic the display of an equation
}%
%
% the same but with subequtions...
\providecommand{\subEqnNum}[2]{%
	%\addtocounter{equation}{1}% works, but instead use refstepcounter for label
	%
	% test currentlabel thing
%	\makeatletter%
%	\def\@currentlabel{#2}%
%	\renewcommand{\thesubequation}{\themainequation.#2}%
%	\makeatother%
	% redefine \theequation
	%
	\refstepcounter{subequation}\label{#1} % This line works, keep it!!!
	(\theequation)% mimic the display of an equation
}%
%
%
\newcommand{\setAstronumTableSizes}{
\setlength{\abovedisplayskip}{3pt}%
\setlength{\belowdisplayskip}{3pt}%
\setlength{\abovedisplayshortskip}{3pt}%
\setlength{\belowdisplayshortskip}{3pt}%
}%%
%
% This is an attemp to graphically display the CCZ4 equations
% in a fancy way. It is rendered as standalone latex because
% there are so many tricks that this can be seen as a "Tikz-like"
% figure.
%
% This is a tex file supposed for inclusion.
% Make sure you include common-definitions.tex before including this file.
% For a standalone version, see standalone-astronum-ccz4-table.tex.
%
%\begingroup % group of no spacing around align environments
\setAstronumTableSizes
\begin{tabular}{lllll}
\toprule
& $Q_i$ & Nonconservative product NCP$_a(Q) = B^{ib}_a(Q) \partial_i Q_b$ & 
Algebraic 
source $S_{a}(Q)$ & Eqn. \\
\midrule
\verticalrow{6}{ODE-ADM}
& \colorSymbol{alpha} $\ln \alpha$
& \colorNCP{alpha} \longSource{\begin{equation*}
0
\end{equation*}}
& \colorSource{alpha} \longSource{\begin{equation*}
{ \beta^k A_k } - \alpha \param{g}(\alpha) ( K - K_0 - 2\Theta {c} )
\end{equation*}}
& \colorSymbol{alpha} \eqnNum{eq.foccz4.alpha}{$\alpha$}\label{eq.foccz4.alpha}  \\
%
%
& \colorSymbol{beta} $\beta^i$
& \colorNCP{beta} \longSource{\begin{align*}
0
\end{align*}}
& \colorSource{beta} \longSource{\begin{equation*}
\param s \beta^k B_k^i + \param s \, \param f \, b^i
\end{equation*}} 
& \colorSymbol{beta} \eqnNum{eq.foccz4.beta}{$\beta$}  \\
%
%
& \colorSymbol{gamma} $\tilde \gamma_{ij}$
& \colorNCP{gamma} \longSource{\begin{align*}
0
\end{align*}}
& \colorSource{gamma} \longSource{\begin{align*}
&{\beta^k 2 D_{kij} + \tilde\gamma_{ki} B_{j}^k  + \tilde\gamma_{kj} B_{i}^k - \nicefrac{2}{3}\tilde\gamma_{ij} B_k^k }
%\\&
	- 2\alpha \big( \tilde A_{ij} - {\nicefrac{1}{3}~ \tilde \gamma_{ij} \textnormal{tr}{\tilde A} } \big)
  - { \nicefrac{1}{\param{\tilde\tau}} ( \tilde{\gamma} -1 ) \, \tilde{\gamma}_{ij}}
\end{align*}}
%\end{align*}}
& \colorSymbol{gamma} \eqnNum{eq.foccz4.gamma}{$\gamma$}
\\
%  
% 
& \colorSymbol{phi} $\ln \phi$
& \colorNCP{phi} \longSource{\begin{align*}
0
\end{align*}}
& \colorSource{phi} \longSource{\begin{align*}
&{ \beta^k P_k } + \nicefrac{1}{3} \left( \param \alpha K - {B_k^k} \right)
\end{align*}} 
& \colorSymbol{phi} \eqnNum{eq.foccz4.phi}{$\phi$} \\
\midrule
\verticalrow{11}{SO-CCZ4}
& \colorSymbol{Atilde} $\tilde A_{ij}$
& \colorNCP{Atilde} \longSource{\begin{align*}
~-\beta^k \partial_k\tilde A_{ij}
+ \phi^2 \left[ -\nabla_i\nabla_j \alpha +  \alpha \left( R_{ij} + \nabla_i Z_j  +  \nabla_j Z_i \right) \right]^\text{TF}_\text{NCP}
\end{align*}}
& \colorSource{Atilde} \longSource{\begin{align*}
&{ \tilde A_{ki} B_j^k + \tilde A_{kj} B_i^k - \nicefrac{2}{3}\,\tilde A_{ij} B_k^k }~
\nicefrac{1}{3} \tilde\gamma_{ij}
-
\phi^2 \left[ -\nabla_i\nabla_j \alpha +  \alpha \left( R_{ij} + \nabla_i Z_j  +  \nabla_j Z_i \right) \right]^\text{TF}_\text{SRC}
%
\\&
+ \alpha \tilde A_{ij}(K - 2 \Theta {c} )  - 2 \alpha\tilde A_{il} \tilde\gamma^{lm} \tilde A_{mj}  - \nicefrac 1{\param{\tilde\tau}} \, \tilde{\gamma}_{ij} \, \textnormal{tr}{\tilde A}    
~\matter{
-\phi^4 8 \pi \left( {S_{ij}} - \nicefrac{1}{3}\, \tau \tilde g_{ij} \right) }
\end{align*}}
& \colorSymbol{Atilde} \eqnNum{eq.foccz4.atilde}{$\tilde A$} \\
%
%
& \colorSymbol{K} $K$
& \colorNCP{K} $~ - \beta^k \partial_k K  + \left[ \nabla^i \nabla_i \alpha - \alpha( R + 2 \nabla_i Z^i) \right]_\text{NCP}$
& \colorSource{K} \longNCP{\begin{align*}
\alpha K (K - 2 \, \Theta \,\param{c} ) - 3\alpha\param{\kappa_1}(1+\param{\kappa_2})\Theta
- \left[ \nabla^i \nabla_i \alpha - \alpha( R + 2 \nabla_i Z^i) \right]_\text{SRC}
 + \matter{4\pi {(S-3\tau)}}
\end{align*}} 
& \colorSymbol{K} \eqnNum{eq.foccz4.trK}{$K$}\\
%
%
& \colorSymbol{Theta} $\Theta$
& \colorNCP{Theta} $~ - \beta^k\partial_k\Theta  -  \nicefrac{1}{2}~\alpha {\param e^2} \left[ R + 2 \nabla_i Z^i \right]_\text{NCP}$
& \colorSource{Theta} \longNCP{\begin{align*}
&\nicefrac{1}{2}~\alpha {\param e^2} ( \nicefrac{2}{3} K^2 - \tilde{A}_{ij} \tilde{A}^{ij} ) - \alpha \Theta K \param{c} - {Z^i \alpha A_i} - \alpha\param{\kappa_1}(2+ \param{\kappa_2})\Theta~ \matter{- 8\pi \alpha {\tau}}
\\
&+ \nicefrac{1}{2}~\alpha {\param e^2} \left[ R + 2 \nabla_i Z^i \right]_\text{SRC}
\end{align*}}
& \colorSymbol{Theta} \eqnNum{eq.foccz4.theta}{$\Theta$} \\
%
%
& \colorSymbol{Gamma} $\hat \Gamma^i$
& \colorNCP{Gamma} \longNCP{\begin{align*}
&- \beta^k \partial_k \hat \Gamma^i + \nicefrac{4}{3}~ \alpha \tilde{\gamma}^{ij} \partial_j K  - 2 \alpha \tilde{\gamma}^{ki} \partial_k \Theta \\
&- \param s\tilde{\gamma}^{kl} \partial_{(k} B_{l)}^i
- \nicefrac{\param s}{3}~ \tilde{\gamma}^{ik}  \partial_{(k} B_{l)}^l - { \param s 2 \alpha \tilde{\gamma}^{ik}  \tilde{\gamma}^{nm} \partial_k \tilde{A}_{nm}   }
\end{align*}}
& \colorSource{Gamma} \longSource{\begin{align*}
&{ \nicefrac{2}{3} \tilde{\Gamma}^i B_k^k - \tilde{\Gamma}^k B_k^i  } +
       2 \alpha ( \tilde{\Gamma}^i_{jk} \tilde{A}^{jk} - 3 \tilde{A}^{ij} P_j ) - 
       2 \alpha \tilde{\gamma}^{ki} \left( \Theta A_k + \nicefrac{2}{3} K Z_k \right)
       \matter{ - 16\pi \alpha \tilde{\gamma}^{ij} {S_j} }
\\&	-	 2 \alpha \tilde{A}^{ij} A_j 
	   - 4\param s \alpha \tilde{\gamma}^{ik} D_k^{nm} \tilde{A}_{nm}
 + 2\param{\kappa_3} \left( \nicefrac{2}{3}~ \tilde{\gamma}^{ij} Z_j B_k^k - \tilde{\gamma}^{jk} Z_j B_k^i \right)
%\\& 
- 2 \alpha \param{\kappa_1} \tilde{\gamma}^{ij} Z_j 
\end{align*}} 
& \colorSymbol{Gamma} \eqnNum{eq.foccz4.gammahat}{$\hat\Gamma$}\\
%
%
& \colorSymbol{b} $b^i$
% Note, the following is very blurry. Of course there needs to be a proper
% seperation of $\partial \hat\Gamma$ into NCP and Source. The same is true
% for the Riemann scalar, Ricci tensor and Z 3-vector. But this review is so
% small that we just don't put it in here.
& \colorNCP{b} $~ - \param s \beta^k \partial_k b^i $
& \colorSource{b} \longNCP{\begin{align*}
\param s (  \partial_t \hat\Gamma^i - \beta^k \partial_k \hat \Gamma^i - \param \eta b^i )
\end{align*}}
& \colorSymbol{b} \eqnNum{eq.foccz4.auxb}{$b$}\\
%
%
\midrule
%
%
\verticalrow{8}{FO-CCZ4}
& \colorSymbol{A} $A_k$
& \colorNCP{A} \longNCP{\begin{align*}
&- {\beta^l \partial_l A_k} + \alpha \param g(\alpha) \left( \partial_k K - \partial_k K_0 - 2 \param c \partial_k \Theta \right) \\
&+ {\param s \, \alpha \, \param g(\alpha) \tilde{\gamma}^{nm} \partial_k \tilde{A}_{nm} }
\end{align*}}
& \colorSource{A} \longSource{\begin{align*}
&- {\param s \, \alpha \, \param g(\alpha) \partial_k \tilde{\gamma}^{nm} \tilde{A}_{nm} } \\
&-\alpha A_k \left( K - K_0 - 2 \Theta \param c \right) \left( \param g(\alpha) + \alpha  \param g'(\alpha)  \right) + B_k^l ~A_{l}
\end{align*}} 
& \colorSymbol{A} \eqnNum{eq.foccz4.auxA}{$A_k$}\\
%
%
& \colorSymbol{B} $B_k^i$
& \colorNCP{B} \longNCP{\begin{align*}
&- \param s\beta^l \partial_l B_k^i - \param s\big(  \param f \partial_k b^i - { \param \mu \, \tilde{\gamma}^{ij} \left( \partial_k P_j - \partial_j P_k \right) } \\
& + \param \mu \, \tilde{\gamma}^{ij} \tilde{\gamma}^{nl} \left( \partial_k D_{ljn} - \partial_l D_{kjn} \right)  \big)
\end{align*}}
& \colorSource{B} $ B^l_k~B^i_l $
& \colorSymbol{B} \eqnNum{eq.foccz4.auxB}{$B$}\\
%
%
& \colorSymbol{D} $D_{kij}$
& \colorNCP{D} \longNCP{\begin{align*}
& - {\beta^l \partial_l D_{kij}}  
         - \nicefrac{\param s}{2}~ \tilde{\gamma}_{mi} \partial_{(k} {B}_{j)}^m
         - \nicefrac{\param s}{2}~ \tilde{\gamma}_{mj} \partial_{(k} {B}_{i)}^m
\\&		 + \nicefrac{\param s}{3}~ \tilde{\gamma}_{ij} \partial_{(k} {B}_{m)}^m   		 +  \alpha \partial_k \tilde{A}_{ij}
		-  {\nicefrac{1}{3}~ \alpha \tilde{\gamma}_{ij} \tilde{\gamma}^{nm} \partial_k \tilde{A}_{nm} } 
\end{align*}} 
& \colorSource{D} \longSource{\begin{align*}
& B_k^l D_{lij} + B_j^l D_{kli} + B_i^l D_{klj} - \nicefrac{2}{3}~ B_l^l D_{kij} + { \nicefrac{1}{3}~ \alpha \tilde{\gamma}_{ij} \partial_k \tilde{\gamma}^{nm} \tilde{A}_{nm} } \\
& - \alpha A_k ( \tilde{A}_{ij} - \nicefrac{1}{3}~ \tilde{\gamma}_{ij} \textnormal{tr} \tilde{A} )
\end{align*}} 
& \colorSymbol{D} \eqnNum{eq.foccz4.auxD}{$D$} \\
%
%
& \colorSymbol{P} $P_k$
& \colorNCP{P} \longNCP{\begin{align*}
{\beta^l \partial_l P_{k} - \nicefrac{1}{3} ~ \alpha \partial_k K
+ \nicefrac{1}{3} ~ \partial_{(k} {B}_{i)}^i  } - \nicefrac{\param s}{3} ~ \alpha \tilde{\gamma}^{nm} \partial_k \tilde{A}_{nm}
\end{align*}}
& \colorSource{P}
$\nicefrac{1}{3} ~ \alpha A_k K + B_k^l P_l + \nicefrac{\param s}{3} ~ \alpha 
\partial_k \tilde{\gamma}^{nm} \tilde{A}_{nm}$
& \colorSymbol{P} \eqnNum{eq.foccz4.auxP}{$P$}
\\
\bottomrule
\end{tabular}%
%\endgroup% group of no spacing around align environments%
\endgroup%
% Thanks to the begingroup/endgroup construct,
% we can also add the other table next to it, just uncomment
% the code below:
%\begingroup%
%%% Common definitions require by both the CCZ4 system table as well
%% as the Riemann helpers table

%
%% Determine how to fix this...
%\setlength{\mathindent}{0pt}% amsmath, also require fleqn locally somehow
%
% left aligned vector
\newenvironment{pvector}{\begin{pmatrix*}[l]}{\end{pmatrix*}}%
%
% Wrap aligns inside the big CCZ4 table so the table can be
% put wiht background colors
\makeatletter%
\newcommand{\embedEQ}[1]{ {\CT@everycr{\the\everycr} #1 }}%
\makeatother%
%
% the parbox is important for getting a width
%\newcommand{\embedNCP}[1]{ \parbox{5cm}{\embedEQ{#1}} }
\newcommand{\longNCP}[1]{ \parbox{4.0cm}{\embedEQ{#1}} }%
\newcommand{\longSource}[1]{ \parbox{4.0cm}{\embedEQ{#1}} }%
%
\newcommand{\NCP}[1]{{\left[#1\right]_\text{NCP}}}%
\newcommand{\SRC}[1]{{\left[#1\right]_\text{SRC}}}%
%
% colorized CCZ4 parameters
\newcommand{\param}[1]{\textcolor{red}{#1}}%
% colorized CCZ4 matter sources
\newcommand{\matter}[1]{\textcolor{blue}{#1}}%
%
% define colors for CCZ4 variables
\definecolor{alpha}{HTML}{D1F2A5}%
\definecolor{beta}{HTML}{FFE3EB}%
\definecolor{gamma}{HTML}{FF9F80}%
\definecolor{phi}{HTML}{FFDFAB}%
%
\definecolor{Atilde}{HTML}{93DFB8} %FFC48C}
\definecolor{K}{HTML}{FFC8BA} %EFFAB4}
\definecolor{Theta}{HTML}{E3AAD6} %FFB397} % too dark: F56991
\definecolor{Gamma}{HTML}{B5D8EB} %EFFAB4}
\definecolor{b}{HTML}{FFBDD8} %CAE2D0}
%
\colorlet{A}{alpha}%
\colorlet{B}{beta}%
\colorlet{D}{gamma}%
\colorlet{P}{phi}%
%
% leftovers:
%\definecolor{D}{HTML}{FFB397}
\definecolor{varPP}{HTML}{DCEBFA}%
\definecolor{varK0}{HTML}{FCF3CD}%
%\definecolor{K}{HTML}{84C1FF}
%
\newcommand{\einefarbe}[1]{#1}%
%
% background colors
% yields errors:
\newcommand{\colorSymbol}[1]{ \cellcolor{#1} }%
\newcommand{\colorNCP}[1]{ \cellcolor{#1!50!white} }%
\newcommand{\colorSource}[1]{ \cellcolor{#1!60!white} }%
%
% Rotate text in multirow table (source: https://tex.stackexchange.com/a/89116)
\newcommand{\verticalrow}[2]{\parbox[t]{2mm}{\multirow{#1}{*}{\rotatebox[origin=c]{90}{#2}}}}%
%
% Integrals with long limits above/below
\newcommand{\Int}[2]{\int\limits_{\mathrlap{#1}}^{\mathrlap{#2}}}%
%
% This fully works, but of course makes one equation per
% 
\providecommand{\eqnNum}[2]{%
%%\addtocounter{equation}{1}% works, but instead use refstepcounter for label
%%
%% Works, but disabled for the time being for display
%\makeatletter%
%\def\@currentlabel{#2}%
%\renewcommand{\theequation}{\arabic{equation}.#2}%
%\makeatother%
%%
%%
\refstepcounter{equation}%%%\label{#1} % This line works, keep it!!!
\latexlabel{#1}% Using \latexlabel instead of label for tufte.
(\theequation)% mimic the display of an equation
}%
%
% the same but with subequtions...
\providecommand{\subEqnNum}[2]{%
	%\addtocounter{equation}{1}% works, but instead use refstepcounter for label
	%
	% test currentlabel thing
%	\makeatletter%
%	\def\@currentlabel{#2}%
%	\renewcommand{\thesubequation}{\themainequation.#2}%
%	\makeatother%
	% redefine \theequation
	%
	\refstepcounter{subequation}\label{#1} % This line works, keep it!!!
	(\theequation)% mimic the display of an equation
}%
%
%
\newcommand{\setAstronumTableSizes}{
\setlength{\abovedisplayskip}{3pt}%
\setlength{\belowdisplayskip}{3pt}%
\setlength{\abovedisplayshortskip}{3pt}%
\setlength{\belowdisplayshortskip}{3pt}%
}%%
%% Based on the standalone ccz4 table.
%
% This is not a standalone document, it needs a to be included, either
% in the thesis.tex or in a standalone version.
% 
% For proper usage, requires a couple of packages, see the standalone
% versions.
%
\begingroup % group of no spacing around align environments
\setAstronumTableSizes
%% display all formulas compact
%\let\OLDdisplaystyle\displaystyle
%\let\displaystyle\textstyle
%
%\begin{subequations}
\begin{tabular}{llllr}
\toprule
& $T$ & $T_\text{NCP}(Q, \nabla Q)$: Nonconservative part & $T_\text{SRC}(Q)$: 
Algebraic part
& Eqn.
\\
\midrule
\verticalrow{6}{ODE-ADM}
& \colorSymbol{A} $\tilde\Gamma^k_{ij}$
& \colorNCP{A} $0$
& \colorSource{A} \longSource{\begin{equation*}
\tilde{\gamma}^{kl} \left( D_{ijl} + D_{jil} - D_{lij} \right)
\end{equation*}}
& \colorSymbol{A} \eqnNum{eq.foccz4.riemmann.gtilde}{$\tilde{\Gamma}$} \\
%
%
& \colorSymbol{B} $\partial_k\tilde\Gamma^m_{ij}$
& \colorNCP{B} \longNCP{\begin{flalign*}
\tilde{\gamma}^{ml} \left( \partial_{(k} {D}_{i)jl} + \partial_{(k} {D}_{j)il} - \partial_{(k} {D}_{l)ij} \right)
\end{flalign*}}
& \colorSource{B} \longSource{\begin{equation*}
-2 D_k^{ml} \left( D_{ijl} + D_{jil} - D_{lij} \right)
\end{equation*}}
& \colorSymbol{B} 
\eqnNum{eq.foccz4.riemmann.dgtilde}{$\partial\tilde{\Gamma}$}  \\
%
%
& \colorSymbol{A} $\Gamma^k_{ij}$
& \colorNCP{A} \longNCP{$0$}
& \colorSource{A} \longSource{\begin{flalign*}
\SRC{\tilde\Gamma^k_{ij}}
- \tilde{\gamma}^{kl} \left( \tilde{\gamma}_{jl} P_i + \tilde{\gamma}_{il} P_j - \tilde{\gamma}_{ij} P_l \right)
\end{flalign*}}
& \colorSymbol{A} \eqnNum{eq.foccz4.riemmann.gamma}{$\Gamma$}
\\
%  
% 
& \colorSymbol{B} $\partial_k \Gamma^m_{ij}$
& \colorNCP{B} \longNCP{\begin{flalign*}
  +\tilde{\gamma}^{ml} \left( \partial_{(k} {D}_{i)jl} + \partial_{(k} {D}_{j)il} - \partial_{(k} {D}_{l)ij} \right) 
  \\
   - \tilde{\gamma}^{ml} \left( \tilde{\gamma}_{jl} \partial_{(k} {P}_{i)} + \tilde{\gamma}_{il} \partial_{(k} {P}_{j)}
  - \tilde{\gamma}_{ij} \partial_{(k} {P}_{l)} \right)
\end{flalign*}}
& \colorSource{B} \longSource{\begin{flalign*}
% -2 D_k^{ml} \left( D_{ijl} + D_{jil} - D_{lij} \right)
& \SRC{\partial_k \tilde\Gamma^m_{ij}}
+ 2 D_k^{ml} \left( \tilde{\gamma}_{jl} P_i + \tilde{\gamma}_{il} P_j- 
\tilde{\gamma}_{ij} P_l \right)
\\
& - 2 \tilde{\gamma}^{ml} \left(  D_{kjl} P_i + D_{kil} P_j  - D_{kij} P_l \right)
\end{flalign*}}
& \colorSymbol{B} \eqnNum{eq.foccz4.riemmann.dgamma}{$\partial\Gamma$}  \\
\midrule
\verticalrow{11}{SO-CCZ4}
& \colorSymbol{Atilde} $R^m_{ikj}$
& \colorNCP{Atilde} \longNCP{\begin{flalign*}
& \NCP{\partial_k \Gamma^m_{ij}} - \NCP{\partial_j \Gamma^m_{ik}}
\\
& + \NCP\Gamma^l_{ij} \NCP\Gamma^m_{lk} - \NCP\Gamma^l_{ik} \NCP\Gamma^m_{lj}
\end{flalign*}}
& \colorSource{Atilde} \longSource{\begin{flalign*}
& \SRC{\partial_k \Gamma^m_{ij}} - \SRC{\partial_j \Gamma^m_{ik}}
\\
& + \SRC\Gamma^l_{ij} \SRC\Gamma^m_{lk} - \SRC\Gamma^l_{ik} \SRC\Gamma^m_{lj}
\end{flalign*}}
& \colorSymbol{Atilde} \eqnNum{eq.foccz4.riemmann}{$R^m_{ikj}$} \\
%
%
& \colorSymbol{K} $R_{ij}$
& \colorNCP{K} $\NCP R^m_{imj}$
& \colorSource{K} \longNCP{\begin{flalign*}
\SRC R^m_{imj}
\end{flalign*}}
& \colorSymbol{K} \eqnNum{eq.foccz4.ricci}{$R_{ij}$}  \\
%
%
& \colorSymbol{Theta} $R$
& \colorNCP{Theta} $\phi^2 \tilde\gamma^{ij} \NCP R^i_i$
& \colorSource{Theta} \longNCP{\begin{flalign*}
\phi^2 \tilde\gamma^{ij}  \SRC R^i_i
\end{flalign*}}
& \colorSymbol{Theta} \eqnNum{eq.foccz4.ricciscalar}{$R^i_i$} \\
%
%
\midrule
%
%
& \colorSymbol{Gamma} $\nabla_i\nabla_j \alpha$
& \colorNCP{Gamma} \longNCP{\begin{flalign*}
\alpha \partial_{(i} A_{j)}
\end{flalign*}}
& \colorSource{Gamma} \longSource{\begin{flalign*}
\alpha A_i A_j - \alpha \SRC \Gamma^k_{ij} A_k
\end{flalign*}}
& \colorSymbol{Gamma} \eqnNum{eq.foccz4.nna}{$\nabla\nabla\alpha$} \\
%
%
& \colorSymbol{b} $\nabla^i\nabla_i \alpha$
& \colorNCP{b} $\phi^2 \tilde\gamma^{ij} \NCP{\nabla_i \nabla_j \alpha} $
& \colorSource{b} \longNCP{\begin{flalign*}
\phi^2 \tilde\gamma^{ij} \SRC{\nabla_i \nabla_j \alpha}
\end{flalign*}}
& \colorSymbol{b} \eqnNum{eq.foccz4.laplacealpha}{$\Delta\alpha$} \\
%
%
\midrule
%
%
\verticalrow{8}{FO-CCZ4}
& \colorSymbol{A} $\tilde\Gamma^i$
& \colorNCP{A} \longNCP{\begin{flalign*}
0
\end{flalign*}}
& \colorSource{A} \longSource{\begin{flalign*}
\tilde\gamma^{jl} \SRC{\tilde\Gamma^i_{jl}}
\end{flalign*}}
& \colorSymbol{A} \eqnNum{eq.foccz4.gtilde}{$\tilde{\Gamma}$} \\
%
%
& \colorSymbol{B} $\partial_k \tilde\Gamma^i$
& \colorNCP{B} \longNCP{\begin{flalign*}
\tilde\gamma^{jl} \NCP{\partial_k \tilde\Gamma^i_{jl}}
\end{flalign*}}
& \colorSource{B} $ -2 D^{jl}_k \SRC{\tilde\Gamma^i_{jl}} + \tilde\gamma^{jl} 
\SRC{\partial_k \tilde\Gamma^i_{jl}}$
& \colorSymbol{B} \eqnNum{eq.foccz4.dgtilde}{$\partial\Delta\alpha$}\\
%
%
& \colorSymbol{D} $Z^i$
& \colorNCP{D} \longNCP{\begin{flalign*}
0
\end{flalign*}} 
& \colorSource{D} \longSource{\begin{flalign*}
\frac 12 \phi^2 \left( \hat\Gamma^i  -   \tilde\Gamma^i\right)
\end{flalign*}}
& \colorSymbol{D} \eqnNum{eq.foccz4.Z}{$Z$}  \\
%
%
& \colorSymbol{P} $\nabla_i Z_j$
& \colorNCP{P} \longNCP{\begin{flalign*}
\frac 12 \tilde\gamma_{jl} \left( \partial_i \hat\Gamma^l 
- \NCP{ \partial_i \tilde\Gamma^l} \right)
\end{flalign*}}
& \colorSource{P} \longSource{\begin{flalign*}
\frac 12 \tilde\gamma_{jl} \left( 
0
- \SRC{ \partial_i \tilde\Gamma^l} \right)
+ D_{ijl} \left( \hat\Gamma^l - \tilde\Gamma^l \right)
- \Gamma^l_{ij} Z_l
\end{flalign*}}
& \colorSymbol{P} \eqnNum{eq.foccz4.nZ}{$\nablaZ$}
\\
\bottomrule
%\\
%%
%% Phantom line to preserve the widths.
%%
%% This line is needed to preserve the same table widths as in the
%% CCZ4 table. Just copy the most lengthy expressions here.
%%
%% This however creates artificial height which probably needs truncation.
%& 
%&
%\phantom{
%\longNCP{\begin{flalign*}
%&- {\beta^l \partial_l A_k} + \alpha \param g(\alpha) \left( \partial_k K - \partial_k K_0 - 2 \param c \partial_k \Theta \right) \\
%&+ {\param s \, \alpha \, \param g(\alpha) \tilde{\gamma}^{nm} \partial_k 
%\tilde{A}_{nm} }
%\end{flalign*}}
%}
%& \phantom{\longSource{\begin{flalign*}
%&{ \nicefrac{2}{3} \tilde{\Gamma}^i B_k^k - \tilde{\Gamma}^k B_k^i  } +
%       2 \alpha ( \tilde{\Gamma}^i_{jk} \tilde{A}^{jk} - 3 \tilde{A}^{ij} P_j ) - 
%       2 \alpha \tilde{\gamma}^{ki} \left( \Theta A_k + \nicefrac{2}{3} K Z_k \right)
%       \matter{ - 16\pi \alpha \tilde{\gamma}^{ij} {S_j} }
%\\&	-	 2 \alpha \tilde{A}^{ij} A_j 
%	   - 4\param s \alpha \tilde{\gamma}^{ik} D_k^{nm} \tilde{A}_{nm}
% + 2\param{\kappa_3} \left( \nicefrac{2}{3}~ \tilde{\gamma}^{ij} Z_j B_k^k - \tilde{\gamma}^{jk} Z_j B_k^i \right)
%%\\& 
%- 2 \alpha \param{\kappa_1} \tilde{\gamma}^{ij} Z_j 
%\end{flalign*}}} \\
\end{tabular}%
%\end{subequations}
%
\endgroup % group of no spacing around align environments
%\end{small}
%%
%\endgroup%
\end{document}
 
