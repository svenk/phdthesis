\chapter*{Allgemeinverständliche Zusammenfassung}

\todo{Idee: Diese Seite mit Inkscape machen und netter bebildern.}
Die vorliegende Doktorarbeit besteht aus fünf Kapiteln. Im ersten
Kapitel werden Methoden vorgestellt, um eine bestimmte Art von
Gleichungen, die man ``hyperbolische partielle
Differentialgleichungen'' nennt, zu lösen. Es ist eine schwierige
Aufgabe, dafür Supercomputer zu verwenden, die aus tausenden von
Einzelcomputern bestehen.

Im zweiten Kapitel geht es um Einsteins allgemeine Relativitätstheorie.
Mit ihrer Hilfe können extreme Phänomene im Universum beschrieben 
werden, etwa der Kollaps eines Sternes zu einem schwarzen Loches. Um
diese Vorgänge zu berechnen, werden Näherungen benötigt. Hierzu
eignen sich Computersimulationen, die Filme dieser Vorgänge erstellen.
Da Einsteins Theorie aber Zeit und Raum zusammenfasst, werden diese
in dem Kapitel erst mühsam wieder auseinandergenommen. Auf diese Weise
entstehen eine Vielzahl obengenannter Gleichungen. In dieser Arbeit
wurde eine Formulierung der Einsteingleichungen hergeleitet, die fast
60 Differentialgleichungen umfasst. Es wurde dann demonstriert, wie gut
diese Gleichungen mit Computern gelöst werden können.

Im dritten Kapitel geht es um Flüssigkeiten und Plasmen. Sie eignen
sich, um die Materie in Sterne zu beschreiben. Es werden wieder
Differentialgleichungen hergeleitet, die beschreiben, wie sich ein
stark magnetisches Plasma bewegt. Die Computerlösung dieser Gleichungen
wird anhand zahlreicher Beispiele demonstriert.

Im vierten Kapitel geht es darum, wie man die vorangenannten Techniken
benutzen kann, um Aussagen über eine Art von Doppelsternsystemen zu
machen. Hierbei soll abgeschätzt werden, wie lange es dauert, bis zwei
kollidierende Sterne zu einem einzigen schwarzen Loch werden. Dies
hängt stark von den Eigenschaften von Atomen und ihren Bestandteilen,
den Neutronen und Protonen, ab. Vor drei Jahren wurden solche 
Kollisionen im Weltall mit einem riesigen Messapperat das erste mal
nachgewiesen. Dabei wurden Einsteins Gravitationswellen gemessen (2017
wurde dafür der Nobelpreis verliehen). Dank dieser Messungen sowie den
Computersimulationen kann die Beschaffenheit von Neutronensternen
besser abgeschätzt werden und damit zuletzt die Beschaffenheit von
Elementarteilchen wie sie auch auf der Erde existieren.
