\section{Notation and conventions}
The notation chosen in this work mostly follows
the standard notations in general relativity:
Indices of four-dimensional tensors
\footnote{or symbols, in general like the Christoffel symbols}
are denoted with lower-case greek letters,
as in $X_{\alpha\beta\gamma\dots}^{\mu\nu\eta\dots}$.
These indices run over $[0,1,2,3]$ (or $[1,2,3,4]$, if prefered).
Three-dimensional tensors are denoted with lower-case latin letters,
as in $X_{abc\dots}^{ijk\dots}$. These indices run over $[1,2,3]$.
Explicitely higher dimensional spatial tensors are denoted
with upper case latin letters, as in $X_{ABC\dots}^{IJK\dots}$.
The indices run over $[1,2,3,\dots,n]$ in $n$ spatial
dimensions.

Einstein sum convention is applied all over the text. For symbols
which do not follow the concept of covariance (\ie not described
with differential geometry), the index ordering (upper or lower)
is arbitrarily chosen in order to meet Einstein sum convention.
With such symbols, lower latin indices are also used for various
kind of indices.
For instance, state vector indices are typically given with lower
case indices, but they do not count over $[0\dots 3]$ but over
$[1,\dots,N]$, where $N$ is the number of state vector variables.
Another example are Gaussian basis functions which run over a
specific set $[1,\dots,N]$ where $N$ is the
typically the order of the method/number of basis points.

An higher-dimensional object \emph{may} be written without indices.
In general relativity, this concept is known as
\emph{abstract index notation} and allows to write simple contractions
as scalar products, revealing a descriptive meaning of some terms.
For covariant symbols, this notation is avoided within this text.
However, it is widely adopted for other types, such as three-vectors
or state vectors. Generally, the arrow in $\vec x$ indicates that
$\vec x = (x_1,x_2,x_3)$ \footnote{
  Intentionally, I do not formally distinguish between a definition
  $\vec x = (x_1,x_2,x_3)$ or $\vec x=(x_1,x_2,x_3)^T$, \ie
  column and row vector, because the nature of the object typically
  depends on the surrounding: A numerical or CAS implementation or
  as part of a linear algebra calculation where it's shape matters.
} is a three-vector. In contrast, a symbol in bold typeface such as
$\mathbf x$ or $\boldsymbol x$ only indicates that the symbol is
an abstract object with one or more indices. For instance, fluxes
$\boldsymbol F^{ij}$ are written in bold, meaning that the fundamental
object is $F^{ij}_k$ (which however does not transform like a tensor).
The shortened notation where only a few indices are written out
emphasizes the implementation where only the $n \times n$ matrix is
of interest. Therefore, the index-free notation is adopted \emph{only}
in situations when the clarity shall be preserved, for instance because
objects are decorated with discreteness indicators. Another example is
$Q_k$, this may indicate the $k$th component of the state vector
$Q$, but $\boldmath Q_n^s$ is the whole state vector at spatial position
$x_n$ and time index $t^s$.

Furthermore, since this thesis deals with field theory, almost all
objects are fields, \ie $X=X(t,\vec x)$. Again, this information
can only be provided by context.

\subsection{Units}
This work uses geometric units with gravitational coupling $G=1$, speed of light $c=1$.
In astrophysical simulations, we set the solar mass (mass of the sun) $\Msol=1$ 
and measure subsequently all quantities
in these units. In context of quantum mechanics, natural units are adopted,
where furthermore the reduced Planck constant $\hbar=1$ and the Boltzmann
constant $k_B=1$.

%In this work it is usually tried to avoid to give quantities
%without dimensionful units.

\subsection{Symbols}\label{apx:symbols}
\begin{itemize}
    \item States (state vectors) in PDEs are usually denoted as $Q$ or $u$.
      $V$ is typically the symbol for a primitive vector.
    \item Landau symbol $\mathcal O$ for the physical ``at the 
      order of'' or the computer science complexity.
	\item Levi-Civita-Symbol (total antisymmetric tensor);
	  in two dimensions:
	  \begin{equation}
	  \epsilon^{ik} = \begin{pmatrix} 0 & 1 \\  -1 & 0 
	  \end{pmatrix}
	  \end{equation}
	\item Levi-Civita-Symbol in three dimensions: Spatial 3-Levi-Civita tensor
  	 density $\tilde{\epsilon}$ given by
	 \begin{equation}
	 \tilde{\epsilon}^{ijk} = \gamma^{-\frac{1}{2}} [ijk]\,, \hspace{0.5cm}
	 \tilde{\epsilon}_{ijk} = \gamma^{\frac{1}{2}} [ijk]\,. \\
	 \end{equation}
	 where 
	 $[ijk]$ is the regular total antisymmetric symbol, commonly known as
	 epsilon tensor/Levi-Civita tensor in flat space,
	 \begin{equation}
	 [ijk] = \begin{cases}
      \phantom{-}1 & \text{for even	permutations of $(1,2,3)$}, 
      \\
	 %
	 -1 & \text{for odd permutations}, 
	 \\
	 \phantom{-}0 &  \text{otherwise.}
	 \end{cases}.
	 \end{equation}
	
	
	\item $\Theta=\Theta(x)$ is in general the symbol for
	   the Heaviside step function
	   \begin{equation}
	  \Theta(x)=\begin{cases}
	  1 & x > 0 \\ 0 & x < 0
	  \end{cases} \,,
	  \end{equation}
	  but sometimes also used for other cases\footnote{
	    An example is $Z_\mu = (\Theta, Z_i)$ in the
	    FO-CCZ4 PDE (Chapter~\ref{chapter:gr}).
      }
	\item $\delta$ is either the Dirac delta function
	  $\delta=\delta(x)$, also refered to as unit impulse
	  or Dirac distribution, defined via its property
	  \begin{equation}
	  f(a) = \int f(x) \delta(a) \d x \,.
	  \end{equation}
	  In $n$ spatial dimensions, the symbol
	  is refered to as $\delta^{(n)}(\vec x)$.
	  $\delta$ can also be the discrete version, \ie
	  the Kronecker Delta
	  \begin{equation}
	  \delta_{ij} =
	  \begin{cases}
	  1 & \text{if } i = j \\
	  0 & \text{else (if } i \neq j \text{)}
	  \end{cases}
	  \end{equation}
	
	\item Lower incomplete Gamma function
	\begin{equation}
	\gamma(s;x) = \int_0^x \d t ~ t^{s-1} e^{-t}
	\end{equation}
	
	\item the $p$ norm, or $L^p$ norm, is defined as
	\begin{equation}\label{apx.def.lpnorm}
	  \left| \vec x \right|_p = \left( \sum_{i=1}^k |x_i|^p \right)^{1/p}
	\end{equation}
	with a vector $\vec x \in \mathbb R^k$ and $|r|$ with the absolute of
	a real number $r\in \mathbb R$. The norms define
    $L^p$ function spaces.
    
    \item Symmetric and antisymmetric part of tensors, exemplary for a $(0,2)$ tensor:
       Symmetric part $T_{(ab)} = \nicefrac{1}{2} \left( T_{ab} + T_{ba} \right)$,
       antisymmetric part $T_{[ab]} = \nicefrac 12 \left( T_{ab} - T_{ba} \right)$.
    
    \item Covariant derivative $\nabla_\mu$, exemplary for a scalar field $\phi$,
      a covector field $t^\mu$, a $(2,0)$ tensor $A^{\mu\nu}$:
      \begin{align}
      	\nabla_\mu \phi &= \partial_\mu \phi \\
      	\nabla_\alpha t^\nu &= \partial_\alpha t^\nu + \Gamma^\nu_{\alpha\gamma} t^\gamma \\
      	\nabla_\lambda A^{\mu\nu} &= \sum_{\delta}
      	\partial_\lambda A^{\mu\nu} + \Gamma^\mu_{\delta\lambda} A^{\delta\mu} + \Gamma^{\nu}_{\delta\lambda} A^{\mu\delta}
     \end{align}
   
    \item Lie derivative $\Lie_{\boldsymbol \xi}$, for a vector field $\xi_\mu$ and the same fields from above:
    \begin{align}
    	 \Lie_\xi \phi &= \xi^\alpha \partial_\alpha \phi \\
    	 \Lie_\xi t^\nu &= \xi^\alpha \partial_\alpha t^\nu - t^\alpha \partial_\alpha \xi\nu \\
    	 \Lie_\xi A^{\mu\nu} &= \xi^\alpha \partial_\alpha A^{\mu\nu}
    	   - \xi^\mu \partial_\alpha A^{\alpha\nu}
    	   - \xi^\nu \partial_\alpha A^{\mu\alpha}
   	\end{align}
    Sometimes, $\Lie_{\boldsymbol \xi}$ is written as $\Lie_{\xi}$ for brevety.
    	
\end{itemize}

\subsection{Symbols with canonical physical meaning}

\begin{itemize}
	\item  Christoffel symbols of first kind
	\begin{equation}
     \Gamma^k_{ij} = \frac{1}{2} g^{kl} \left(\partial_i g_{jl} 
     + \partial_j g_{il} - \partial_l g_{ij}\right)
	\end{equation}

	\item 4-metric, expressed by ADM quantities:
    \begin{equation}
	g_{\mu\nu} = 
	\begin{pmatrix}
	-\alpha^2 + \beta_i \beta^i  & \beta_i \\
	\beta_i                      & \gamma_{ij}
	\end{pmatrix}
    \end{equation}
    
    \item See Table \ref{table:overview-gup-symbols} for an overview
      of symbols used within the context of quantum black holes.
	
\end{itemize}

\begin{table}[t]
	\begin{tabularx}{\linewidth}{llX}
	\firsthline
	symbol && semantics \\
	\hline
	$M$ & & Generic total mass of a black hole \\
	$L$ && Any length scale \\
	$T$  & & Hawking temperature \\
	$d$ & & Number of total space-time dimensions \\
	$n$ && Number of total spatial dimensions \\
	$G_\mathrm N$  & & Newton's constant in 3+1 dimensions. \\
	$M_\mathrm{Pl}$ & & Planck mass in 3+1 dimensions. \\
	$L_\mathrm{Pl}$   & & Planck length in 3+1 dimensions. \\
	$\mathbb G(\dots)$ && Newton's constant as differential operator \\
	$M_*$ && Fundamental mass scale in any dimension \\
	$L_*$ && Fundamental length scale in any dimension \\
	$\sqrt{\beta}$ && Length scale in the GUP \\
	$\mathcal M(r)$ & & Cumulative mass distribution \\
	$M_0$ & & Mass of critical black hole configuration \\
	$r_0$ && Radius of critical black hole configuration \\
	$T_\textrm{max}$ && Temperatur of critical black hole (maximum temperature) 
	\\
	$\mathbb T^\mu_\nu$ & & Nonlocal energy-momentum tensor \\
	$\rho(\vec x)$ && Density distribution  \\
	$f_n(r)$ && Metric function in $n$ spatial dimensions \\
%%%		
	$r_\pm$ & & The inner and outer horizons radii. \\
	$r_0$ && The size of the extremal black hole configuration. \\
	$r_C$ && The critical radius, at this radius the temperatures are maximal. 
	\\
	\hline
	\end{tabularx}
	\caption[List of symbols used in the GUP chapter]{
		This is an overview of symbols as they are introduced and used in
		Chapter~\protect\ref{chapter:qgr}, exclusively.
    }\label{table:overview-gup-symbols}
\end{table}


