\begin{fullwidth} % for tufte, stay on 1 page
	\chapter*{Acknowledgement}
	\addcontentsline{toc}{section}{Acknowledgement}
	
	My gratitude goes to my supervisors for the close supervision, patience in
	explaining concepts, the many oppertunities for traveling and exchange
	and meetings at any time and place. Especially I thank my primary supervisor,
	Luciano Rezzolla, for the one of a kind way of remote collaboration since my
	move-out of Frankfurt in 2017. My secondary supervisor, Piero Nicolini,
	is expressed my special thanks for the countless cofees in Otto-Stern-Zentrum.
	
	I also want to thank the large group of Relativistic Astrophysics
	at the Institute for Theoretical Physics (ITP) in Frankfurt,
    especially my fellow students of the hydrodynamics subgroup,
	Federico Guercilena, Jens Papenfort, Elias Most and Luke Bovard for
    many fruitful discussions and exchange of the knowledge.
	I also benefited from insights provided by Filipo Galeazzi,
	Bruno Mundim, Matthias Hanauske and Mariafelicia De Laurentis and most
	of the other people in this large international, fluctuating and colorful
	group.	
	Special thanks go to my collegues Alejandro Cruz Osorio and 
	Roman Gold with which I shared my office at ITP consecutively.

	% However, I also express my gratitude to all other group members: Hector
	% Olivares, Jonas Köhler, Marcio de Avellar, Antonios Tsokaros, Natascha
	% Wechselberger, Ziri Younsi, David Kling and Laura Tolos, just to 
	% mention a few. Given the long time I stayed with this large international, 
	% fluctuating colorful group, I surely forgot some people!
	
	Special thanks goes to the ExaHyPE consortium and in the frontline 
	Michael Dumbser for his	tireless efforts of spreading the knowledge
	about sophisticated numerical methods and for the various invitations to
	Trento University. I express my special thanks to my fellows Dominic Charrier,
	Jean-Matthieu Gallard, Angelika Schwarz, Ben Hazelwood and
	Leonard Rannabauer for their great
	collaboration and the many discussions. Special thanks also
	to Tobias Weinzierl for his relentless efforts of convincing about the
	power of loosing control and various invitations to Durham University.
	Gratitude is further owed to Olindo Zanotti, Michael Bader and
	Vasco Varduhn for start-up support in their research fields.
	
    I also thank the exotic high energy physics (quantum gravity) group at 
    Giersch Science Center/Frankfurt Institute for Advanced Studies
    (GSC/FIAS) in Frankfurt, especially my fellow students
    Marco Knipfer, Antonia Frassino and Michael Wondrak for many inspiring
    discussions.
	
	Last but not least I thank FIAS and ITP for their kind hospitality, 
	furthermore the graduate schools
	HGS-HiRE (HIC for FAIR/GSI) and FIGSS (FIAS) for infrastructure, travel 
	support and softskill
	schooling. For running simulations within this thesis, I enjoyed access to 
	various clusters of the
	of the Gauss Supercomputing Center, especially the LRZ computing center in 
	Munich/Garching
	(SuperMUC and CoolMUC) and the HLHRS in 
	Stuttgart (Hazelhen),
	the CSC in Frankfurt (Loewe and Fuchs), the Durham-based supercomputer 
	(Hamilton) as well as local
	computing facilities in ITP and FIAS. Special thanks goes to the 
	administrative teams of CSC, ITP and FIAS which provided personal support 
	in any condition.
	
	I~recieved fundamental support from 
	the European Union's
	Horizon 2020 Research and Innovation Programme (Grant 671698, call 
	FETHPC-1-2014, project ExaHyPE)
	and the COST Action MP1304 ``NewCompStar''.
	
    In order to appreciate the work of open source software authors, I want to
    cite scientific codes which ask for credits:
	GNU Parallel \cite{Tange2011a}, scipy \cite{SciPy} and matplotlib 
	\cite{matplotlib}. Also technically this work is standing on the shoulder
	of giants, as I made use of the rich contemporary GNU/Linux software
	infrastructure and all the scientific ecosystems --- primarily scientific
	python --- available for free.
	

\newpage
\chapter*{Curriculum Vitae}
\addcontentsline{toc}{section}{Curriculum Vitae}\label{sec:cv}
%
%\begin{figure}
%	\hspace*{-2cm}
%	\includegraphics[width=1.5\textwidth]{cv/sven-fias.jpg}
%	\caption{Photography of Sven (hide this text but show in lof)}
%\end{figure}
%
%
\begin{multicols}{2}
%
\noindent
\includegraphics[width=.9\linewidth]{cv/sven-fias.jpg}
\section*{Personal information}


%\begin{multicols}{2}[%
\begin{description}
	\item[Name] Sven Köppel
	\item[E-Mail] koeppel@fias.uni-frankfurt.de
	\item[Date of Birth] January 31, 1989
	\item[Place of Birth] Königstein (Taunus), Germany
	\item[Web page] 
	\href{http://fias.uni-frankfurt.de/~koeppel}{fias.uni-frankfurt.de/{$\sim$}koeppel}
	\item[Place of residence] Münster (Westfalen), Germany
	\item[Descendants] Lola *2017
\end{description}

\section*{Education}
\begin{description}
	\item[2015] PhD studies, Institut für theoretische Physik,
	   Goethe Universität Frankfurt am Main, Supervisor: Prof. L. Rezzolla.
	\item[2014] Master's thesis, Frankfurt Institute for Advanced Studies,
	   Title: \emph{Ultraviolet improved Black Holes},
	   Supervisor: Prof. P. Nicolini.
	\item[2011] Bachelor's thesis, Institut für theoretische Physik,
	   Title: \emph{The QCD Equation of State in the Early Universe},
	   Goethe Universität Frankfurt am Main, Supervisor: Prof. O. Phillipsen.
    \item[2008] Abitur, Eichendorffschule mit gymnasialer Oberstufe
       Immanuel-Kant, Kelkheim (Taunus).
\end{description}

\section*{Peer-reviewed Publications}
\begin{itemize}
	\item
	S. Köppel, L. Bovard, L. Rezzolla,
	``Universal relations for binary neutron star lifetimes'',
	published in Astrophysical Journal Letters 2019 \cite{Koeppel2019}
	
    \item 
    F. Fambri, M. Dumbser, S. Köppel, L. Rezzolla, O. Zanotti,
    ``ADER discontnous Galerkin schemes for general-relatvistc ideal 
    magnetohydrodynamics'',
    published in Mon.Not.Roy.Astron.Soc. 2018 \cite{Fambri2018}
    
    \item
    M. Dumbser, F. Guercilena, S. Köppel, L. Rezzolla, O. Zanotti,
    ``Conformal and covariant Z4 formulaton of the Einstein equatonss strongly 
    hyperbolic frst-order reducton and solutons with discontnous Galerkin 
    schemes'',
    published in Phys.Rev.D. 2018 \cite{Dumbser2017}
   
    \item
    A. Frassino, S. Köppel, P. Nicolini,
    ``Geometric model of black hole quantum N-portrait, extradimensions and 
    thermodynamics'',
    published in Entropy 2016~\cite{FKN16}
\end{itemize}

\section*{Upcoming Publications}
\begin{itemize}
	\item
	S. Köppel, M. Knipfer, P. Nicolini,
	``Generalized Uncertainty Principle and Black
	Holes in Higher Dimensional Self Complete
	Gravity'',
	to be published \cite{Knipfer2019}
\end{itemize}

\section*{Peer-reviewed Conference Proceedings}
\begin{itemize}
	\item
	S. Köppel,
	``Towards an exascale code for GRMHD on dynamical spacetimes'',
	published in J.Phys.Conf.Ser. 2018 \cite{Koeppel2017}
	\item
	S. Köppel, M. Knipfer, M. Isi, J. Mureika, P. Nicolini,
	``Generalized uncertainty principle and extra dimensions'',
	published in Springer Proc.Phys. 2018 \cite{Koppel:2017rsf}
\end{itemize}

\vfill{}
\hfill
\textcolor{black!80!white}{$\to$ CV, page 2/2}

\newpage
\small % Page 2
\section*{Presentations, Invitations and Posters}
\begin{itemize}
	\item Predicting gravitational waves on computers, at When Gravitational 
	Waves hit Durham - an ExaHyPE workshop, Durham School of Engineering and 
	Computing Sciences, on 2017-11-30 (invited talk).
	
	\item Towards an Exascale code for MHD on dynamical spacetimes, at Astronum 
	2017, Centre de congres des Saint-Malo, France, on 2017-06-27.
	
	\item Mini Overview, ExaHyPE: Exascale Spacetree ADER-DG, at Mini-Workshop 
	on Preparing for PRACE Exascale Systems, Juelich Supercomputing Centre, 
	Forschungszentrum Juelich, on 2017-06-01 (invited talk).
	
	\item Einsteins Equations in ExaHyPE, at ExaHyPE Research Council, LRZ, 
      Technische Universität M\"unchen, Garching, on 2017-04-03  
      (and 4 more talks there).
	
	\item Probing star collisions with Exascale computers, FIGSS seminar,
	  FIAS, Uni Frankfurt, on 2016-12-12.
	
	\item Donuts in Space, at Palaver SS 2016, ITP Uni Frankfurt, on 2016-05-30
	  (and one more talk there at 2014).
	
	\item On gravity self-completeness on Anti-de Sitter background, at 
	DPG-Frühjahrestagung 2016, Uni Hamburg, on 2016-03-01.
	
	\item ExaHyPE: Various Equations and Quick Adaption to the Users' Needs
	Becoming Open Source, SIAM-CSE, Atlanta (USA), on 2016-02-28
	(invited poster).
	
	\item The Generalized Uncertainty Principle and Extra Dimensions, at 
	Geometry and Physics seminar, DFT at IFIN-HH, Bucharest, on 2015-09-25 
	(invited talk).
	
	\item The Generalized Uncertainty Principle in extra dimensions, at Karl 
	Schwarzschild Meeting on Gravitational Physics 2015, Frankfurt, on 
	2015-07-21.

	\item The quest for physics at shortest scales, at FIAS scientific retreat, 
	Riezlern, on 2015-01-18.
	
	\item Quantum gravity improved black holes, at Astro Coffee, FIAS 0.101, on 
	2015-01-27.
	
	\item On gravity self-completeness in higher dimensions, at 
	DPG-Frühjahrestagung 2015, TU-Berlin, on 2015-03-20.
	
	\item Quantum gravity improved black holes, at DPG Physics School General 
	Relativity 99, DPG Zentrum Bad Honnef, Uni Bonn, on 2014-09-14 (Poster).
\end{itemize}

\subsection*{Publications and Invitations in eLearning-related matter}
\begin{itemize}
	\item
	S. Köppel,
	``Soziale Wissenschaft'',
	published in USE: Universität Studieren / Studieren Erforschen, 2014
	(proceeding).
	
	\item
	S. Köppel,
	``POKAL: Kollaboratives eLearning neu erfunden'',
	published in Medien in der Wissenschaft \cite{Koeppel2013}.
	
	\item UNIversal 2014 Studienkongress, ``Wissenschaft als soziales Netzwerk'',
	  on 2014-07-15 (invited talk).
	
	\item Hessisches eLearning-Fachforum 2014, Hochschule Rhein-Main,
	``eLearning von Studierenden für Studierende,
	Studentische Contests und eLearning-Settings an der Goethe-Universität 
	Frankfurt'' (invited talk).
	
	\item GMW 2013
	(Jahrestagung der Gesellschaft für Medien in der
	Wissenschaft) an der
	Goethe-Universität Frankfurt,
	``POKAL – Kollabioratves eLearning neu erfunden''
	(invited talk).

    \item And 20 more talks about e-Learning at Uni Frankfurt between 2011-2015.
	
\end{itemize}

\section*{Relevant academic work experience}
\begin{description}
	\item[2015-2017] ERC Horizon 2020 Project: ``ExaHyPE: An ExaScale 
	Hyperbolic PDE Engine'', with PIs Prof. M. Bader, Prof. M. Dumbser, Prof. L. 
	Rezzolla, Prof. T. Weinzierl.
	\item[2010-2016] PhysikOnline: eLearning for physicists, 
	  Center for Scientific Computing, Goethe Universität Frankfurt,
	  Prof. H. Lüdde.
	\item[2015] Uni$\Phi$, OpenScience social publication platform,
	   Studiumdigitale, Institute for Computer Science, Goethe
	   Universität Frankfurt, Prof. D. Krömker.
	\item[2011-2012] Psychology eLearning, Interdisciplinary
	   College Of University Didactics (IKH), Goethe Universität Frankfurt,
       Prof. H. Horz.
    \item[2008-2010] BioKemika: Chief development of a biochemical
       search engine, Institute for biophysical chemistry and Life sciences,
       Goethe Universität Frankfurt, Prof. Clemens Glaubitz.
\end{description}

\section*{Relevant academic grants}
\begin{itemize}
  \item 2015, 2016  ``RiedbergTV'' (20k\euro{}, SeLF eLearning funds)
  \item 2011---2016  ``PhysikOnline'' (300k\euro{}, QSL funds)
  \item 2011, 2013 ``POKAL: PhysikOnline Kollaborative Arbeits- und 
  Lenrplattform'' (25k\euro{} in total, SeLF)
\end{itemize}

\end{multicols}
\end{fullwidth}% tufte
