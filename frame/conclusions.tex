% Overall Conclusions for phdthesis

\vspace{.5cm} % seperation to chapter mark artwork
\begin{fullwidth}

In this thesis, the applicability of sophisticated family of high order and 
communication avoiding dicontinous Galerkin (DG) schemes wer eexamined for
solving partial differential equations in relativistic astrophysics.
These schemes where supplemented with an ADER time integration and a finite
volume limiter in order to deal withspacetime singularities and hydrodynamical 
discontinuities (\ie shocks).
Different to what has been done in literature before, an
explicit split of differential contributions in source terms combined with
a path-conservative treatment, first proposed by Castro and Par\'es in the 
finite-volume context \cite{Castro2006,Pares2006} and later extended also
to ADER-DG schemes in \cite{Dumbser2009a,Dumbser2010}, was carried out. The
schemes were implemented in different state of the art
adaptive mesh refinement (AMR) codes.

In order to employ the Einstein equations for the proposed schemes, a novel 
first order formulation with conformal and constraint correcting features had to
be found: Tthe conformal and covariant Z4 formulation. This formulation was
proven to be strongly hyperbolic for popular gauge choices. The evolution
laws for the ADM quantities $\alpha,\beta^i,\tilde{\gamma}_{ij}$ as well as the
conformal factor $\phi$ 
could be reduced to a pure system of ODEs, thanks to the auxilliary variables
which render derivatives as purely algebraic source terms. These ODEs have
zero eigenvalues and unit eigenvectors, leaving a PDE system with 47 variables
open for hyperbolization. For the remaining system of 47 variables, first
approximate symmetrization by use of ordering constraints is a key ingredient,
and second the dependency of the ADM (composing) variables $\alpha,
\beta^i, \tilde{\gamma}_{ij}$ and $\phi$ only results in a fully
linearly degenerate PDE system. Furthermore, the adoption of the genuinely
non-conservative from (in contrast to a flux-conservative form such as in
\cite{Bona97a,Alic:2009}) was also essential for applying the previously
mentioned simplifications, as the Jacobian $\partial F^i_k / \partial Q^j$
of the flux $F^i_k(Q)$ in direction $k$ also depends on the dynamical
variables. The
quasi-linear form of the flux-conservative system therefore contains
differential terms in the ADM variables. The proof
of strong hyperbolicity was given for two standard gauge choices, \ie 
harmonic lapse and zero shift as well as 1+log slicing with Gamma driver. In
both cases, the full eigenstructure was computed. This is the first time
that hyperbolicity of a first-order reduction of the CCZ4 system is
analyzed, in particular including the Gamma-driver shift condition.
However, the system is \emph{not} symmetric hyperbolic in the sense of
Friedrichs \cite{Friedrichs1954}. Further work
in this direction will be necessary to try and achieve a symmetric
hyperbolic form of FO-CCZ4 with a convex extension.

Subsequently, the equations of relativistic hydrodynamics were brought into a
similar form, with a clear seperation between conserved hydrodynamic flux and
non-conserved differential gravitational source, induced by the curved 
spacetime. This much simpler rewrite did not change the hyperbolic nature of
the PDE system. At this  stage, the development of exactly well-balanced
numerical methods for the GRMHD equations is still out of scope, but further
developments in this direction would definitely deserve attention.

In a variety of tests, the applicability of the ADER-DG scheme was demonstrated
on the Einstein-Euler system. Here, the spacetime evolution was tested still
separately from the hydrodynamic matter evolution, which itself was evolved
on a fixed background spacetime. Nevertheless, these were the first simulations
of black hole spacetimes and (fixed) general relativistic
magnetohydrodynamic flows performed in three spatial dimensions with a
high-order DG code.
All previous simulations of black-hole spacetimes with high-order DG
schemes, in fact, were limited to the one-dimensional case only.

Future research will consider the implemention of the algorithms within
a code which scales well on thousands of codes/computers. Furthermore,
the coupled FO-CCZ4 and GRMHD system, solved with a single numerical scheme,
will be evolved using such a code. This will allow to simulate binary neutron
star merger binaries with the Limiting ADER-DG scheme.

We have also considered a general-relativistic estimate on the treshold mass
which seperates prompt collapse from a delayed collapse scenario in the case of
binary neutron star mergers. As part of a systematic parameter study,
a large number of binary neutron star merger simulations was carried out,
adopting a series of realistic nuclear equations of state. A number of
universal laws were found, which either relate the treshold mass of a
particular equation of state to its maximum stable static equilibrium (TOV)
solution or its compactness. We also could constraint the maximum radius and 
mass of neutron stars in general which go hand in hand with observational 
constraints from gravitational wave observations.

These results can be improved in at least two ways.  First, as new hot
equations of state becomes available for numerical simulations, it will be
possible to extend the analysis carried here, reducing  its uncertainty.
Second, as new detections from binary neutron-star mergers will be revealed,
the masses of these systems and their electromagnetic counterparts will be
used to set ever more precise lower bounds on the radii of neutron stars.

As a final project, we have studied higher-dimensinal black hole solutions and
their thermodynamics from GUP-inspired approachs. These include the
Kempf-Mann-Mangano (KMM) momentum measure, a new modified momentum measure,
and also the Scardigli-Casadio-Carr-Lake (SCCL) approach. The fact that the
standard Hawking temperature diverges for small mass black holes is one
of the indicators that a theory of quantum gravity is needed. The GUP is
such a candidate and has been shown to modify this behaviour in the
standard $(3+1)$-dimensional case (SCRAM phase).  Since the GUP can
for example be motivated by string theory which needs extra dimensions
this strongly suggests that the cure of the temperature divergence
should also work in higher dimensions.

In the former case, we have shown that for $n=4$ spatial dimensions, the
metric function admits an extremal solution whose temperature exhibits
the SCRAM phase of the regular $(3+1)$-D KMM black hole.  Additionally,
the solution reveals a conical singularity at the origin, which can be
interpreted as a Barriola-Vileking like monopole.  For $n > 4$, howver,
the GUP has no effect and the temperatures diverge as with the
standard Schwarzschild solution.

We have also investigated a revised GUP approach that produces an alternate
momentum measure that uniformly suppresses the GUP in the limit of large
$\vec{p}$.  In this case, the solution reveals a much richer thermodynamic
behaviour that is characterized by a ``lighthouse effect''.  That is, for 
increasing $n > 4$, both the mass distribution and temperature oscilate
as the black hole evaporates. This is possibly due to negative energy
density contributions close to the origin, or alternatively due to the
presence of tachyon states. 

The metric function also admits multiple horion structures depending
on the black hole mass, relative to a critical value $M_0$.  This
particular characteristic has the effect of dividing the solutions
into two classes, {\it i.e.} small and large black holes.

A variety of future investigations are possible, in particular those
which involve different interpretations of the GUP (\eg a higher
dimensional version of the metric derived in \cite{CMN15}).  Alternatively,
GUP-inspired Reissner-Nordstr\"om and Kerr metrics might introduce
additional new physics similar to the lighthouse effect discussed herein.

\end{fullwidth}