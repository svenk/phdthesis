% Titlepage and TOC
%
% Supposed to be included from phdthesis.tex

\thispagestyle{empty}
\begin{fullwidth} % for tufte
	\begin{center}
		%%    \includegraphics[draft, width=9.5cm]{someTitleFigure}
		%\begin{pgfpicture}
		%	\pgftext{\pgfimage[width=9.5cm,height=4cm]{someTitleFigurePlaceholder}}
		%\end{pgfpicture}
		%%\\
		% dezente Schattierung
		\phantom{empty}
		\vspace{2cm}
		\Huge \textcolor{black!90}{\textbf{\thetitle}} \\
		%\vspace{.5cm}
		%\LARGE \textcolor{black!80}{\textbf{\thesubtitle}} \\
		\Large
		\vspace{1.9cm}
		%\textcolor{black!70}{
			Dissertation zur Erlangung des \\
			Doktorgrades der Naturwissenschaften \\
			\vspace{1.9cm}
			vorgelegt beim Fachbereich Physik der \\
			Johann Wolfgang Goethe-Universität in Frankfurt am Main \\
		%}%endofcolor
		\vspace{1.9cm}
		von\\
		\textsc{\theauthor} \\
		aus \\
		\textsc{Königstein im Taunus} \\
		\vspace{.9cm} 
		Frankfurt am Main, März 2019 \\
		(D30)
		\vspace{1.2cm}
		% for the list of figures :D
		% \addcontentsline{lof}{figure}{Describing the front page picture}%
		
		\ifdefined\releasemode\else
		\colorbox{red!40}{This PDF was compiled at {\bf \thedate}}\\[5ex]
		\fi
		\vfill{}
		%\large
		%\begin{tabular}{@{}>{\itshape}rl@{}}
		%	First Supervisor: & \textsc{Prof. Dr. Luciano Rezzolla} \\[1ex]
		%	Second Supervisor: & \textsc{Prof. Dr. Piero Nicolini}
		%\end{tabular}
		%\\
		%\vspace{2cm}
		\includegraphics[width=4.9cm]{logos/goethe-logo.pdf}
		%\\
		%\vspace{1cm}
		%\large
	\end{center}
	
\newpage % end of title page
\end{fullwidth}
	\Large
	
	\noindent
	Geschrieben am \\[.3cm]
	
	\noindent
	\textsc{Institut für theoretische Physik (ITP)} \\
	Max-von-Laue-Straße 1 \\
	60431 Frankfurt am Main \\[.3cm]
	
	\noindent
	und am \\[.3cm]
	
	\noindent
	\textsc{Frankfurt Institute for Advanced Studies (FIAS)} \\
	Ruth-Moufang-Straße 1 \\
	60431 Frankfurt am Main
	
	\vfill
	
	\noindent{}%
	vom Fachbereich Physik der \\[2pt]
	Goethe-Universit\"at als Dissertation angenommen.
	
	\vspace*{2cm}
	
	\noindent{}%
	\begin{tabular}{@{}l@{\hskip 1ex}l}
		Dekan: & Prof. Dr. Michael Lang\\
		\phantom{Gutachter:}&~
	\end{tabular}
	
%	\vspace*{1cm}
	
	\noindent{}%
	\begin{tabular}{@{}l@{\hskip 1ex}l}
		Gutachter: & Prof. Dr. Luciano Rezzolla\\
		& Prof. Dr. Piero Nicolini \\
		& Prof. Dr. David Radice (Princeton University/IAS)
	\end{tabular}
	
	\vspace*{2cm}
	\noindent{}%
	Datum der Disputation:
	\vspace*{2cm}
	\normalsize  % end of "Large" 
	
\newpage

\begin{fullwidth}

\begin{center}
\parbox{\textwidth}{
%
\begin{center}\textbf{Abstract}\end{center}

This thesis is a summary of existing and upcoming publications
\cite{Dumbser2017,Fambri2018,Koeppel2017,
	exahype-guidebook,Koeppel2019,FKN16,Koppel:2017rsf,Knipfer2019},
with a focus on high order methods in numerical relativity and general
relativistic flows.
The text is structed in five chapters. In the first three ones, the
ADER-DG technique and its application to the Einstein-Euler
equations is introduced. Novel formulations for both the Einstein
equations in the 3+1 split as well as the general relativistic
magnetohydrodynamics (GRMHD) had to be derived.
The first order conformal and covariant Z4 formulation of Einstein equations
(FO-CCZ4) is proposed and proven to be strongly hyperbolic.
Together with the fluid equations of general
relativistic magnetohydodynamics (GRMHD), a number of benchmark scenarios is
presented to show both the correctness of the PDEs as well as the
applicability of the numerical scheme.

As an application in astrophysics, a general-relativistic study of the
treshold mass for a prompt-collapse of a binary neutron star merger with
realistic nuclear equation of states has been carried out. A nonlinear
universal relation between the treshold mass and the maximum compactness
is found. Furthermore, by taking recent measurements of GW170817 into account,
lower limits on the stellar radii for any mass can be given.

Furthermore, an (unpaired) work in quantum mechanical black hole engineering
is presented. Higher dimensional extensions of generalized Heisenberg's
uncertainty principle (GUP) are studied. A number of new phenomenology is
found, such as the existence of a conical singularity which mimics the
effect of a gravitational monopole on short scale and that of a
Schwarzschild black hole at a large scale, as well as oscillating Hawking
temperatures which we call ``lighthouse effect''. All results are consistent
with the self complete paradigm and a cold evaporation endpoint remnant.
}% end of parbox
\end{center}

	
\phantom{foo}\vfill

\textcolor{black!80!white}{
This document is written with \LaTeX{} and a style inspired by the books
of the data visualization pioneer \textsc{Edward Tufte}. In physics, his
style was first adopted in the textbooks of \textsc{Richard Feynman}.
It is characterized by the large margin column and the flat structure,
among others. I chose this format to include a lot of illustrating figures
and  supplementary comments. Deeply nested hierarchies are omitted.
%\todo[inline]{sowie: Zitationsstil LRR, falls gewählt.}
} % end of textcolor

\vspace{.5cm}
\noindent
\textcolor{black!80!white}{
The text and figures (see page \pageref{sec:lists} for a list) in this 
work are
licensed under a Creative Commons Attribution-ShareAlike license 
(\href{https://creativecommons.org/licenses/by-sa/4.0/}{CC BY-SA}).
} % end of textcolor
%\newpage
\end{fullwidth}
% probably not here
%\chapter*{Allgemeinverständliche Zusammenfassung}

\todo{Idee: Diese Seite mit Inkscape machen und netter bebildern.}
Die vorliegende Doktorarbeit besteht aus fünf Kapiteln. Im ersten
Kapitel werden Methoden vorgestellt, um eine bestimmte Art von
Gleichungen, die man ``hyperbolische partielle
Differentialgleichungen'' nennt, zu lösen. Es ist eine schwierige
Aufgabe, dafür Supercomputer zu verwenden, die aus tausenden von
Einzelcomputern bestehen.

Im zweiten Kapitel geht es um Einsteins allgemeine Relativitätstheorie.
Mit ihrer Hilfe können extreme Phänomene im Universum beschrieben 
werden, etwa der Kollaps eines Sternes zu einem schwarzen Loches. Um
diese Vorgänge zu berechnen, werden Näherungen benötigt. Hierzu
eignen sich Computersimulationen, die Filme dieser Vorgänge erstellen.
Da Einsteins Theorie aber Zeit und Raum zusammenfasst, werden diese
in dem Kapitel erst mühsam wieder auseinandergenommen. Auf diese Weise
entstehen eine Vielzahl obengenannter Gleichungen. In dieser Arbeit
wurde eine Formulierung der Einsteingleichungen hergeleitet, die fast
60 Differentialgleichungen umfasst. Es wurde dann demonstriert, wie gut
diese Gleichungen mit Computern gelöst werden können.

Im dritten Kapitel geht es um Flüssigkeiten und Plasmen. Sie eignen
sich, um die Materie in Sterne zu beschreiben. Es werden wieder
Differentialgleichungen hergeleitet, die beschreiben, wie sich ein
stark magnetisches Plasma bewegt. Die Computerlösung dieser Gleichungen
wird anhand zahlreicher Beispiele demonstriert.

Im vierten Kapitel geht es darum, wie man die vorangenannten Techniken
benutzen kann, um Aussagen über eine Art von Doppelsternsystemen zu
machen. Hierbei soll abgeschätzt werden, wie lange es dauert, bis zwei
kollidierende Sterne zu einem einzigen schwarzen Loch werden. Dies
hängt stark von den Eigenschaften von Atomen und ihren Bestandteilen,
den Neutronen und Protonen, ab. Vor drei Jahren wurden solche 
Kollisionen im Weltall mit einem riesigen Messapperat das erste mal
nachgewiesen. Dabei wurden Einsteins Gravitationswellen gemessen (2017
wurde dafür der Nobelpreis verliehen). Dank dieser Messungen sowie den
Computersimulationen kann die Beschaffenheit von Neutronensternen
besser abgeschätzt werden und damit zuletzt die Beschaffenheit von
Elementarteilchen wie sie auch auf der Erde existieren.


\ifdefined\releasemode\else
\begin{fullwidth} \listoftodos \end{fullwidth}
\fi

\begin{fullwidth}
% \chapter*{Contents} % eToc brings its own Contents header
\tableofcontents
\end{fullwidth}

%\newpage% just to be sure
\normalsize% just to be sure
\normalfont% just to be sure