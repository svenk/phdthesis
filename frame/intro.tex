\vspace{.5cm}
\begin{fullwidth}
%
Shortly after the beginning of my PhD position in theoretical
astrophysics end of 2015, the first
direct observation of gravitational waves was made, where the merger of
$36\Msol$ and $29\Msol$ black holes to a single $62\Msol$ black hole was
measured. This remarkable event identified as GW150914 \cite{Abbott2016aa}
prove the existence of
gravitational waves, of stellar mass binary black hole systems and heralded a
new era of gravitational wave multimessenger astronomy. The breakthrough gained
international attention due to the 2017 Nobel prize awarding to Weiss,
Barish and Thorne ``for decisive contributions to the LIGO
detector and the observation of gravitational waves''.

In 2017 the first direct observation of two merging neutron
stars was made, with constituent neutron star masses up to $1.6\Msol$ and
a merger mass of $2.74\Msol$. This event, identified as GW170817 
\cite{Abbott2017}, was followed by a short gamma-ray burst and a number of
other observations in the electromagnetic spectrum.

Due to their bad signal to noise ratio, gravitational wave interferometry
depends on precomputed gravitational wave-forms. These wave-forms are computed
by large scale perturbative but especially numerical approaches to general
relativity. Numerical relativity had emerged as a powerful tool
for the study of astrophysical systems, following the breakthrough
calculations of the merger of binary ``moving puncture''
black holes in mid-2000
\cite{Pretorius:2005gq,Campanelli06,
	Baker05a,Baker-etal:2007a,Bruegmann:2006at,Hannam:2008sg}.

\subsection{Numerical schemes for the Exascale era}
% Scheme intro from the GRMHD paper, adopted for generality.

The interest for high accuracy simulations of general-relativistic
spacetimes has only been strengthened by the observational
breakthrought. There is a need for stable and accurate methods
which can exploit the computational ressources available.
Chapter~\vref{chapter:numerics} is a mathematical one which purely
concentrates on numerical aspects in hyperbolic partial differential
equations and introduces a sophisticated numerical scheme with
arbitrary convergence order in time and space. At the same time,
it is communication avoiding and therefore suitable for the upcoming
generation of exascale computers --- machines which can compute
$10^{18}$ basic arithmetic operations per second. The scheme was
used in the coauthored publications \cite{Dumbser2017,Fambri2018} as
well as in \cite{exahype-review,exahype-guidebook}. Particular
challenges in computer science/high performance computing (HPC) are
discussed and the \code{ExaHyPE} code is introduced, a dynamical
adaptive mesh refinement (AMR) code
which uses finite state machines for determining the control flow
and take off the control from the user. \code{ExaHyPE} can solve a
specific class of hyperbolic partial differential equations (PDEs)
and will be open sourced end of 2019.

Discontinuous Galerkin methods belong to the family of finite-element
methods which consider the numerical approximation of a weak formulation
of the governing system of partial differential equations over a set of
non-overlapping elements. The discrete solution space is restricted to
the space of piecewise polynomials of maximum degree $N \geq 0$ and the
degrees of freedom (\ie the expansion coefficients) of the chosen
polynomial basis are directly evolved in time.
In particular, in the DG formulation the
numerical solution is allowed to be {discontinuous} at element interfaces.
%
In the last twenty years, DG methods became increasingly popular mainly
because of four attractive properties:
\begin{enumerate}[(i)]
  \item nonlinear $L_2$ stability (See 
    Appendix \ref{apx:symbols} for standard definitions such as ``$L_2$'')
    has been proven for general nonlinear scalar conservation laws
    \cite{Jiang1994};
  \item arbitrary high order of accuracy can be easily
    achieved for smooth solutions by simply increasing the polynomial degree
    $N$ of the chosen basis functions;
  \item high parallel scalability makes DG methods better suited for
    large-scale simulations even on general
    unstructured meshes when compared with high-order finite-difference or
    finite-volume methods;
  \item high-order DG methods are not very dissipative and dispersive, even
    when compared with high-order finite-volume and finite-difference schemes
    and are thus essential for accurate long-term simulations. 
\end{enumerate}
    
%The main drawback that afflicts
%{explicit} DG schemes is the rather severe CFL stability condition that
%constrains the timestep of the simulations to scale with approximately
%$h/(2N+1)$ for hyperbolic partial differential equations, where 
%$N$ is the degree  of the nodal polynomial basis used within the element 
%and $h$ is the characteristic size of one DG element (not the distance 
%between the individual nodal degrees of freedom). 
%A way to alleviate the severe CFL timestep restriction is the use
%of efficient semi-implicit DG schemes, as those proposed, for instance by
%\cite{Tavelli2016,Fambri2016}.

\subsection{A novel hyperbolic formulation of Einsteins Equations}
\label{sec:gr-intro-intro}
Chapter \vref{chapter:gr} concentrates on general relativity and casts
these equations in the mathematical framework proposed in the the
previous chapter.
In fact,
the development of hyperbolic formulations of the
Einstein equations that allow for long-term simulations of generic
spacetimes, including the ones encompassing the physical singularities
arising in the presence of black holes, has been of great
importance in numerical relativity. The first step in this direction has
been the derivation of the Arnowitt-Deser-Misner (ADM) formulation.
While this formulation splits time and
space and presents general relativity as an initial
boundary-value problem, suitable for numerical implementation, it is
known to be not hyperbolic, and therefore unstable in numerical
applications.

Subsequently, a lot of effort has been devoted to find hyperbolic
formulations of the Einstein equations. These efforts have lead to the
derivation of the Baumgarte-Shapiro-Shibata-Nakamura-Oohara-Kojima
(BSSNOK) formulation \cite{Shibata95,Baumgarte99,Nakamura87,Brown09},
which achieves hyperbolicity via a conformal transformation of the
3-metric and the promotion of some contractions of the Christoffel
symbols to independently evolved variables and, most importantly, by
inserting the momentum and Hamiltonian constraint expressions in the
evolution system. A general-covariant alternative is the Z4 formulation
of \cite{Bona:2003fj,Bona:2003qn,Alic:2009}, which has been presented
both in first- and second-order form in the spatial derivatives. More
successful have been formulations based on the Z4 one that include a
conformal transformation of the metric. These are the Z4c formulation,
that removes some source terms in the Einstein equations in order to
bring the evolution equations into a form which is closer to the BSSNOK
system \cite{Bernuzzi:2009ex}, and the CCZ4 formulation \cite{Alic:2011a,
	Alic2013}, which also includes a mechanism to damp constraint
violations as they arise during the evolution (see also
\cite{Sanchis2014, Bezares2017} for some recent and slight variants).

Parallel to the quest for better formulations of the equations, the
development and implementation of better numerical methods has been a
main priority of ongoing research. While most general-relativistic codes
use finite-differences (\eg \cite{Brown2007b,Zhang2006, mignone_2010_hoc,
  Radice2012a,Radice2013b}) or spectral methods (\eg
\cite{Szilagyi:2009qz}) for the spacetime evolution, increasing interests
is being focused towards DG methods, which are very attractive due to
their excellent scalability and wave-propagation properties.
The latter allow
the propagation of smooth linear and nonlinear waves over long distances
with little dissipation and dispersion errors, and turn out to be 
well suited for the solution of the Einstein equations, where (apart
from physical singularities in black holes) the fields are smooth and
high accuracy can be achieved.

So far, however, only a rather limited number of attempts have been made
to solve Einstein equations with DG methods. Field et al. \cite{field10}
tested a second-order BSSNOK formulation, while Brown et
al. \cite{Brown2012} developed a first-order formulation of BSSNOK,
however both works were limited to spherical symmetry and vacuum
spacetimes. The first DG implementation in non-vacuum spacetimes was
published by Radice \& Rezzolla \cite{Radice2011}, but was still
restricted to spherical symmetry. The first three-dimensional (3D)
implementation, albeit in a fixed spacetime and focused on hydrodynamics
was developed by Bugner et al. \cite{Bugner2015,Bugner2018}.
%More recently,
Miller and Schnetter \cite{Miller2016} proposed an operator-based DG 
method suitable also for second-order systems and applied it to the 
BSSNOK
system, while Kidder et al. \cite{Kidder2016} developed a task based
relativistic magnetohydrodynamics code.

Within theis thesis, a novel first-order (FO) form of the CCZ4 system is
presented, refered to as FO-CCZ4. The system's eigenstructure is studied
and strong hyperbolicity is shown for a particular choice of gauges.
Subsequently, different numerical implementations of this PDE system
are discussed. The scheme is solved with the three dimensional code
proposed in the previous chapter, using an ADER-DG algorithm with
adaptive mesh refinement (AMR) and local time-stepping (LTS),
supplemented with a high order ADER-WENO \cite{eno,Liu1994, Jiang1996}
finite-volume subcell limiter
\cite{Dumbser2009a,Dumbser2010,Dumbser2011} to deal with
singularities in black-hole spacetimes.

A series of standard tests for general-relativistic codes
\cite{Alcubierre:2003pc,Babiuc:2007vr} are adopted to demonstrate 
the stability and accuracy of both the PDE itself and its ADER-DG 
discretization.
Verifications for the expected convergence order are given and long-time
robustness and stability is proven. Finally, the scheme is applied at a
long-term evolution of single black-hole spacetimes, showing that the 
new code is able to stably evolve a puncture black-hole spacetime for a
time scale of $\sim 1000$ $M$ ($M$ being the mass of the black hole).
Preliminary results for the head-on collisions of two black holes are
presented.

While the equations are given in a general way including matter terms,
they are vanishing in the presented tests and Einstein field equations
are only presented ``in vacuum''. Scientific results of this chapter
have been published in \cite{Dumbser2017}, as 
well as in \cite{Koeppel2017,exahype-review}.

\subsection{General relativistic Magnetohydrodynamics within the
  new framework}\label{sec:hd-intro}

Chapter \vref{chapter:hydro} discusses relativistic hydrodynamics. It is 
the complementary topic compared to the previous chapter, concentrating
solely on the right hand side of Einsteins equations.

In fact, electromagnetism plays an important role in many astrophysical 
processes
such as compact objects and binaries consisting of black holes and
neutron stars. The general-relativistic theory of magnetohydrodynamics
(GRMHD) is a successful theory to describe these systems, combining the
fluid description of matter with a simplified theory for electromagnetic
fields in the absence of free charge carriers. Similar to
general-relativistic hydrodynamics (GRHD), first successful
(lower-dimensional) simulations of the GRMHD system date back to the
pioneering work of \cite{Wilson1975} more than 40 years ago
(See \cite{Font08,Marti2015} for recent reviews in the progress of GRMHD
simulations).

In the past years, several groups
started to recast the system of GRMHD equations into a conservative form
in order
to make use of conservative Godunov-type finite-volume schemes based on
approximate Riemann solvers and high-resolution shock-capturing schemes
(HRSC). Many GRHD and GRMHD codes have been developed over the last
decade (for instance \cite{Baiotti04, Duez05MHD0, Anninos05c, Anton06,
	Giacomazzo:2007ti, Anderson2008, Kiuchi2009, Bucciantini2011,
	Radice2012a, Dionysopoulou:2012pp, Radice2013b, Osorio2015, 
	White2016, 
	Porth2017})
and applied to various topics in astrophysics. Some codes also evolve the
spacetime by feeding back the fluid and magnetic energy-momentum tensor
in the Einstein field equations, which govern the time evolution of the
metric tensor; some codes even incorporate radiation transfer like the
one proposed by \cite{Takahashi2017}, or include the full Maxwell theory
in a resistive relativistic MHD formulation
\cite{Palenzuela:2008sf, Dumbser2009,
	Dionysopoulou:2012pp, Bucciantini2012a, Bugli2014, Aloy2016}.

DG methods have attracted the interest of the computational-astro\-physics
community only over the last few years. In particular, the first
DG-based method for general-relativistic hydrodynamics has been developed
by \cite{Radice2011}, but it was limited to spherically symmetric
spacetimes. The first three dimensional implementation of a DG method for
relativistic flows on curved but fixed background spacetimes has been
recently presented by \cite{Bugner2015}, but without considering the
magnetic field interaction. Recently, \cite{Kidder2016}
provided a DG implementation within a task-based parallelism model for
GRMHD, while \cite{Anninos2017} presented also a DG code with
hp-refinement.

In this work, the previously introduced new numerical scheme is proposed
for the solution of the GRMHD equations. Its
shock capturing properties and high-order accuracy on spacetime
adaptive meshes (AMR), supplemented by a high-order a posteriori
subcell 2nd order TVD finite-volume limiter (adopted for shocks
and discontinuities) make it well suited for the evolution of the MHD
flow on curved background spacetimes. In fact, this method
was already successfully applied on special-relativistic MHD
equations (SRMHD) in \cite{Zanotti2015d}.

An important and novel aspect of this approach is the interpretation of
the source terms in the GRMHD equations that account for the
gravitational field in curved spacetimes as seperate nonconservative
products. In other words, while the GRMHD equations are normally written
in a flux-conservative form with a generic source term which holds
derivatives stemming from the spacetime curvature, in the presented
framework, it is written without an algebraic source but with a
differential nonconservative term which is part of the system
eigenstructure. This is a simple rewrite of the equations which neither
changes its hyperbolic nature  \cite{Anile_book,Komissarov1999} nor the
fact that it is already first order, suitable for a large number of
finite volume methods.

After the presentation of the modified PDE system, the solution of
number of smooth and non-smooth (\ie shocks and large gradients
including) benchmark situations are provided which demonstrate the
correctness of the PDE and the quality of its numerical solution.

The results of this chapter have been published in \cite{Fambri2018}
and also in \cite{Koeppel2018,exahype-review,exahype-guidebook}.

\subsection{A general-relativistic criterion to seperate prompt from delayed
	collapse in a binary neutron star merger}\label{sec:bnslt-intro}
Chapter \vref{chapter:bnslt}  is the first chapter discussing a purely
astrophysically driven research topic where all the techniques
developed in the previous chapters serve as a tool for modeling
four dimensional spacetimes of astrophysical phenomena. The chapter
presents research done in respect to the issue of lifetimes of binary
neutron star collision remnants. Thanks to the first neutron star
binary merger observations \cite{Abbott2017} which offers new insight
into the nature of neutron stars, constraints in the properties of
nuclear matter can be probed \cite{Annala2017, Paschalidis2017,
 Bauswein2017b,Radice2017b,Most2018, Burgio2018, Montana2018}.

When two neutron stars merge, they will produce an object that either
collapses promptly to a black hole, or does not \cite{Baiotti08}. In the
latter case, the remnant may be a metastable object, \eg a hypermassive
neutron star (HMNS), eventually collapsing to a black hole on a secular
timescale, or survive for much longer times, either as a rotating or a
nonrotating star (see \eg \cite{Baiotti2016} for a review). In the case
of the first detection of merging neutron stars, GW170817
\cite{Abbott2017}, the precise fate of the merger remnant is
presently unknown, although the formation of a black hole naturally
matches the simultaneous observation of a short gamma-ray burst
\cite{Eichler89, Rezzolla:2011}, and has been the working hypothesis to
set new limits on the maximum mass of neutron stars \cite{Margalit2017,
	Shibata2017c, Rezzolla2017, Ruiz2017}.

Determining the time of collapse of the merger remnant is particularly
challenging as there are a number of physical processes that either
determine or undermine the stability of merger remnant. These include:
the ejection of matter \cite{Rosswog1999, Kyutoku2012, Radice2016,
	Lehner2016, Dietrich2016, Bovard2017}, the angular-momentum transfer
via magnetic fields \cite{Kiuchi2015a, Siegel2013, Kawamura2016}, the
evolution of the degree of differential rotation \cite{Kastaun2016,
	Hanauske2016}, and possible viscous effects mediated either by
neutrinos or magnetic fields \cite{Duez2004b, Shibata:2017b, Radice2017,
	Alford2017}.

As already said, observationally, the two scenarios can be clearly kept apart
by both the gravitational wave and electromagic spectra which are much
richer in the case of the delayed collapse. A first attempt to seperate
prompt to delayed collapse can be made by the mass of the system $M$.
It is simple to seperate the two scenarios by a critical 
(threshold) mass $\Mth$ which distinguishes prompt collapses ($M>\Mth$) from
delayed ones ($M<\Mth$), althought it still poses numerical and conceptual
challenges. Bauswein et. al. \cite{Bauswein2013} have been the 
first to explore this problem by employing a smooth-particle approximation for
the hydrodynamics and a conformally flat approximation to general
relativity. In this way, they were able to find a linear universal
relationship between $M_{\rm th}$ and the compactness of the maximum-mass
model, $\mathcal{C}_\tov:=M_\tov/R_\tov$, where
$M_\tov$ and $R_\tov$ are respectively the mass and
radius of the maximum-mass nonrotating star. Here, we improve on this
result by using a fully general-relativistic approach, a wider range of
compactnesses, and a rigorous definition of the threshold mass. As a
result, we find a {nonlinear} relation between $M_{\rm th}$ and
$\mathcal{C}_\tov$, which offers a better match to the
numerical-relativity results. Furthermore, exploiting the information
from GW170817, we use the new relation to set more stringent lower bounds
on the radii neutron stars \cite{Bauswein2013,Bozzola2017,Bauswein2017b}.
%
The results presented in this chapter have been published in
\cite{Koeppel2019}.

\subsection{Quantum modified Schwarzschild solutions}\label{sec:qbh-intro}
Chapter \vref{chapter:qgr} discussed black hole spacetimes at the smallest scales. 
The Schwarzschild solution is only characterized by its mass $M$ which can have
any value. When it comes to quantum size black holes, a natural
mass scale is given by the Planck scale,
$
M_\text{Pl} = \sqrt{{\hbar c}/{G}} \sim 2.1 \times 10^{-8}\, \text{kg} \sim 5.6 \times 10^{27}
\,\text{eV}/c^2
$,
which is tiny compared to astrophysical scales but nevertheless huge when compared
to the masses of the standard model particles. The theory of quantum mechanical black holes
is called quantum gravity (QG), and its theoretical description is a longstanding
problem, experimentally it also remains inaccessible. Quantum field theory on curved
space (QFTCS) is an approximating theory to describe quantum matter on classical curved
background space, and Hawkings famous result that quantum black holes are actually
black bodies that emit thermal readiation at a temperature proportional to their
surface gravity~\cite{Haw75} belongs to this class of first order quantum gravity effects.
As soon as the Hawking temperature comes in the regime of the Planckian black hole mass
itself, this thermodynamical description breaks down.

Black holes also question the understanding of quantum mechanics itself.
Conventionally the Compton wavelength is thought to assume arbitrarily small values,
provided one smashes particles at higher enough energies.
This way of reasoning, however, breaks down at the Planck scale.
A Planckian black hole is expected to form due to the collapse of 
particles at such extreme energies~\cite{Adl10}. This is equivalent to saying 
that gravity is ultraviolet self-complete, \textit{i.e.}, there are no 
propagating quantum degrees of freedom in the trans-Planckian regime and length 
scales below the Planck length are inaccessible 
\cite{DvG10,DFG11,DGG11,SpA11,DFG12,DvG12,NiS12,MuN12,AuS13,Car14,DGI15,CMN15,FKN16}.
Such features are effectively captured by a modification of commutation relations known
as generalized uncertainty principle (GUP)~\cite{Ven86,ACV89,ACV93,Mag93,KMM95},
resulting in a modification of Heisenberg's uncertainty principle
$\Delta x \Delta p \leq \nicefrac \hbar2 \left( 1 + \beta (\Delta p)^2 \right)$,
where $\sqrt \beta \sim \ell_\text{Pl}$ and $\ell_\text{Pl}$ the Planck length.
For $\Delta p \ll \ell_\text{Pl}$, length scales become proportional to $\Delta p$,
as expected from the presence of a black hole in the trans-Planckian regime.
For reviews see \cite{SNB12,Hos13,TaM15}. 

The GUP has been invoked to improve the scenario of black hole evaporation, 
that is customarily affected by a divergent profile of the Hawking temperature 
$T$ in the terminal phase. With $\Delta p\sim T$ 
and $\Delta x\sim GM$, the temperature profile is no longer divergent and  a 
Planckian black hole remnant forms as an evaporation endpoint 
\cite{AdS99,APS01}. Such a remnant has also been considered as a candidate for 
cold dark matter component \cite{ChA03}. There are, however, potential problems 
at the basis of such results. Planckian remnants have Planckian temperatures. 
The surface gravity description of the temperature no longer holds.

To amend the above limitations, a new approach has been proposed in order to 
implement GUP effects in gravitational systems \cite{IMN13}. As a start, one 
can notice that the GUP introduces nonlocality by preventing infinitesimal 
resolution.  Therefore one might be led to consider a nonlocal version of 
Einstein equations \cite{Kra87,Tom97,Bar03,Mod12a}, where nonlocal
spacetime (\ie a nonlocal Einstein tensor, smeared by a 
operator-valued gravitational coupling constant)
is coupled to classical spacetime (\ie the Schwarzschild source term).
Such a theory 
can be either used to described large scale degravitating effects 
\cite{ADD02,DHK07,Bar05,Bar12} or short scale modified gravity theories 
\cite{GHS10,MMN11,Nic12,CMN14,FKN16}. 
One can select a specific smeared gravitational coupling constant 
$G_\mathrm{N}^{-1}\left(L^2\Box\right)$ to reproduce the GUP 
momentum space deformation
for the static potential due to virtual particle exchange.
The resulting non-rotating black hole metric allows for 
horizon extremisation with consequent formation of a zero temperature black 
hole remnant at the end of the evaporation \cite{IMN13}. Such a black hole 
solution not only supersedes the aforementioned limitations of the scenario 
proposed in \cite{AdS99,APS01}, but offers additional interesting properties: 
it removes the scale ambiguity of the Schwarzschild metric and fulfills the 
gravity ultraviolet self completeness by preventing black hole radii smaller 
than the Planck length; it allows for a semiclassical description of the whole 
evaporation process and for the presence of a final heating phase (SCRAM 
phase \cite{Nic09}) before the remnant formation.

Within this thesis, the case of GUP effects in higher dimensional black hole
metrics is studied. It should be noted that there is no
unique prescription for the GUP in the presence of extra
dimensions \cite{Koppel:2017rsf}. As a result, an analysis of the existing
proposals
\cite{ScC03,Maz13,Car13,Car14,LaC15,LaC16,Carr:2017grh,Maz12,DMS15,Maz15,LaC18}
for the GUP in higher dimensions will be given.

The results presented in this chapter have been published in
\cite{Koppel:2017rsf}, while other are going to be published in an upcoming
publication \cite{Knipfer2019}.

%\subsection*{Motivations and general structure}
%In this monograph, each chapter subsumes one research topic which typically
%lead in one or more publications. Each of these topics roughly corresponds
%to a well-established community which brought up plenty of primers, text-books
%and comprehensive reviews.
%%The lengthy bibliopgrahy (page \pagref{chapter:bib})
%%can hardly cover the most relevant work. Also, introductions such
%%as the foregoing pages can only provide a short overview...
%
%I decided to prepend each chapter with a motivational (\ie fresmen-level)
%opener as a sweep in high energy physics to create links and connections
%instead of providing an outright review.
%
%
%As a motiviation for the sophisticated numerical methods to solve hyperbolic
%PDEs in Chapter~\ref{chapter:numerics}, Section~\ref{sec:hamiltonian-motivation}
%reviews Hamiltonian dynamic and the concept of time evolution.
%
%The motivation for the numerical relativity Chapter~\ref{chapter:gr} is given
%in Section~\ref{sec:gr-intro}. It introduces the two body problem in general
%relativity and motivates to approximate in the strong field regime not on the
%field equations, but instead on the spacetime itself.
%
%The relativistic hydrodynamics Chapter~\ref{chapter:hydro} is prepended with
%Section~\ref{sec:hd-nuclear-matter} which briefly introduces the need or an
%effective theory when describing the spacetime of neutron stars.
%
%The binary neutron star lifetimes Chapter~\ref{chapter:bnslt} is equipped with
%Section~\ref{sec:bnslt-motivation} which makes an excursion to nuclear physics
%and its equations of states.
%
%The quantum black hole Chapter~\ref{chapter:qgr} is not equipped with such
%an excursion since the topic itself is probably the excursion from the
%other chapters. Therefore, Section~\ref{sec:motivation-qbh} just provides
%a motiviation why to study quantum black holes.
%
%Finally, Chapter~\ref{chapter:conclusions} provides a conclusive final
%discussion about the work done within this thesis and where to go next.
\end{fullwidth}