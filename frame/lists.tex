\chapter{References}
\setcounter{section}{0} % each appendix has his own sections
%
%\chapter{List of figures and tables}
%\addcontentsline{toc}{chapter}{List of figures and tables}

This list helps to find and identify figures; they are labeled by their size 
and type (illustration or plots) and marked if they were not self-made.

\begin{fullwidth} % for tufte
%\listoffigures
% hide title/own title
\section{List of figures}\label{sec:lists}
{\makeatletter
	% ENDLICH, das klappt:
	\renewcommand*{\addvspace}[1]{}
	\@starttoc{lof}
	\makeatother }

\newpage % design improvement
% \listoftables
%\tinysection{List of tables}
\section{List of tables}

{\makeatletter
	\renewcommand*{\addvspace}[1]{}
	\@starttoc{lot}
	\makeatother}

\section{List of computer codes and libraries}\label{apx.codes}
This section lists codes which where used within this work, and adresses of internet websites where
it is possible to download them.

\subsection{Initial data codes}
The following codes are a selection of codes which I modified in order to
be read in by the \code{ExaHyPE} code.

\begin{description}
	\item[Lorene] An extensive code and library for creating
      general-relativistic initial data for various compact objects, such as
      single rotating (magnetized) stars and binary neutron stars. Whenever
      binary neutron star systems were evolved in this work, the initial data
      stems from Lorene. \\ The code is available at
      \href{https://lorene.obspm.fr/}{lorene.obspm.fr} or
      \href{https://bitbucket.org/relastro/lorene}{bitbucket.org/relastro/lorene}
      as well as
      \href{https://bitbucket.org/relastro/lorene-pointwiseexport}{bitbucket.org/relastro/lorene-pointwiseexport}.
     
	\item[RNSID] A fast code for creating rotating neutron star initial data.
	  This code is also part of the Einstein Toolkit,
	  available at
	  \href{https://bitbucket.org/einsteintoolkit/einsteininitialdata}{bitbucket.org/einsteintoolkit/einsteininitialdata}
      but also at 
	  \href{https://bitbucket.org/relastro/rnsid_standalone}{bitbucket.org/relastro/rnsid\_standalone}
	
	\item[ToriID] A code for creating torus initial data (hydrodynamic 
	  quantities) on a given spacetime from a compact object. This code
	  is available at
	  \href{https://bitbucket.org/relastro/toriid-standalone}{bitbucket.org/relastro/toriid-standalone}.
	
	\item[TOV Solvers] Various codes for solving the TOV equations were used
	  in this work, for instance the \code{TOVSolver} from EinsteinToolkit
	  which is available as a standalone version at
	  \href{https://bitbucket.org/relastro/tovsolver_hybrid/}{bitbucket.org/relastro/tovsolver\_hybrid}.
	  Other codes used are the closed soure codes \code{PizzaTOV} and
	  \code{MargheritaTOV}.
	
	\item[TwoPunctures] The TwoPunctures code allows to create a spacetime
	  with an arbitrary number of puncture black holes. A version is available
	  at 
	  \href{https://bitbucket.org/relastro/twopunctures-standalone}{bitbucket.org/relastro/twopunctures-standalone}.
	
\end{description}

\subsection{Time evolution codes}\label{apx.codes.time-evolution}
\begin{description}
	\item[EinsteinToolkit] The Einstein toolkit is a mature and extensive code
	  collection which is available as open source at
	  \href{https://einsteintoolkit.org/}{einsteintoolkit.org}. It is a modular
	  code based on \code{Cactus}. The \code{Carpet} code 
	    \cite{Schnetter-etal-03b, Schnetter:2006pg,Goodale02a} 
	  provides by default the computational grid/meshing. The \code{GRHydro}
	  and \code{IllinoisGRMHD} are two open source relativistic hydrodynamic
	  codes which are part of the Einstein toolkit. Howveer, there are also
	  a large number of proprietary extensions/modules, such as any recent
	  instance of the \code{Whisky} code.
	\item[WhiskyTHC] The Whisky templated hydrodynamics code
	  \cite{Radice2013,Radice2011}. It is the merger of various codes for
	  evolving hydrodynamics, such as the \code{Pizza} code. It gets it
	  name from templated C++ and is a high order FD/FV code for general
	  relativist hydrodynamics. It is supposed to be used within the
	  Einstein toolkit. It is closed source.
	\item[ExaHyPE] ExaHyPE is a next generation ADER-DG code with dynamic AMR.
      Its computational grid/meshing is provided by the \code{Peano}
      framework \cite{Peano1,Peano2}.
      It is available at \href{http://exahype.eu/}{exahype.eu} and/or
	  \href{http://www.peano-framework.org/}{peano-framework.org}.
	
	\item[SVEC] The State vector enhancement code is a general relativistic
	  magnetohydrodynamics code which can provide the PDE parts within the
	  ExaHyPE framework. The code is available at
	  \href{https://bitbucket.org/svek/svec/}{bitbucket.org/svek/svec/}
	  but also part of ExaHyPE.
	  
	\item[Antelope] Antelope is a code for solving Einsteins equations in a
	  MoL framework, as provided by the Einstein toolkit. It has different
	  formulations of Einsteins equations implemented
	  (Z4, Z4c, CCZ4, FO-CCZ4) and is based on the \code{TensorTemplates}
	  framework. At the time being, Antelope is closed source.
\end{description}
	
\section{List of co-authored papers}
In the following, all co-authored papers written within the relativistic
astrophysics group in Frankfurt are listed, with my individual contributions.
See also page \pageref{sec:cv} for my curriculum vitae which lists all
peer reviewed papers, \ie also single authored publications and publications
written not within the relativistic astrophysics group.

\begin{enumerate}
	\item[\cite{Dumbser2017}] FO-CCZ4 (Chapter \ref{chapter:gr}):
	   Parallel/comparative system matrix analysis with Mathematica and Maple,
	   both for a preliminary FO-Z4 candidate as the FO-CCZ4 system,
	   implementation and code generation for C++/\-Fortran/\-TensorTemplates,
	   running benchmarks and tests on different supercomputers,
       producing figures and texts for the paper,
	   revised and improved article during review.
	\item[\cite{Fambri2017}] ADER-GRMHD (Chapter \ref{chapter:hydro}):
	   Implementation and tests with Fortran/C++ version of PDE system,
	   producing texts for the paper,
	   revised and improved article during review.
	\item[\cite{Koeppel2019}] BNS Lifetimes (Chapter \ref{chapter:bnslt}):
		Operating over 200 (completed) binary neutron star merger simulations
		on different supercomputers,
		Interpreting the data,
		Writing an analysis code and contributing to a 
		comparative second one maintained by the coworker,
		Producing figures and text for the paper,
		Running a convergence study, testing different parameter spaces
		(like headon mergers),
		revised and improved article during review.
	%\item[\cite{Koeppel2017}] Astronum-Proceeding (single authored)
	%\item[\cite{Koppel:2017rsf}] GUP-Proceeding: 
	%   Produced figures and text, did calculations and derivations.
	%\item[\cite{Knipfer2019}] GUP-Paper:
	%   Produced figures and text, did calculations and derivations.
	%\item[\cite{FKN16}] Holography-Paper:
	%   Produced figures and text, did calculations and derivations.
\end{enumerate}

% This list is only for Luciano; thus no need to write about QBH papers.

\end{fullwidth}
% End list of figures and tables chapter