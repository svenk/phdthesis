% Macros for PhDthesis in tufte style,
% borrowing ideas from various other styles, most notably the classicthesis
% and cleanthesis styles.

% The macros also define the documentclass

% doesnt work anyway
\iffalse
% This has no effect at all:
% has to go before documentclass
% from cleanthesis.sty
\RequirePackage[					% modify figure and table captions
font={small}, 					% 	- small font size
labelfont={bf,sf,color=ctcolorfloatlabel},% 	- label in bold, sans-serif and 
%magenta
labelsep=endash,				% 	- separator: none, colon, 
%period, space, quad, newline, endash
singlelinecheck=false			% 	- no centered single-lined captions
]{caption}[2008/08/24]
% for compatiblity with tufte-book:
\let\abovecaptionskip\undefined
\let\belowcaptionskip\undefined
\fi



% We want backref=page in order to have back references from the
% bibliography. However, tufte-book wants to load hyperref without this
% option. Enabling \documentclass[nohyper]{tufte-book} on the other hand
% renders the figure captions in standard size and I found no way to
% revert it back in smaller size. Therefore we register the backref
% option *before* the documentclass loads and let tufte-book setup hyperref.
\PassOptionsToPackage{backref=page}{hyperref}
%
\PassOptionsToPackage{table}{xcolor}
%\documentclass[titlepage,nobib,nohyper,marginals=raggedleft]{tufte-book}
\documentclass[nobib,a4paper,oneside,nofonts,justified,notitlepage]{tufte-book}
% further options: justified, notitlepage
%                  twoside  vs oneside       (only for pagenumbers)
%                  symmetric                 (symmetric pages)
% option: nofonts <-> Palatino/Helvetica, turn on again at one time.

% TODO: Different layout for print (L+R) and screen (not switching L+R)

% Bugfix for accidentally doublesided cleared pages when in oneside mode.
% Helps avoid empty pages before new chapters and around the TOC/beginning
\ifthenelse{\boolean{@tufte@twoside}}%
{}{\renewcommand{\cleardoublepage}{\clearpage}}%


\usepackage[utf8]{inputenc}
%\usepackage[backref=page]{hyperref} % is done by tufte-book
\usepackage[german,english]{babel} % ngerman doesnt work; last language is main.
\usepackage{amsmath,amssymb,mathtools,braket}
\allowdisplaybreaks  % amsmath: allow page breaks in align enviromnents.

\usepackage{color,verbatim}
\usepackage{xcolor}
\usepackage{tikz}
\usepackage{environ} % \NewEnviron
\usepackage{nicefrac}
\usepackage{adjustbox} % scales content to box width
\usepackage{datetime}
\usepackage{titling} % \thetitle, \theauthor, ...
\usepackage{tabularx} % even better tables
\usepackage[toc]{appendix} % appendices
\usepackage{varioref} % \vref which includes page number
\usepackage{eurosym} % CV grants
\usepackage{bbding} % checkmarks
\usepackage{pifont} % dingbats
\usepackage{mathbbol} % use lowercase letters in mathbb

% cross mark and check mark https://tex.stackexchange.com/a/42620/49958
\newcommand{\cmark}{\ding{51}}%
\newcommand{\xmark}{\ding{55}}%

% just to be sure we have these
\usepackage{array,multirow,mathtools,adjustbox,etoolbox}

% Status: tried \subref for equations, does not work.
%    Package probably be removed again!
% -> at least it mangles the figure fontsize!
\usepackage{subcaption} % also providing \subref
 % fix subcaption https://tex.stackexchange.com/a/31927/49958
\captionsetup{compatibility=false,font=footnotesize}

% Colors:
% Classicthesis input
% from https://ctan.org/tex-archive/macros/latex/contrib/classicthesis/
\definecolor{CTsemi}{gray}{0.55} % chapter numbers will be semi transparent .5 .55 .6 .0
\definecolor{CTcitation}{rgb}{0,0.5,0} % WebGreen
\definecolor{CTurl}{named}{Maroon} % Maroon
\definecolor{CTtitle}{named}{Maroon} % Maroon {cmyk}{0, 0.87, 0.68, 0.32}
\definecolor{CTlink}{named}{RoyalBlue} % RoyalBlue {cmyk}{1, 0.50, 0, 0}
\definecolor{halfgray}{gray}{0.55} % chapter numbers will be semi transparent .5 .55 .6 .0
\definecolor{webgreen}{rgb}{0,0.5,0}
\definecolor{webbrown}{rgb}{0.6,0,0}
% clean thesis colors
\definecolor{ctmain}{cmyk}{1, .50, .10, .01}
\definecolor{ctaux}{cmyk}{.18, .98, .18, 0}
% Standard colors for links
\definecolor{darkblue}{rgb}{0,0,.6}
\definecolor{darkred}{rgb}{.1,0,0}
\definecolor{darkgreen}{rgb}{0,.5,0}

% Fonts!
%\IfFileExists{bergamo.sty}{\usepackage[osf]{bergamo}}{}% Bembo
%\IfFileExists{chantill.sty}{\usepackage{chantill}}{}% Gill Sans
\setcaptionfont{\footnotesize}

% This is just ignored, assumably overwritten by Tufte-Book
% TODO: Would like to have caption labels to look like in cleanthesis
%%% \usepackage{caption}
%%% -> has to go before document class! See start of flie.
% this is *not* ignored but only works for real \figures and not 
%marginside-figures.
%\captionsetup[figure]{name={WhyDoYouIgnoreMe.},labelfont={color=red}}
% This is ignored
%\captionsetup{font={color=blue},labelfont={bf,sf,color=red}}

% tufte-specific, this line works but modifies the whole caption,
% not only the "Figure X.X:" string:
% \setcaptionfont{\sffamily \color{ctmain} }

% This does not really work (it used to work shortly..)
%% then redefine the caption
%\let\savedCaption=\caption
%\renewcommand*{\caption}[2][foo]{%
%	\normalsize \normalfont \color{green}
%	\savedCaption[#1]{#2}}

% klappt auch nicht wirklich:
%\def\caption{\dosingleempty\doMyCaption}
%def\doMyCaption[#1]#2{%
%   hiho
%   \iffirstargument \savedCaption[#1]{#2} \else \savedCaption{#2} \fi
%}



%% works also, but is unneccessary complicated:
%%\makeatletter
%%\renewcommand*{\@tufte@caption@font}{\sffamily \color{ctmain}}
%%\renewcommand*{\setcaptionfont}[1]{\renewcommand*{\@tufte@caption@font}{#1}}
%%\makeatother

% Fonts:
\usepackage{tgheros} % TEX Gyre Heros (sans-serif, qhv)
\usepackage{libertine} % Libertine as serif font
\usepackage{inconsolata} % monospace font
\normalfont % reset for tt

% Number sections independent of chapters,
% https://tex.stackexchange.com/questions/207968/
\usepackage{remreset}
\makeatletter
\@removefromreset{section}{chapter}
\makeatother
% Which numerals to use for headings
% (without this line, it is "II.4.2" instead of "4.2")
\let\theOldSection\thesection % save for later
\renewcommand{\thesection}{\arabic{section}}

% Count chapters from zero, not one
\setcounter{chapter}{-1} % works, but currently I 
 %dont like that anymore

% A bugfix to display a "zero" when using roman literals for chapter numbering
% https://tex.stackexchange.com/a/211416
\newcommand{\ZeroRoman}[1]{% 0 + \Roman
	\ifcase\value{#1}\relax 0\else% Chapter 0
	\Roman{#1}\fi}% All other chapters
\renewcommand{\thechapter}{\ZeroRoman{chapter}}
%\renewcommand{\thechapter}{\Roman{chapter}} 



\usepackage{graphicx}
\setkeys{Gin}{width=\linewidth,totalheight=\textheight,keepaspectratio}

%%% Needed by the GRMHD paper
\usepackage{multirow}
\usepackage{booktabs}
\usepackage{bm}
\usepackage{cancel}	
%%% Needed by the GRMHD paper

\usepackage[sort]{natbib} % neccessary when not using revtex4

%\ifshowfigures
 % show figures, standard
%\else
% \usepackage[allfiguresdraft,filename,content={draft mode}]{draftfigure}
%\fi

% Circled character in normal text
\newcommand*\circled[1]{\tikz[baseline=(char.base)]{
            \node[shape=circle,draw,inner sep=2pt] (char) {#1};}}
        

\usepackage{anyfontsize} % scale fonts, yes

% "Release mode" in terms of todo notes, type "make release" into your
% command line.
\ifdefined\releasemode
\usepackage[disable]{todonotes}
% Should also disable the listing of todos
\else

% colored boxes for things todo
% Alternative dazu (gut fuer Seitenanmerkungen):
% http://tex.stackexchange.com/a/73418/49958
\usepackage[draft,colorinlistoftodos,prependcaption,textsize=small]{todonotes}  
 % notes showed
\presetkeys{todonotes}{noline}{}

\fi% endif defined \releasemode


% Hyperref setup
\hypersetup{
    pdftitle={Thesis draft},
    pdfauthor={Sven Köppel},
    pdfsubject={phd thesis},
    pdfkeywords={a} {b} {c},
    colorlinks=true,        % no borders
    linkcolor=ctmain,          % color of internal links
    citecolor=ctaux,    % color of links to bibliography
    filecolor=ctaux,      % color of file links
    urlcolor=ctmain,          % color of external links
    bookmarks=true          % show the toc
}

% Color of footnote links (light circled background color)
%\newcommand*\circledBackground[1]{\tikz[baseline=(char.base)]{
%            \node[shape=circle,draw,inner sep=2pt] (char) {#1};}}
%            \node[shape=circle,draw,inner sep=2pt] (char) {#1};}}
%		\node[shape=circle,fill,inner sep=0pt] (char) {#1};}}
% TIKZ circle does not work, but the colorbox works
% This works, but makes it really big
\renewcommand{\thefootnote}{\protect\colorbox{ctmain!5!white}{\protect\textcolor{ctmain}{\arabic{footnote}}}}

% prep of different raising of footnote
\newcommand*\raiseup[1]{%
        \begingroup
        \setbox0\hbox{\footnotesize\strut #1}%
        \leavevmode
        \raise\dimexpr \ht\strutbox - \ht0\box0
        \endgroup
}

% Avoid footnote marks to be superscript
% https://tex.stackexchange.com/a/333024
\makeatletter
% Default:
% \def\@makefnmark{\hbox{\@textsuperscript{\normalfont\@thefnmark}}}
% \renewcommand{\@makefnmark}{\makebox{\normalfont\@thefnmark}}
\renewcommand{\@makefnmark}{\raiseup{\makebox{\normalfont\@thefnmark}}}
\makeatother



% double equations next to each other
% http://tex.stackexchange.com/questions/120670/equations-side-by-side-both-numbers-on-the-right
\NewEnviron{double}[2]
  {\begin{equation*}
     \refstepcounter{equation}\latexlabel{#1}
     \refstepcounter{equation}\latexlabel{#2}
     \BODY % write something like a=b\qquad c=d
     \tag{\ref*{#1}, \ref*{#2}}
   \end{equation*}}
\AtBeginDocument{\let\latexlabel\label}
% ^^ this very line also saves our life in context of tufte-floats.
% Tufte floats are known for eating labels, for instance in (floating) tables.
% Store our global label just to be sure
% (cf https://github.com/Tufte-LaTeX/tufte-latex/blob/master/tufte-common.def#L1246)
% These two did *not* work, but the upper one does. Just use \latexlabel inside
% anything tufte-float-like.
%\newcommand{\globalLabel}[1]{\label{#1}}
%\let\globalLabel=\label

% Backrereferences in Bibliography,
% http://tex.stackexchange.com/questions/15971/bibliography-with-page-numbers
\renewcommand*{\backref}[1]{}
\renewcommand*{\backrefalt}[4]{%
    \ifcase #1 (Not cited.)%
    \or        (Cited on page~#2.)%
    \else      (Cited on pages~#2.)%
    \fi}

% for tufte:
\renewcommand{\subsubsection}[1]{ {\bf #1} }

% Bibliography
\bibliographystyle{plain}
% jetzt: Living Reviews, Vorteil: DOI und ArxiV-Verlinkung, Backlinks.
% http://blog.relativity.livingreviews.org/author-information/citation-guidelines/#bibtex
% \bibliographystyle{livrevrel}
% -> does not eat my bibliographies, probably have to hand-craft the refs!

% Numberswithin was "chapter", but since chapters now have roman literals,
% we use section names.
\numberwithin{equation}{section}
\numberwithin{figure}{section}
\numberwithin{table}{section}
%\setcounter{chapter}{1} % because it seems to start with 0

% **************************************************
% Chapter and Section display
% **************************************************

% TUFTE MODIFICATION: Show numbers,
% https://tex.stackexchange.com/a/96125/49958
% add numbers to chapters, sections, subsections, using the titlesec
% package: http://texdoc.net/texmf-dist/doc/latex/titlesec/titlesec.pdf
\setcounter{secnumdepth}{2}

% inspired by classicthesis at http://www.miede.de/ and CTAN
\newcommand\formatchapter[1]{%
	\vbox to \ht\strutbox{%
		\setbox0=\hbox{\chapterNumber\thechapter\hspace{10pt}\vline\ }%
		\advance\hsize-\wd0 \advance\hsize-10pt\raggedright%
		\spacedallcaps{#1}\vss}}

\PassOptionsToPackage{newparttoc}{titlesec}% newparttoc to write \part to .toc with \numberline

% there is a bug in titlesec to display the numbers, for certain
% latex versions (like at the FIAS and ITP machines).
% what happens is that there are just no numbers shown, nothing else.
% fix from https://tex.stackexchange.com/a/300259
\usepackage{etoolbox}
\makeatletter
\patchcmd{\ttlh@hang}{\parindent\z@}{\parindent\z@\leavevmode}{}{}
\patchcmd{\ttlh@hang}{\noindent}{}{}{}
\makeatother

% Subtitles for chapters: https://tex.stackexchange.com/a/102682
\makeatletter
\newcommand*\nameundef[1]{\expandafter\let\csname #1\endcsname\@undefined}
\newcommand\chaptersubtitle[1]{\def\@chaptersubtitle{#1}}
\newcommand\undefChaptersubtitle{\nameundef{@chaptersubtitle}}


% chapter format
\titleformat{\chapter}[block]%
	{\huge\fontfamily{qhv}\selectfont\raggedright}% raggedright: no hyphenation in title
	{\protect\marginpar{ \begin{tikzpicture}
		\draw[fill,color=ctmain!10!white] (0,0) rectangle (\linewidth,\linewidth);
		\draw[color=ctmain] (1.5cm,4cm) node { \LARGE\smallcaps{Chapter} };
		\draw[color=ctmain,anchor=east] (4.5cm,1.5cm) node {\color{ctmain!60!white}\normalfont\Huge\fontsize{85}{86}\selectfont\thechapter};
	\end{tikzpicture}}}
	{0pt}%
	{}
	[\@ifundefined{@chaptersubtitle}{}{\color{CTsemi}\large\@chaptersubtitle}]% after title body

\makeatother % end of subtitles-containing definition

% Subtitles for chapters. They are not supposed to be shown in the TOC
%   but only on the particular chapter page.
%\def\chaptersubtitle#1{
%	{{\Large\fontfamily{qhv}\selectfont #1}}
%}%
%   \addcontentsline{toc}{chaptersubtitle}{\protect\numberline{}#1}%

%\titleformat{\chapter}[block]%
%  {\huge\rmfamily\itshape\color{red}}% format applied to label+text
%  {
%{\chapterNumber\thechapter}\ \,\hspace{10pt}\vline   
%}% label
%%  	\llap{\colorbox{red}{\parbox{1.5cm}{\hfill\itshape\huge\color{white}\thechapter}}}
%  {2pt}% horizontal separation between label and title body
%  {}% before the title body
%  []% after the title body

\newcommand{\mySectionFont}{\fontfamily{qhv}\selectfont}

% section format
\titleformat{\section}%
  {\fontfamily{qhv}\selectfont\color{ctmain}}% format applied to label+text
  {\llap{\color{black}\hfill{\thesection}\hspace*{10pt}}}% label
  {0pt}% horizontal separation between label and title body
  {}% before the title body
  []% after the title body

% subsection format
\titleformat{\subsection}%
  {\fontfamily{qhv}\selectfont\color{ctmain}}% format applied to label+text
  {\llap{\color{black}\hfill\thesubsection\hspace*{10pt}}}% label
  {0pt}% horizontal separation between label and title body
  {}% before the title body
  []% after the title body 
  
 
  
% **************************************************
% Page headers
% **************************************************

% store the current chapter name, caveats: https://texfaq.org/FAQ-cmdstar
\newcommand{\currentchapter}{}
\let\oldchapter\chapter
\usepackage{suffix}
% hah, apparently this works!
\renewcommand{\chapter}[1]{\oldchapter{#1}\renewcommand{\currentchapter}{#1}}
\WithSuffix\newcommand\chapter*[1]{\oldchapter*{#1}\renewcommand{\currentchapter}{#1}}

% syntactic sugar
\newcommand{\subtitledChapter}[2]{\chaptersubtitle{#1}\chapter{#2}\undefChaptersubtitle}

%\makeatletter
%\renewcommand{\chapter}[1]{
%    \@ifstar
%    {\oldchapter*{#1}\renewcommand{\currentchapter}{#1}}
%	{\oldchapter{#1}\renewcommand{\currentchapter}{#1}}
%}
%\makeatother

% This snippet directly comes from tufte-common.def. We replace
% \plainauthor and \plaintitle to the actual chapter title to have a better
% context
 
\makeatletter
  \ifthenelse{\boolean{@tufte@twoside}}
  { 
     \fancyhead[LE]{\thepage\quad\smallcaps{\newlinetospace{\currentchapter}}}%
     %\fancyhead[RO]{\smallcaps{\newlinetospace{\plaintitle}}\thepage\quad}
     \fancyhead[RO]{\smallcaps{\newlinetospace{\currentchapter}}\quad\thepage}%
     % for development:
     %\fancyhead[RO]{\smallcaps{\newlinetospace{PDF compiled on \today at \currenttime}}\quad\thepage}
  }
  % else: single sided:
  {
    \fancyhead[RE,RO]{\smallcaps{\newlinetospace{\currentchapter
     % (PDF compiled on \today at \currenttime)
      }}\quad\thepage}
  }
\makeatother

%% Redefine the plain page, so we have page numbers on chapter pages (but also plain pages!)
%% SRC: https://tex.stackexchange.com/a/117334/49958 et al
%\fancypagestyle{section-page}{\fancyhead[RE,RO]{\thepage}}
%\usepackage{etoolbox}
%\patchcmd{\chapter}{\thispagestyle{plain}}{\thispagestyle{section-page}}{}{}
% -> does not work!

%% Instead, change the plain style which is used by tufte-common.def. There is still
%% the empty style for empty pages.
\makeatletter
\fancypagestyle{plain}{
  \ifthenelse{\boolean{@tufte@twoside}}
  { 
  	\fancyhead[LE]{\thepage}%
  	%\fancyhead[RO]{\smallcaps{\newlinetospace{\plaintitle}}\thepage\quad}
  	\fancyhead[RO]{\thepage}%
  	% for development:
  	%\fancyhead[RO]{\smallcaps{\newlinetospace{PDF compiled on \today at \currenttime}}\quad\thepage}
  }
  % else: single sided:
  {
  	\fancyhead[RE,RO]{\thepage}
  }
}% end of fancypagestyle
\makeatother

% try this instead:
\fancypagestyle{page}{\fancyhead[RE,RO]{\thepage}}

% **************************************************
% Table of Contents (TOC)
% **************************************************

% Make the TOC fancier: package {etoc}
%
% Idea: Display subsections without page numbers, getting an overall
%       more compact TOC (seen in the Thesis of Reisswig).
%       Then also apply the sans-serif font and colors from the sections
% 
%ftp://ftp.rrzn.uni-hannover.de/pub/mirror/tex-archive/macros/latex/contrib/etoc/etoc-DE.pdf

\iftrue %\iffalse: %disable for the time being
\usepackage[linktoc=all]{etoc}
\usepackage{etoolbox} % for https://tex.stackexchange.com/a/55926
\makeatletter
\newlength{\tocleftmargin}    \setlength{\tocleftmargin}{5cm}
\newlength{\tocrightmargin}   \setlength{\tocrightmargin}{1cm}


% test empty strings https://tex.stackexchange.com/a/377231/49958
\def\ifempty#1{\def\temp{#1} \ifx\temp\empty }
\def\ifShowChapter#1{\ifempty{#1}\else Chapter\fi} % unnumbered capters

% Trick to use another color for subsection links
% -> doesnt work
\newcommand{\setAuxLinkColor}{}%\hypersetup{linkcolor=black}}       

\etocsetstyle{chapter}
   {\begingroup\addvspace{1ex}\parfillskip0pt
	\leftskip\tocleftmargin
	\rightskip\the\tocrightmargin plus 1fil
	\parindent0pt}
    {\bfseries\LARGE\upshape\addvspace{2ex}\leavevmode}
    {\llap{\color{ctmain}\mySectionFont\setAuxLinkColor 
    % This works and allows to disable page numbers for content chapters (with
    % sections). Removed however due to Introduction+Conclusion chapter
    % where numbering is neccessary
    % TODO Disable
    \etocifnumbered{Chapter}{}\hspace{.5em}{\setAuxLinkColor\etocnumber}\hspace{.75cm}}{\mySectionFont\setAuxLinkColor\etocname}
    %\etocifnumbered{}{\hfill\makebox[-\tocrightmargin][l]{\makebox[0pt]{\normalfont\normalsize\mySectionFont\color{ctmain}\large\etocpage}}}
    {\hfill\makebox[-\tocrightmargin][l]{\makebox[0pt]{\bfseries\LARGE\color{ctmain}\mySectionFont\etocpage}}}
	\par}
    {\endgroup}
    
    % Idea of the section display: Mimic how they look like with \titleformat
\etocsetstyle{section}
    {\begingroup\leftskip\tocleftmargin
    	\rightskip\the\tocrightmargin plus 1fil
    	\parindent0pt}
    {\mdseries\large\addvspace{.5ex}\leavevmode}
    {\llap{\mySectionFont\color{black}\etocnumber\hspace{.75cm}}\mySectionFont\color{ctmain}\etocname%
%   {\llap{\etocnumber\hspace{.75cm}}\textit{\etocname}%
	 \hfill\makebox[-\tocrightmargin][l]{\makebox[0pt]{\etocpage}}\par}
   {\endgroup}

% Trick to use another color for subsection links
\renewcommand{\setAuxLinkColor}{\hypersetup{linkcolor=ctmain!70!white}}       
%\newcommand{\resetLinkColor}{\hypersetup{linkcolor=ctmain}} % default...

	\etocsetstyle{subsection}
	{\par\nopagebreak\begingroup
		\leftskip\tocleftmargin \rightskip\@tocrmarg \parfillskip\@flushglue
		\parindent 0pt \normalfont\normalsize\rmfamily\small%\itshape
		% \columnsep1em
		% \begin{minipage}{\dimexpr\linewidth-\leftskip-\rightskip\relax}%
		% \begin{multicols}{2}%
		\color{ctmain!70!white}
		\etocskipfirstprefix}
	{,\allowbreak\,}
	{{\mySectionFont\setAuxLinkColor\etocname}}%\  % ~(\etocpage)
	%\textup{(\etocnumber)}
	{.\par\endgroup}%
	% {.\par\end{multicols}\end{minipage}\par\endgroup}%

\makeatother
% end of crazy etoc TOC redefinition
\fi % endof foobar test

\setcounter{tocdepth}{2} % overwrite tufte
% tocdepth = 1 was useful in beginning

% This stems from
% https://github.com/derric/cleanthesis/blob/master/cleanthesis.sty#L642
\usepackage{tocloft}
% - fixes wrong fonts in the toc, thanks to magnucki
\renewcommand{\cftchapfont}{\usefont{T1}{bch}{b}{n}\selectfont}
\renewcommand{\cfttoctitlefont}{\normalfont}
\renewcommand{\cftloftitlefont}{\normalfont}
\renewcommand{\cftlottitlefont}{\normalfont}

% This works for removing the malformed TOC title
\makeatletter \renewcommand{\@cftmaketoctitle}{} \makeatother

% https://tex.stackexchange.com/a/187871/49958
\newcommand{\phantomsectiontotoc}[1]{%
	\addtocontents{toc}{\protect\contentsline{section}{\protect\numberline{\thesection}#1}{}{}}%
}

% Textual stuff

% for math
\renewcommand{\d}{\mathrm{d}}
\newcommand{\dd}[2]{\frac{\d #1}{\d #2}}
\newcommand{\pp}[2]{\frac{\partial #1}{\partial #2}}
\newcommand{\Lie}{\mathcal{L}}
\newcommand{\gammat}{\tilde\gamma}
\newcommand{\Gtilde}{\tilde\Gamma}
\newcommand{\Ghat}{\hat\Gamma}
\newcommand{\Atilde}{\tilde A}
\newcommand{\TraceFree}[1]{\left[#1\right]^{\mathrm{TF}}}

% Terms coming from...
\newcommand{\NewTerms}[1]{\textcolor{red}{#1}}

% for regular text
\newcommand{\understand}{\circled{?}}


%%% Start of commands defined in the CCZ4 paper

\newcommand{\cf}{{cf.}~}
\newcommand{\ie}{{i.e.,}~}
\newcommand{\eg}{{e.g.,}~}
\newcommand{\ms}{{\rm ms}}
\newcommand{\km}{{\rm km}}
\newcommand{\hz}{{\rm Hz}}
\newcommand{\khz}{{\rm kHz}}
\newcommand{\mpc}{{\rm Mpc}}
\newcommand{\pls}{~~~}
\newcommand{\hG}{{\hat\Gamma^k}}
\newcommand{\gij}{{\tilde\gamma_{ij}}}
\renewcommand{\u}{\boldsymbol{u}}
\newcommand{\halb}{\frac{1}{2}}

\renewcommand{\d}{\mathrm{d}}

\newcommand{\x}{\boldsymbol{x}}
\newcommand{\Q}{\boldsymbol{Q}}
\renewcommand{\u}{\boldsymbol{u}}
\newcommand{\w}{\boldsymbol{w}}
\newcommand{\q}{\boldsymbol{q}}
\newcommand{\F}{\boldsymbol{F}}
\newcommand{\f}{\boldsymbol{f}}
\newcommand{\g}{\boldsymbol{g}}
\newcommand{\h}{\boldsymbol{h}}
%\renewcommand{\v}{\boldsymbol{v}}
\newcommand{\B}{\boldsymbol{B}}
\newcommand{\emm}{m}

% Naming a computer code name
\newcommand{\code}[1]{\texttt{#1}}
%%% End of commands defined in the CCZ4 paper

\providecommand{\tr}{\operatorname{tr}} % trace

\providecommand{\cf}{cf.,~}
\providecommand{\ie}{i.e.,~}
\providecommand{\eg}{e.g.,~}
\providecommand{\dd}{\mathrm{d}}
\providecommand{\mev}{\,{\rm MeV}}
\providecommand{\s}{\rm s}
\providecommand{\ms}{\rm ms}
\providecommand{\km}{\,{\mathrm{km}}}
\providecommand{\gcm}{\,{\mathrm{g}/\mathrm{cm}^{3}}}
\providecommand{\sol}{\odot}
\providecommand{\Msun}{M_{\odot}}
\providecommand{\Msol}{M_{\odot}}
\providecommand{\tov}{\mathrm{TOV}}
\providecommand{\Mtov}{M_\tov}
\providecommand{\tautov}{\tau_{\tov}}
\providecommand{\Rtov}{R_\tov}
\providecommand{\Mth}{M_\textrm{th}}
\providecommand{\sys}{{\mathrm{sys}}}
\providecommand{\Ctov}{\mathcal{C}_{\rm{TOV}}}

% for Tensortemplates:
\providecommand{\TT}[2]{\mathtt{#1}_{\langle #2 \rangle}}


% TODO:
% Bibtex distinguish between aeireferences, web publications, math references,
% cs references, qgr references? Or probably with a label in the display?
% cf.
% 
%http://packages.oth-regensburg.de/ctan/macros/latex/contrib/bibtopic/bibtopic.pdf

% numbering equations in an tabular array, used in aderdg-foccz4.tex:
% src: https://tex.stackexchange.com/a/66838/49958
\newcommand{\tagarray}{%
	\mbox{}\refstepcounter{equation}%
	$(\theequation)$%
}


% Shorthands for the list of figures/tables, kind of label markers
\newcommand{\by}[1]{Author: #1}
\newcommand{\publishedIn}[1]{Published in \cite{#1}}
\newcommand{\modifiedAfter}[1]{Modified after \cite{#1}}
\newcommand{\colorizedAfter}[1]{Colorized from \cite{#1}}
\newcommand{\ownPub}[1]{Part of own publication \cite{#1}}
\newcommand{\exclusive}{Exclusive material.}

% **************************************************
% Metrics
% **************************************************
%% Thought about adding a list of metric (number of total equations,
%% etc.), but it seems to be a bit self-praise...
%
%\usepackage{totcount,lastpage}
%\newtotcounter{eqnnum}
%
%% Count sections
%\regtotcounter{section}
%%\total{section}
%
%% Count citations
%\newtotcounter{citnum}
%\def\oldbibitem{} \let\oldbibitem=\bibitem
%\def\bibitem{\stepcounter{citnum}\oldbibitem}
%%\totalheightof{citenum}

% 
