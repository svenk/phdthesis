
\section[
  Aspects of input and output on AMR and DG
]{Aspects of input and output on adaptive meshes and discontinous
  Galerkin methods}
For completeness, this section presents a few aspects of input and output in
a numerical time evolution code, especially in the context of dynamical AMR
and DG. These features were implemented for the \code{ExaHyPE} code.

\subsection{In-situ initial data}
Reading initial data on an AMR grid is more challenging then on a predefined
grid, as local features in the initial data shall already be resolved, not only
features which appear after certain time during the evolution. In order to
setup an AMR grid hierarchy which adaptively adopts to non-analytical initial
data, an AMR grid code needs to evaluate the refinement criterion on the ID.

One way to do this is to evaluate the initial state $Q_0(\vec x)$ on subsequent
grid levels until the required accuracy is gained. At the same time, in the
context of parallelism, an AMR code has to load balance the spawned spacetree
over the available processors.

% SOURCE: 
%http://sven.köppel.org/uni/ws1617/2017-04-03-ExaHyPE-ReviewMeeting-FIAS-SK/
% Another source: ExaHyPE guidebook
\begin{marginfigure}
	\includegraphics[width=\textwidth]{dg-fv-id-derivatives/01_initial.pdf}
	\includegraphics[width=\textwidth]{dg-fv-id-derivatives/02_projection.pdf}
	\includegraphics[width=\textwidth]{dg-fv-id-derivatives/03_derivative.pdf}
	\includegraphics[width=\textwidth]{dg-fv-id-derivatives/04_fd_local_derivs.pdf}
	\caption[
	FV vs DG derivatives for ID, drawn with Inkscape, \exclusive
	]{Cartoon of FD vs. DG derivatives for initial data in one dimension.
		Each panel shows the same exemplaric patch, where a fifth order DG polynomial
		is embedded on a Euler-Legendre nodal basis (\ie the filled circles indicate the
		spatial positions where the field values are stored).
		First panel: The exemplaric field solution which is subsequently projected
		onto the DG polynomial (green, second panel) and subsequently differentiated
		(blue, third panel). In contrast, the last panel shows the cartoon of a
		finite differencing stencil for off-grid computation of the initial data
		around a requested point. Here, the initial data have to be provided at
		each bullet, \ie have to be provided five time as much as in the DG derivative
		case.
	}
	\label{fig:dg-fv-id-derivatives}
\end{marginfigure}

The paradigm shift from a classical block-regular code (such as \code{Carpet}) towards
an AMR code (such as \code{ExaHyPE}) shall be demonstrated on an example: As per
definition, a-priori there is no grid, the evaluation of an initial data
(ID)
subroutine has to happen locally for a given spatial point $\vec x$. Typically,
within this thesis, initial data itself are the numerical solution of an
elliptic PDE and therefore present on a certain grid. Reading in the data from
the ID code to the time evolution code typically results in interpolating
between two grids.

As soon as non-local computations are neccessary, this attemp fails. An example
are initial data available as solution of a second order elliptic PDE but
to be evolved with a first order formulation of that PDE (as with the 
formulation of Einstein field equations presented in 
section~\ref{sec:fo-ccz4}). In this case,
auxilliary fields $a_i = \partial_i \phi$ for some field $\phi$ have to be
computed, and the derivative operator is nonlocal by definition.

The two solutions are either to precompute the auxiliary field also on the
initial data code grid, or to compute them on-demand (in-situ). The later
option is preferable to have point-wise access to the initial data in the
time evolution at any time. For computing derivatives, in \code{ExaHyPE} we chose
two approaches (Figure~\ref{fig:dg-fv-id-derivatives}): Patchwise, using
the DG derivatives (which can be thought of a semi-local attempt) or off-grid with finite
differences.

\subsection{In-situ postprocessing}
Within the time evolution codes presented so far, it is common practice to
write out the system state also during the time evolution (and not only at
its arbitrary end). This outputting is done repeatedly by a time or iteration
criterion (formulated for instance as ``write every $N$th iteration'' or 
``write every $\Delta t$''), but any other kind of \emph{query based plotting}
can be imagined. \footnote{We use the terms \emph{plotting}, \emph{writing},
	\emph{dumping} of data synonymously.
} The concept of a writeout is to store 
the current \emph{snapshot}
$\u(t,\x)$ for a given time $t$ permanently, while it is neccessarily erased
at one point in an evolution scheme which thrives to have a constant memory
need during time evolution.

When it comes to the writeout phase, any kind of \emph{postprocessing} can be
done. This term shall be defined by a local mapping
$f: \mathrm \Omega^n \to \mathrm \Omega^m$ which maps the solution vector
$\u(\x)$ at a fixed given time $t$
to a vector of written quantities $\boldsymbol w(\x)$ on the spatial 
$d$-dimensional simulation domain $\Omega$\footnote{
  This mapping can be non-locally extended to allow for derivatives, such
  as $\boldsymbol w(\x) = f(\u(\x), \partial_i \u(\x), \dots)$.
}. One can imagine that
$\boldsymbol w = \boldsymbol f(\u)$ is computed either during the time
evolution (``online''), in order to write out \emph{only} the mapped state
vector, or afterwards, so $\u$ is written and $\boldsymbol f(\u)$ is computed
in an offline postprocessing step. 
Online post processing might save a lot of computing ressources
if $m \ll n$ and $\boldsymbol f$ is computationally cheap.

Reductions or volume/surface integrals can be modeled with a mapping
$g: \mathrm R^{d\times n} \to \mathrm R^{e \times m}$, where $e<d$ is the
dimensional reduction and again $m<n$ stands for a lossy mapping of the
solution vector to less degrees of freedom. For the highly symmetric spacetimes
of compact objects, as presented in the next chapters, simple 1D and 2D cuts
of the Cartesian coordinate systems on the axis are typical dimensional
reductions, while $L_1$, $L_2$ and $L_\infty$  (volume) integrals are evaluated
on the whole space and provide ``0D'' scalar output. Surface integrals over
an arbitrary surface (defined by its normal vector field $S^i$) are technically
more complex to implement but nevertheless fall into the class of mappings $g$.
\todo[color=green]{
  In-Situ postprocessing: Could give examples, such as
  ADM Mass, Angular momentum, Apparent Horizon
  detectors. Something different are tracers/trackers (nonlocal)
}

\subsection{Visualization of ADER-DG simulations}
When simulation data should be stored permanently, a well-defined file format
provides interoperability between different (possibly postprocessing) codes.
The \code{ExaHyPE} grid structure can be casted as \emph{block regular}:
The DG polynomials in each patch can be sampled on a unigrid, without loss
of accuracy (\ie by keeping the number of nodal points the same). This is
the basic idea of the implementation of the \code{CarpetHDF5} file format
within \code{ExaHyPE},
which was originally developed for \code{Carpet} \cite{Reisswig:2010}.

While the \code{CarpetHDF5} \footnote{
  \code{HDF} is short for ``Hierarchical Data Format'' and is a container file format
  for scientific data, especially multidimensional arrays. In principle,
  the propsed discussion is independent of the container file format.
  See \cite{exahype-guidebook} for details about the file formats.
  
  In contrast to the formats discussed in the main text,
  \code{VTK} (``Visualization Toolkit'') is a widely supported
  standarized format for describing higher dimensional geometries.
  \code{ExaHyPE} can write \code{VTK} files, but these files are only
  optimized for interoperability.
} file format is tailored to the memory layout of
\code{Carpet}, where block regular \emph{components} are sized in order to
match one parallel rank (\ie a MPI process), in \code{ExaHyPE} the patches
are kept intentionally small to hold a single DG polynomial. Therefore,
the number of patches per dimension is typically one to two orders of
magnitude larger in \code{Carpet} grids compared to \code{ExaHyPE} grids,
and the amount of \emph{meta data} in the \code{CarpetHDF5} file format
dominates over the \emph{payload data}. In this context, the term ``meta
data'' refers to the description of the grid structure and component
geometry, while ``payload data'' are the actual field values. This
property makes the \code{CarpetHDF5} format especially inattractive
in \code{ExaHyPE} as a volumetric file format
(\ie in three spatial dimensions). \footnote{
  The structure-of-array/array-of-structure representation of fields
  is the underlying difference between \code{Carpet} (basically
  storing $(Q_0(\vec x), Q_1(\vec x), \dots)$ and \code{ExaHyPE}
  (storing $\vec Q(\vec x_0), \vec Q(\vec x_1), \dots)$).
  This is also the main difference between the Carpet file format and
  the tailored Peano block regular file format~\cite{exahype-guidebook}.
}


\section{Summary}
In Chapter~\ref{chapter:numerics}, all the ingredients for a new code developed
within this thesis were proposed. At the very heart, it is a coupled
evolution scheme which uses a local ADER-DG scheme which is by construction
communication avoiding and thus suitable for the next generation of
machines, in contrast to traditional schemes, as well as an undemanding
traditional Finite Volume scheme, which is however time variation diminishing
and suitable to evolve a solution at low order. The infrastructure
for running three-dimensional astrophysical simulations was presented, and it
will be brought to life in the subsequent chapters \vref{chapter:gr}
(with general relativity) and chapter \vref{chapter:hydro}
(with hydrodynamics).






