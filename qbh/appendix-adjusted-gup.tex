\newpage
\section{Adjusted GUP profiles}\label{apx:adjusted-gup}
Figures \ref{fig:adj-gup-extra1} and \ref{fig:adj-gup-extra2} are provided
as an alternative to display Figure \vref{fig:adjusted-gup-n8-indepth}.
The different metric functions $g_{00}(M)$ are plotted here in a single
panel (for different $M$). The units are given in such a way that
the large BH remnant is (arbitarily) set to $r_1=r_*$.

\begin{figure}[h!]
	% This new figure of Sven replaces Marcos gup-figures/g00kempf.pdf
	\includegraphics[width=\linewidth]{svens-new-python-figures/adjusted-gup-n8-indepth.pdf}
	\caption[
	Adjusted GUP potential for $n=8$, selfmade (Python), \exclusive
	]{
		Upper panel: Metric component $g_{00}$ for the 
		adjusted GUP in $n=8$ spatial dimensions
		over radius, for different
		masses (each line corresponds to a particular mass).
		Lower panel: Hawking temperature for a given horizon $r_H$ for the
		adjusted GUP in $n=8$ spatial dimensions, compared to the
		corresponding ordinary temperature of the Schwarzschild-Tangherlini
		metric. Radii are given in fundamental Planck units $r_*$, $\beta$ is
		tuned for self-completeness (see main text).
		
		Here, the two panels are put ontop of each other with a single
		radius axis, which however has a different meaning in the different
		panels. The diagram can be read as that for a metric profile (upper
		panel), at the outer horizon by going to the bottom panel, the
		temperature can be read off.
	}\label{fig:adj-gup-extra1}
\end{figure}
\vspace*{-1cm}
\begin{figure}[h!]
	\includegraphics[width=\linewidth]{svens-new-python-figures/adjusted-gup-n9-indepth.pdf}
	\caption[
	Adjusted GUP potential for $n=9$, selfmade (Python), \exclusive
	]{
		A diagram similar to Figure~\protect\ref{fig:adj-gup-extra1} 
		but in $n=9$ spatial dimensions. Adding only one dimension made the situation
		much more dramatic, as there is a third self-complete region and the
		regions are seperated by very high temperature peaks $\sim 10^4 T_*$.
		It is likely that the semiclassical description fails to describe the
		inner part of the black hole already below $r<r_*$.
	}\label{fig:adj-gup-extra2}
\end{figure}